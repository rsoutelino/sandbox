\documentclass{beamer}
\usetheme{AnnArbor}
% \usetheme{Singapore}
\usepackage[utf8x]{inputenc}
% \usepackage{bookman} 
\usepackage[T1]{fontenc} 
\usepackage{textcomp}
\usepackage{graphics}
\usepackage{animate}

\mode<presentation>

\setbeamertemplate{navigation symbols}{}

\definecolor{darkblue}{RGB}{22,52,136}
\definecolor{midblue}{RGB}{72,118,162}
\definecolor{midgray}{RGB}{127,127,127}
\definecolor{lightgray}{RGB}{204,204,204}
\definecolor{yell}{RGB}{120,10,2}

\setbeamercolor{alerted text}{fg=yell}
\setbeamercolor*{palette primary}{fg=black,bg=midblue}
\setbeamercolor*{palette secondary}{fg=darkblue,bg=lightgray}
\setbeamercolor*{palette tertiary}{bg=darkblue,fg=white}
\setbeamercolor*{palette quaternary}{fg=darkblue,fg=midgray}

\setbeamercolor*{sidebar}{fg=darkblue,bg=orange!75!white}

\setbeamercolor*{palette sidebar primary}{fg=darkblue!10!black}
\setbeamercolor*{palette sidebar secondary}{fg=white}
\setbeamercolor*{palette sidebar tertiary}{fg=darkblue!50!black}
\setbeamercolor*{palette sidebar quaternary}{fg=yellow!10!orange}

%\setbeamercolor*{titlelike}{parent=palette primary}
\setbeamercolor{titlelike}{bg=lightgray,fg=darkblue}
\setbeamercolor{frametitle}{bg=lightgray,fg=darkblue}
\setbeamercolor{frametitle right}{bg=midgray,fg=black}

\setbeamercolor*{separation line}{}
\setbeamercolor*{fine separation line}{}

%\usebackgroundtemplate{\includegraphics[width=\paperwidth]{logoio.png}}

\mode
<all>


\title{\bf{On the dynamics of the Brazil Current site of origin}}

\subtitle{São Paulo, Nov 13, 2012} 

\author[Rafael Soutelino]{Student: {\bf Rafael Soutelino}\\
Advisor: Ilson da Silveira | IO-USP\\
Co-Advisor: Avijit Gangopadhyay | SMAST-UMASSD}

\institute[IOUSP]{
\includegraphics[height=1.5cm]{lado.png} 
\includegraphics[height=1.6cm]{SMASTlogo_umass_circle.pdf} 
} 
% ----O teu logo aí em cima e o do io ali em baixo -----
\date{}
\mode<beamer>{\logo{\includegraphics[height=1cm]{IO.jpg}}}

\AtBeginSubsection[] {
  \begin{frame}<beamer> \frametitle{Summary}
    \tableofcontents[currentsubsection]
  \end{frame}
} \setbeamercovered{highly dynamic}

\begin{document}

% ------------------------------------------------------------
\usebackgroundtemplate{\includegraphics[width=\paperwidth]{header.png}}
\begin{frame}
\vspace{1.5cm}
  \titlepage
\end{frame}
\usebackgroundtemplate{}
% ------------------------------------------------------------
\begin{frame}
  \frametitle{Summary}
  \tableofcontents
\end{frame}
% ------------------------------------------------------------
% daqui para frente é por tua conta e risco
% ------------------------------------------------------------
\section{Introduction}

\subsection{Study Area, Literature Review}
% -------------------------------------------------
\frame
{
  \frametitle{Presenting the study area}
\includegraphics[width=4cm,keepaspectratio=true]{figures/atlantic.pdf}
\includegraphics[width=8cm,keepaspectratio=true]{figures/stommel.pdf}

\footnotesize{Tomczak - Regional Oceanography}
}

% -------------------------------------------------
\frame
{
  \frametitle{Peculiarities of the Brazil Current System}
\includegraphics[width=6cm,keepaspectratio=true]{figures/brasil_3D-en.pdf}\hspace{1cm}
\includegraphics[width=4cm,keepaspectratio=true]{figures/esq_transports.pdf}\\
\footnotesize{According to {\it Stramma \& Englang (1999)}}
}

% -------------------------------------------------
\frame
{
  \frametitle{Peculiarities of the Brazil Current System}
\begin{center}
\includegraphics[width=9cm,keepaspectratio=true]{figures/topog.png}\\
\footnotesize{Mapped from ETOPO 2'}
\end{center}
}


% -------------------------------------------------
\frame
{
  \frametitle{Synoptic circulation review: north of BiSEC}
\includegraphics[width=6cm,keepaspectratio=true]{figures/schott_etal2005.png}\hspace{1cm}
\includegraphics[width=4cm,keepaspectratio=true]{figures/esq_transports.pdf}

\footnotesize{\it Schott et al. (2005)}
}

% -------------------------------------------------
\frame
{
  \frametitle{Synoptic circulation review: south of BiSEC}
\includegraphics[width=6cm,keepaspectratio=true]{figures/silva_19S.png}\hspace{1cm}
\includegraphics[width=4cm,keepaspectratio=true]{figures/esq_transports.pdf}\\
\footnotesize{\it Silva et al. (2009)}
}

% -------------------------------------------------
\frame
{
  \frametitle{Synoptic circulation review - mesoscale eddies}
\includegraphics[width=11cm,keepaspectratio=true]{figures/campos2006.pdf}

\footnotesize{\it Campos (2006)}
}


% -------------------------------------------------
\frame
{
  \frametitle{Typical mesoscale features}
\begin{center}
\includegraphics[width=8cm,keepaspectratio=true]{figures/brazil_cartoon.pdf}
\end{center}

\footnotesize{According to previous research}
}


\subsection{Scientific Hypothesis and Objectives}
% -------------------------------------------------
\frame
{
  \frametitle{Scientific Hypothesis and Objectives}
\begin{footnotesize}
\begin{alertblock}{\textcolor{black}{\bf Hypothesis}}
  \begin{enumerate}
    \item{The BC site of origin and surface SEC bifurcation consists of a broad and eddy-dominated
          area, instead of a narrow limit between different flow regimes.}
    \item{The rich mesoscale activity observed at the BC site of origin is caused by a combined
          effect of jet-topography and jet-jet interaction, where the existence of a western
          boundary undercurrent plays a key role.}
  \end{enumerate}
\end{alertblock}

\begin{alertblock}{\textcolor{black}{\bf Objectives}}
  \begin{enumerate}
    \item{Build a regional synoptic picture of the origin, formation and organization of the BC as a 
          boundary current, as well as of the mesoscale activity between 10-20$^\circ$S.}
    \item{Identify the roles of topography and BC-NBUC shear on the eddy activity in the near-surface
          flow at the BC formation region.}
  \end{enumerate}
\end{alertblock}
\end{footnotesize}

}

% \subsection{Broad Impacts}
% % -------------------------------------------------
\frame
{
  \frametitle{Broad impacts}
\begin{small}
\begin{block}{\textcolor{black}{\bf Motivation to study synoptic circulation at BC origin}}
\begin{itemize}
 \item [$\checkmark$] \textcolor{blue}{Verify which typical mesoscale features occurs;}
% \vspace{.2cm}
 \item [$\checkmark$] \textcolor{blue}{Accurately represent real dynamics in numerical simulations, important
to process understanding, nowcasting and forecasting;}
% \vspace{.2cm}
 \item [$\checkmark$] Annual and inter-annual variability - meridional overturning circulation;
% \vspace{.2cm}
%  \item [$\checkmark$] Large scale heat poleward transport - climate change mechanisms;
% \vspace{.2cm}
 \item [$\checkmark$] Shelf-slope interactions and exchanges;
% \vspace{.2cm}
 \item [$\checkmark$] Sedimentological implications of deep currents;
% \vspace{.2cm}
 \item [$\checkmark$] Biological implications of eddy activity.
\end{itemize}
\end{block}
\end{small}
}




\section{Part I: Regional description of the study area}

\subsection{{\it In situ} velocity evidences}


% -------------------------------------------------
\frame
{
  \frametitle{Recent In Situ velocity data}
\begin{center}
\includegraphics[width=12cm,keepaspectratio=true]{figures/cruises.pdf}
\end{center}
\vspace{-0.5cm}
\footnotesize{\it Soutelino et al. (2011)}
}


% -------------------------------------------------
\frame
{
  \frametitle{Recent In Situ velocity data}
\begin{columns}
\begin{column}{15cm}
\hspace{-1cm}
\includegraphics[width=14.5cm,keepaspectratio=true]{figures/obs.pdf}
\end{column}
\end{columns}
\vspace{-0.5cm}
\footnotesize{\it Soutelino et al. (2011)}
}

\subsection{OGCM velocity evidences}


% -------------------------------------------------
\frame
{
  \frametitle{OGCM results - OCCAM}
\begin{columns}
\begin{column}{15cm}
\hspace{-1cm}
\includegraphics[width=14.5cm,keepaspectratio=true]{figures/occam.pdf}
\end{column}
\end{columns}
\vspace{-0.5cm}
\footnotesize{\it Soutelino et al. (2011)}
}

\subsection{Satellite velocity evidences}


% -------------------------------------------------
\frame
{
  \frametitle{Satellite altimetry results}
\includegraphics[width=11cm,keepaspectratio=true]{figures/aviso_time_series.pdf}
}


% -------------------------------------------------
\frame
{
  \frametitle{Summary of Part I}
\begin{itemize}
\item{According to recent {\it in situ} observations, satellite altimetry and OGCM results, the
flow at the BC formation region is weak and dominated by mesoscale eddies.}
\item{Three large anticyclonic mesoscale eddies are possibly steady or recurrent at the region. } 
\end{itemize} 

\begin{alertblock}{\textcolor{black}{\bf Hypothesis 2}}
  \begin{itemize}
    \item{The rich mesoscale activity observed at the BC site of origin is caused by a combined
          effect of jet-topography and jet-jet interaction, where the presence of a western
          boundary undercurrent (NBUC) plays a key role.}
  \end{itemize}
\end{alertblock}
}

\section{Part II: Dynamics of the study area}

\subsection{Methodology}

% -------------------------------------------------
\frame
{
  \frametitle{Strategy to test the hypothesis: The FORMS technique}
% \begin{center}
\includegraphics[width=7cm,keepaspectratio=true]{figures/FORMS.png}\hspace{0.5cm}
\includegraphics[width=3.5cm,keepaspectratio=true]{figures/esq_transports.pdf}
% \end{center}
}

% -------------------------------------------------
\frame
{
  \frametitle{Feature Models Formulation - Velocity}

\begin{columns}
  \begin{column}{6cm}

\begin{footnotesize}
\begin{eqnarray*}
    \mathcal{V}(x,y,z) = v(y,z)\ exp \left[- \frac{(x-x_c)^2}{2 \delta^2} \right],
    \label{eq:main}
\end{eqnarray*}


\begin{eqnarray*}
    \label{eq:nbuc_alongstream}
    \hspace{-0.5cm} v_{NBUC} = v_c(y) 
    \begin{cases} 
        exp \left[ - \frac{(z-z_c(y))^2}{2\delta_t^2} \right], \  \text{at}  \ z_c < z < 0 \\
        \ \\
        exp \left[ - \frac{(z-z_c(y))^2}{2\delta_b^2} \right], \  \text{at}  \ z_b < z < z_c \\
    \end{cases}
\end{eqnarray*}

\begin{eqnarray*}
    v_{BC} = v_c(y) . exp \left[ - \frac{(z-z_c)^2}{2\delta_{bc}^2} \right], \ \ at \ \ z < 0
    \label{eq:bc_alongstream}
\end{eqnarray*}

\end{footnotesize}

  \end{column}

  \begin{column}{5cm}
    \includegraphics[width=5cm,keepaspectratio=true]{figures/bc-nbuc_diagram_south.pdf}
  \end{column}

\end{columns}
}

% -------------------------------------------------
\frame
{
  \frametitle{Feature Models Formulation - Velocity}

    \begin{tiny}
    \begin{center}
    % use packages: array
    \renewcommand{\arraystretch}{1.2}
    \renewcommand{\tabcolsep}{2mm}
    \begin{tabular}{cccccccc}
        \hline
        \hline
        Feature & Relevant studies \\
        \hline
        NBUC (10$^\circ$-15$^\circ$S) & Silveira et al. (1994) \\
               & Stramma et al. (1995) \\
               & Stramma \& Schott (1999) \\
               & Boebel et al. (1999) \\
               & Silveira et al. (2000) \\
               & Dengler et al. (2004) \\
               & Schott et al. (2005) \\
               & Rodrigues et al. (2007) \\
        \hline
        IWBC/NBUC (15$^\circ$-25$^\circ$S) & Stramma \& England (1999) \\
               & Boebel et al. (1999) \\
               & Silveira et al. (2000) \\
               & Silveira et al. (2004) \\
               & Rodrigues et al. (2007) \\
               & Schmidt et al. (2007) \\
        \hline
        BC (10$^\circ$-28$^\circ$S)& Stramma et al. (1990) \\
           & Miranda \& Castro (1981) \\
           & Silveira et al. (2000) \\
           & Silveira et al. (2004) \\
           & Campos (2006) \\
           & Rodrigues et al. (2007) \\
           & Soutelino et al. (2011) \\
        \hline
    \end{tabular}
    \end{center}
    \end{tiny}
}

% -------------------------------------------------
\frame
{
  \frametitle{Feature Models Formulation - Velocity}

    \begin{footnotesize}
    \begin{center}
    % use packages: array
    \renewcommand{\arraystretch}{1.5}
    \renewcommand{\tabcolsep}{2mm}
    \begin{tabular}{cccccccc}
        \hline
        \hline
        Feature & $x_c$ & $\delta$ & $z_c$ & $\delta_t/2$ & $\delta_b/2$ & $v_c$ & $T$\\
        \hline
        NBUC-S & 160 $km$ & 100 $km$ & 500 $m$ & 100 $m$ & 360 $m$ & 20 $cm\ s^{-1}$ & 9 $Sv$ \\
        NBUC-N & 160 $km$ & 100 $km$ & 200 $m$ & 100 $m$ & 360 $m$ & 50 $cm\ s^{-1}$ & 23 $Sv$ \\
        \hline
        BC-S & 160 $km$ & 100 $km$ & 0 $m$ & $\times$ & 150 $m$ & 20 $cm\ s^{-1}$ &  3$Sv$ \\
        BC-N & 160 $km$ & 100 $km$ & 0 $m$ & $\times$ & 0 $m$   & 0 $cm\ s^{-1}$ & 0 $Sv$ \\
        \hline
    \end{tabular}
    \end{center}
    \end{footnotesize}
}


% -------------------------------------------------
\frame
{
  \frametitle{Feature Models Formulation - Mass Field}

\begin{small}
\begin{eqnarray*}
    f_0 \frac{\partial v}{\partial z} = - \frac{g}{\bar{\rho}} \frac{\partial \rho}{\partial x}, 
    \label{eq:thermal_wind}
\end{eqnarray*}

\begin{eqnarray*}
    \rho(x,z) = \rho(0,z) - \frac{f_0 \bar{\rho}}{g} \int_0^L \frac{\partial v}{\partial z} dx, 
    \label{eq:rho}
\end{eqnarray*}

\begin{eqnarray*}
    T(x,z) = \frac{\frac{-\rho}{\bar{\rho}} + 1 + \bar{\beta}S_0(z)}{\bar{\alpha}}.
    \label{eq:temp}
\end{eqnarray*}

\begin{eqnarray*}
    S(x,z) = S_0(z) - 10^{-2} T(x,z).
    \label{eq:salt}
\end{eqnarray*}

\end{small}

\begin{small}
Following {\it Schmidt at al. (2007) and Fernandes (2007)}
\end{small}
}


% -------------------------------------------------
\frame
{
  \frametitle{Feature Models vertical structure}
\begin{center}
% \vspace{-0.3cm}
\includegraphics[width=6cm,keepaspectratio=true]{figures/bc-nbuc_diagram_south.pdf}
\includegraphics[width=6cm,keepaspectratio=true]{figures/bc-nbuc_diagram_rho_south.pdf}

\end{center}
}


% -------------------------------------------------
\frame
{
  \frametitle{Feature Models vertical structure}
\begin{center}
% \vspace{-0.3cm}
\includegraphics[width=6cm,keepaspectratio=true]{figures/bc-nbuc_diagram_north.pdf}
\includegraphics[width=6cm,keepaspectratio=true]{figures/bc-nbuc_diagram_rho_north.pdf}
\end{center}
}

% -------------------------------------------------
\frame
{
  \frametitle{Feature Models Structure}
\begin{center}
% \vspace{-0.3cm}
\animategraphics[autoplay,loop,height=6cm]{1}{figures/transect}{3}{17}
\animategraphics[autoplay,loop,height=6cm]{1}{figures/frame}{3}{17}
\end{center}
}

% -------------------------------------------------
\frame
{
  \frametitle{Feature Models Structure}
\begin{center}
% \vspace{-0.3cm}
\animategraphics[autoplay,loop,height=6cm]{1}{figures/transect}{3}{17}
\animategraphics[autoplay,loop,height=6cm]{1}{figures/rho}{3}{17}
\end{center}
}

% -------------------------------------------------
\frame
{
  \frametitle{Feature Models vertical structure}
\begin{center}
% \vspace{-0.3cm}
\includegraphics[width=6cm,keepaspectratio=true]{figures/sec_rho.pdf}
\includegraphics[width=6cm,keepaspectratio=true]{figures/sec_vel.pdf}
\end{center}
}

% % -------------------------------------------------
% \frame
% {
%   \frametitle{3-D Idealized Field Representation}
% \begin{center}
% % \vspace{-0.3cm}
% \includegraphics[width=5cm,keepaspectratio=true]{figures/esq_transports.pdf}
% \end{center}
% }


% -------------------------------------------------
\frame
{
  \frametitle{Feature Models horizontal structure}
\begin{center}
% \vspace{-0.3cm}
\includegraphics[width=12cm,keepaspectratio=true]{figures/fm_vel_horiz.pdf}
\end{center}
}

% -------------------------------------------------
\frame
{
  \frametitle{Feature Models horizontal structure}
\begin{center}
% \vspace{-0.3cm}
\includegraphics[width=12cm,keepaspectratio=true]{figures/fm_rho_horiz.pdf}
\end{center}
}


% % -------------------------------------------------
% \frame
% {
%   \frametitle{Feature Models horizontal structure}
% \begin{center}
% % \vspace{-0.3cm}
% \includegraphics[width=12cm,keepaspectratio=true]{figures/vbat_zeta_horiz.pdf}
% \end{center}
% }

% -------------------------------------------------
\frame
{
  \frametitle{Model Domain / Grid}
\begin{center}
% \vspace{-0.3cm}
\includegraphics[width=12cm,keepaspectratio=true]{figures/model_grid.pdf}
\end{center}
}

% -------------------------------------------------
\frame
{
  \frametitle{Sensitivity experiments setup}
\begin{center}
% \vspace{-0.3cm}
\includegraphics[width=12cm,keepaspectratio=true]{figures/experiments.pdf}
\end{center}
}



\subsection{Results}

% -------------------------------------------------
\frame
{
  \frametitle{CONTROL Experiment - snapshots}
\begin{center}
% \vspace{-0.3cm}
\includegraphics[width=5cm]{figures/phd15_vel_day1_100m.pdf}
\includegraphics[width=5cm]{figures/phd15_vel_day10_100m.pdf}
\end{center}
}

% -------------------------------------------------
\frame
{
  \frametitle{CONTROL Experiment - snapshots}
\begin{center}
% \vspace{-0.3cm}
\includegraphics[width=5cm]{figures/phd15_vel_day20_100m.pdf}
\includegraphics[width=5cm]{figures/phd15_vel_day30_100m.pdf}
\end{center}
}

% % -------------------------------------------------
% \frame
% {
%   \frametitle{CONTROL Experiment - 3 month averages}
% \begin{center}
% % \vspace{-0.3cm}
% \includegraphics[width=5cm]{figures/phd15_vel_day90-120_100m.pdf}
% \includegraphics[width=5cm]{figures/phd15_vel_day150-180_100m.pdf}
% \end{center}
% }

% % -------------------------------------------------
% \frame
% {
%   \frametitle{CONTROL Experiment - 3 month averages}
% \begin{center}
% % \vspace{-0.3cm}
% \includegraphics[width=5cm]{figures/phd15_vel_day210-240_100m.pdf}
% \includegraphics[width=5cm]{figures/phd15_vel_day270-300_100m.pdf}
% \end{center}
% }

% -------------------------------------------------
\frame
{
  \frametitle{CONTROL Experiment - Run average}
\begin{center}
% \vspace{-0.3cm}
\includegraphics[width=5cm]{figures/phd15_vel_day30-360_1m.pdf}
\includegraphics[width=5cm]{figures/phd15_vel_day30-360_100m.pdf}
\end{center}
}

% -------------------------------------------------
\frame
{
  \frametitle{CONTROL Experiment - Run average}
\begin{center}
% \vspace{-0.3cm}
\includegraphics[width=5cm]{figures/phd15_vel_day30-360_200m.pdf}
\includegraphics[width=5cm]{figures/phd15_vel_day30-360_400m.pdf}
\end{center}
}

% -------------------------------------------------
\frame
{
  \frametitle{S1 Experiment - Run Average}
\begin{center}
% \vspace{-0.3cm}
\includegraphics[width=5cm]{figures/phd16_vel_day30-360_1m.pdf}
\includegraphics[width=5cm]{figures/phd16_vel_day30-360_400m.pdf}
\end{center}
}

% -------------------------------------------------
\frame
{
  \frametitle{S2 Experiment - snapshots}
\begin{center}
% \vspace{-0.3cm}
\includegraphics[width=5cm]{figures/phd18_vel_day10_100m.pdf}
\includegraphics[width=5cm]{figures/phd18_vel_day30_100m.pdf}
\end{center}
}


% -------------------------------------------------
\frame
{
  \frametitle{S2 Experiment - snapshots}
\begin{center}
% \vspace{-0.3cm}
\includegraphics[width=5cm]{figures/phd18_vel_day60_100m.pdf}
\includegraphics[width=5cm]{figures/phd18_vel_day90_100m.pdf}
\end{center}
}

% -------------------------------------------------
\frame
{
  \frametitle{Results Summary - model vs data comparison}
\begin{center}
% \vspace{-0.3cm}
\includegraphics[width=12cm]{figures/comparison.pdf}
\end{center}
}




\subsection{Dynamics of the Eddy Formation}

% -------------------------------------------------
\frame
{
  \frametitle{Run-averaged fields: the basic state}
\begin{center}
% \vspace{-0.3cm}
\includegraphics[width=13cm]{figures/average_fields.pdf}
\end{center}
}

% -------------------------------------------------
\frame
{
  \frametitle{QG Approximation}
\begin{small}
\begin{eqnarray*}
Ro = \frac{|\zeta|}{|f_0|} << 1
\label{eq:Ro}
\end{eqnarray*}
\vspace{-0.5cm}
\end{small}
\begin{center}
% \vspace{-0.3cm}
\includegraphics[width=9cm]{figures/rossby_map.pdf}
\end{center}
}


% -------------------------------------------------
\frame
{
  \frametitle{Conditions for leeward eddy formation}

\begin{footnotesize}
According to {\it Magaldi et al. (2008)}, when the Burger Number is larger than one ($B_u > 1$), 
leeward eddies emerge at the lee of topographic promontories. 

\begin{eqnarray*}
    Bu = \frac{R_d}{D}
    \label{eq:burger}
\end{eqnarray*}

\begin{eqnarray*}
    R_d = \frac{ \sqrt{ \frac{\Delta \rho}{\rho_0} g H_i} } {|f_0|}
    \label{eq:rd}
\end{eqnarray*}

$B_u$(BC) = 16.61\\
$B_u$(NBUC) = 5.27\\
Anticyclones are expected at the NBUC layer: \textcolor{blue}{We have them!}\\
Cyclones are expected at the BC layer: \textcolor{red}{We do not have them. Why?}

\end{footnotesize}
}


% -------------------------------------------------
\frame
{
  \frametitle{Baroclinic instability}
\begin{center}
% \vspace{-0.3cm}
\includegraphics[width=10cm]{figures/instability.pdf}
\end{center}
\begin{small}
\begin{eqnarray*}
EC = \frac{g\alpha}{\theta_z} \left( \overline{u'T'} \frac{\partial\bar{T}}{\partial x} + 
                                     \overline{v'T'} \frac{\partial\bar{T}}{\partial y}  \right )
\label{eq:bc_conversion}
\end{eqnarray*}

{\it Cronin \& Watts (1996), Mano et al (2009), Francisco et al (2011)}
\end{small}
}


% -------------------------------------------------
\frame
{
  \frametitle{Energy Conversion Analysis}
\begin{center}
% \vspace{-0.3cm}
\includegraphics[width=4cm]{figures/phd15_zoom_10_1m.pdf}
\includegraphics[width=4cm]{figures/phd15_zoom_10_400m.pdf}\\
\includegraphics[width=6cm]{figures/energy_conversion.pdf}
\end{center}
}

% -------------------------------------------------
\frame
{
  \frametitle{Energy Conversion Analysis}
\begin{center}
% \vspace{-0.3cm}
\includegraphics[width=4cm]{figures/phd15_zoom_20_1m.pdf}
\includegraphics[width=4cm]{figures/phd15_zoom_20_400m.pdf}\\
\includegraphics[width=6cm]{figures/energy_conversion.pdf}
\end{center}
}

% -------------------------------------------------
\frame
{
  \frametitle{Energy Conversion Analysis}
\begin{center}
% \vspace{-0.3cm}
\includegraphics[width=4cm]{figures/phd15_zoom_30_1m.pdf}
\includegraphics[width=4cm]{figures/phd15_zoom_30_400m.pdf}\\
\includegraphics[width=6cm]{figures/energy_conversion.pdf}
\end{center}
}

\section{Final Remarks}

% -------------------------------------------------
\frame
{
  \frametitle{Final Remarks}
\begin{footnotesize}
\begin{alertblock}{\textcolor{black}{\bf Conclusions}}
  \begin{enumerate}
    \item{The near-surface flow in the surroundings of the BC site of origin (10-20$^\circ$)S
          is recurrently dominated by anticyclonic mesoscale eddies.}
    \item{Topography and BC-NBUC interaction are both 
          essential to explain the observed mesoscale patterns. Geophysical baroclinic instability appears to be playing a key role in the eddy formation. }
  \end{enumerate}
\end{alertblock}

\begin{alertblock}{\textcolor{black}{\bf Future Work}}
  \begin{enumerate}
    \item{The obvious one: more {\it in situ} observations are needed!  }
    \item{Modeling initiatives in the region, for process studies or nowcast and forecast, should
          never underestimate the importance of the undercurrent.}
    \item{Include the DWBC in the system and evaluate its role. }
    \item{Study the response of different BC-NBUC transport ratios.}
  \end{enumerate}
\end{alertblock}
\end{footnotesize}

}


% -------------------------------------------------
\frame
{
  \frametitle{Acknowledgments}

\includegraphics[width=2cm]{USP.png}\hspace{1cm}
\includegraphics[width=2cm]{IO.jpg}\hspace{1cm}
\includegraphics[width=2cm]{lado.png}\hspace{1cm}
\includegraphics[width=2cm]{SMASTlogo_umass_circle.pdf}\\
\includegraphics[width=2cm]{logo_capes.jpg}\hspace{1cm}
\includegraphics[width=2cm]{logo_cnpq.png}\hspace{1cm}
\includegraphics[width=1cm]{brasao.jpg}\hspace{1cm}
\includegraphics[width=2cm]{latex_logo.png}\\
\includegraphics[width=1cm]{logo_linux.jpg}\hspace{1cm}
\includegraphics[width=3cm]{logo_python.png}\hspace{1cm}
\includegraphics[width=3cm]{logo_matplotlib.png}\hspace{1cm}
\includegraphics[width=1cm]{logo_inkscape.png}

}

% -------------------------------------------------
\frame
{
  \frametitle{}
\hspace{3.5cm} {\huge Thank you!}\\

\vspace{2cm}

\hspace{3.7cm} {\huge Obrigado!}

}

\end{document}
