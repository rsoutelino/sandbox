% $Header: /cvsroot/latex-beamer/latex-beamer/examples/beamerexample3.tex,v 1.8 2004/10/07 20:53:07 tantau Exp $

\documentclass[pdftex]{beamer}

% \documentclass[12pt,portuguese,a4paper,pdftex]{article}
%\usepackage{epstopdf}
% \usepackage[pdftex]{hyperref}
%\documentclass[12pt,a4paper]{article}
% \usepackage[brazil]{babel}
%\usepackage[latin1]{inputenc} %traduz acentos
% \usepackage{natbib}
%\usepackage{epsf,psfig}
% \usepackage[dvips]{graphicx}
%\usepackage[square]{natbib}
% \usepackage{subfigure}
% \usepackage{rotating}
% \usepackage{palatino}
\usepackage{amssymb}
\usepackage{ucs}
\usepackage[utf8x]{inputenc}



% Copyright 2003 by Till Tantau <tantau@cs.tu-berlin.de>.
%
% This program can be redistributed and/or modified under the terms
% of the LaTeX Project Public License Distributed from CTAN
% archives in directory macros/latex/base/lppl.txt.

%
% The purpose of this example is to show how \part can be used to
% organize a lecture.
%
% \usetheme{Warsaw}
\usetheme{Frankfurt}
\usepackage[english]{babel}
% \usepackage[latin1]{inputenc}

\setbeamercovered{transparent}

\setbeamertemplate{footline}{
  \begin{beamercolorbox}[wd=\paperwidth,ht=0.2cm,dp=0.1cm]{footcol}
    \Tiny\hspace*{4mm}\insertshortauthor\hfill\insertshorttitle\hfill\insertframenumber\hspace{4mm}
  \end{beamercolorbox}
}

\logo{\includegraphics[height=1cm]{lado.png}}
%
% The following info should normally be given in you main file:
%


\title{Descrição Sinótica da Origem da Corrente do Brasil}
\author{Rafael Soutelino}
\institute{Laboratório de Dinâmica Oceânica - IOUSP\\
Orientador: Ilson Carlos Almeida da Silveira}
\date{02 de julho de 2007}

\beamertemplatenavigationsymbolsempty

\begin{document}

\frame{\titlepage}


\section*{Descrição Sinótica da Origem da Corrente do Brasil}

\subsection{Sumário}

\frame{
  \nameslide{Sumário}
  \frametitle{Sumário da Apresentação}
  \tableofcontents[pausesections,part=1]
}






\part{Review of Previous Lecture}

\section[Introdução]{Introdução}
\frame<beamer>{\tableofcontents[current]}

\subsection[Larga Escala]{Padrões de Larga Escala}

% -------------------------------------------------
\frame
{
  \frametitle{A Circulação de Larga Escala no Atlântico Sul}
\includegraphics[width=10cm,keepaspectratio=true]{../figuras/brasil_3D_new.pdf}
}

% -------------------------------------------------
\frame
{
  \frametitle{A Circulação de Larga Escala no Atlântico Sul}
\includegraphics[width=10cm,keepaspectratio=true]{../figuras/CB_3D.pdf}
}

% -------------------------------------------------
\frame
{
  \frametitle{A Circulação de Larga Escala no Atlântico Sul}
\includegraphics[width=10cm,keepaspectratio=true]{../figuras/CCI_3D.pdf}
}

% -------------------------------------------------
\frame
{
  \frametitle{A Circulação de Larga Escala no Atlântico Sul}
\includegraphics[width=10cm,keepaspectratio=true]{../figuras/CCP_3D.pdf}
}


% -------------------------------------------------
\frame
{
  \frametitle{A Bifurcação da CSE}
\begin{block}{\textcolor{yellow}{Principais Considerações}}
\begin{itemize}
\vspace{0.3cm}
      \item A bifurcação ocorre de forma \textcolor{red}{estratificada};
\vspace{0.5cm}
      \item Pode ter implicações na \textcolor{red}{varibilidade climática global}, pois o sistema 
            CSE/SNB é o principal condutor para o retorno de águas da MOC;
\vspace{0.5cm}
      \item A localização da bifurcação pode prover informações a cerca da quantidade
            de  água que é exportada para o Atlântico Norte e recircula  no giro subtropical;
\vspace{0.5cm}
      \item A latitude precisa da bifurcação ainda é \textcolor{red}{obscura}.
    \end{itemize}
\vspace{0.1cm}
\end{block}
}


\subsection[Circulação Regional]{Circulação Regional - Costa Leste}

% -------------------------------------------------
\frame
{
  \frametitle{A Costa Leste Brasileira}

\includegraphics[width=10cm,keepaspectratio=true]{../figuras/mapa_leste.png}

}


% -------------------------------------------------
\frame
{
  \frametitle{Circulação Regional - Sistema SNB/CNB}
{\small {\it Silveira et al.} (1994) $\rightarrow$ Velocidades baroclínicas relativas a 1000 dbar.}
\begin{columns}
  \begin{column}{7cm}
\vspace{0.5cm} 
      \hspace{2.5cm} \textcolor{blue}{$\approx$ 10$^\circ$S}
\vspace{-0.5cm}
      \begin{flushleft}
        \includegraphics[width=7cm,keepaspectratio=true]{../figuras/secaoCNB_silveira94.pdf}
      \end{flushleft}    
  \end{column}
  \begin{column}{6cm}
     \begin{flushright}
       \hspace{-2cm}\includegraphics[width=7cm,keepaspectratio=true]{../figuras/secaoCNB_silveira94_2.pdf}
     \end{flushright}
      \hspace{2.1cm} \textcolor{blue}{$\approx$ 5$^\circ$S}    
\vspace{1.5cm}

  \end{column}
\end{columns}

}


% -------------------------------------------------
\frame
{
  \frametitle{Circulação Regional - Sistema SNB/CNB}
{\small {\it Stramma et al.} (1995) $\rightarrow$ Velocidades medidas diretamente - ADCP.}
\vspace{0.5cm}
\begin{columns}
  \begin{column}{6cm}
\vspace{1.5cm} 
      \textcolor{blue}{$\approx$ 10$^\circ$S}
\vspace{-2cm}
      \begin{flushleft}
        \includegraphics[width=7cm,keepaspectratio=true]{../figuras/secaoCNB_strammaetal95.pdf}
      \end{flushleft}    
\vspace{-3cm}
  \end{column}
  \begin{column}{5cm}
     \begin{flushright}
       \hspace{-2cm}\includegraphics[width=7cm,keepaspectratio=true]{../figuras/secaoCNB_strammaetal95_2.pdf}
     \end{flushright}
\vspace{-0.3cm}
      \hspace{3.5cm} \textcolor{blue}{$\approx$ 5$^\circ$S}    
\vspace{1.5cm}

  \end{column}
\end{columns}

}


% -------------------------------------------------
\frame
{
  \frametitle{Circulação Regional - Sistema CB/CCI}
\begin{center}
{\small {\it Soutelino} (2005) $\rightarrow$ Velocidades baroclínicas absolutas (POM) em torno de 
19$^\circ$S. {\bf \textcolor{red}{Setembro de 2001}}.}
\end{center}
\vspace{0.5cm}
\begin{center}
\includegraphics[width=9cm,keepaspectratio=true]{../figuras/vel_pom_rad13.pdf}  
\end{center}
}

% -------------------------------------------------
\frame
{
  \frametitle{Circulação Regional - Sistema CB/CCI}
\begin{center}
{\small {\it Silveira et al.} (2006) $\rightarrow$ Velocidades baroclínicas relativas a 1000 dbar.
 {\bf \textcolor{red}{Março de 2005}}.}
\end{center}
\vspace{0.5cm}
\begin{columns}
  \begin{column}{6cm}
    \includegraphics[width=6cm,keepaspectratio=true]{../figuras/hor_geopsi_AO_0100m_color.pdf}
  \end{column}
  \begin{column}{6cm}
    \includegraphics[width=6cm,keepaspectratio=true]{../figuras/hor_geopsi_AO_0600m_color.pdf}
  \end{column}
\end{columns} 
}

\section[Objetivos]{Objetivos}
\frame<beamer>{\tableofcontents[current]}

\frame
{
  \frametitle{Objetivos}
\begin{alertblock}{Objetivo Central}
Descrição sinótica do sítio de bifurcação CSE e por consequência, sítio de origem da CB.
\end{alertblock}

\begin{block}{Objetivos Específicos}
{\small
\vspace{0.2cm}
\begin{itemize}
\item Descrição sinótica da circulação geostrófica regional na área da bifurcação da
 CSE através da construção de campos de função de corrente em vários níveis;
\vspace{0.1cm}
\item Comparação da circulação geostrófica com aquela obtida por medidas diretas de
 velocidade via ADCP, através da construção dos campos de função de corrente observada em vários níveis.
\vspace{0.1cm}
\item Investigar a assinatura geostrófica climatológica da bifurcação da CSE em uma área maior de abrangência.
\end{itemize}
}
\end{block}
}

\section[Dados]{Conjunto de Dados}
\frame<beamer>{\tableofcontents[current]}

\subsection[Climatologia]{Hidrografia - Climatologia}

% -------------------------------------------------
\frame
{
  \frametitle{Hidrografia - Cenário Climatológico}
\begin{center}
{\bf World Ocean Atlas - resolução de 0,25$^\circ$}
\end{center}
\begin{center}
\includegraphics[width=10cm,keepaspectratio=true]{../figuras/woa_global_temp.png}
\end{center}
}

\subsection[Cenário Sinótico]{Hidrografia - Cenário Sinótico}

% -------------------------------------------------
\frame
{
  \frametitle{Hidrografia - Cenário Sinótico}
\vspace{-0.5cm}
\begin{columns}
  \begin{column}{6cm}
    \begin{center}
    {\bf Oceano Leste I}\\
    \textcolor{red}{Setembro de 2001}
    \includegraphics[width=5.5cm,keepaspectratio=true]{../proc/hidrografia/figuras/grade_leste1.pdf}
    \end{center}
  \end{column}
  \begin{column}{6cm}
    \begin{center}
    {\bf Oceano Leste II}\\
    \textcolor{red}{Março de 2005}
    \includegraphics[width=5.5cm,keepaspectratio=true]{../proc/hidrografia/figuras/grade_leste2.pdf}
    \end{center}
  \end{column}
\end{columns} 
}


\subsection[ADCP]{Velocidades Diretas - ADCP}

% -------------------------------------------------
\frame
{
  \frametitle{ADCP - Metodologia de processamento}
\vspace{0.5cm}
\begin{columns}
  \begin{column}{6cm}
    \begin{block}{Etapas}
    \begin{enumerate}
      \item Exportação via ``WinADCP'' - médias longas;
\vspace{0.5cm}
      \item Remoção dos dados obtidos com o navio ``parado'';
\vspace{0.5cm}
      \item Remoção de spikes;
    \end{enumerate}
    \end{block}
  \end{column}
  \begin{column}{5cm}
    \vspace{-1cm}
    \begin{center}
    \includegraphics[width=5cm,keepaspectratio=true]{../proc/adcp/figuras/adcp_bruto_1m.pdf}
    \end{center}
  \end{column}
\end{columns} 
}

\frame
{
  \frametitle{ADCP - Metodologia de processamento II}

\begin{block}{\textcolor{yellow}{Remoção do Transporte de Ekman}}
{\small
    \begin{itemize}
      \item Promediação em caixas;
\vspace{0.2cm}
      \item Dados de vento oriundos do ``quickscat'';
\vspace{0.2cm}
      \item Cálculo do campo médio de velocidades observadas no interior da camada de Ekman ($\vec{V}$). É estimado um valor médio de espessura para o cruzeiro: \textcolor{blue}{$\delta_E \approx$  20 m};
\vspace{0.2cm}
      \item Cálculo do Transporte de Ekman, individualmente, para cada dia do cruzeiro: $\vec{V_e} = \frac{\vec{\tau} x \vec{k}}{\rho f}$;
\vspace{0.2cm}
      \item Simples subtração vetorial: $\vec{V} - \vec{V_e}$.
    \end{itemize}
}
\end{block}
}

\frame
{
  \frametitle{ADCP - Metodologia de processamento II}
\begin{center}
{\bf Velocidades na Camada de Ekman}\\
\vspace{1cm}
\includegraphics[width=3.6cm,keepaspectratio=true]{../proc/adcp/figuras/adcp_trat_20m.pdf}
\includegraphics[width=3.2cm,keepaspectratio=true]{../proc/adcp/figuras/adcp_ek_20m.pdf}
\includegraphics[width=3.6cm,keepaspectratio=true]{../proc/adcp/figuras/adcp_20m.pdf}
\end{center}
}

\frame
{
  \frametitle{ADCP - Metodologia de processamento III}
\begin{center}
{\bf Função de Corrente Observada - }\textcolor{blue}{Análise Objetiva Vetorial}
\end{center}
}


\section[Resultados]{Resultados Preliminares}
\frame<beamer>{\tableofcontents[current]}



\frame
{
  \frametitle{$\Psi$ Observada x $\Psi$ Geostrófica - OEII - \textcolor{yellow}{1-20m}}

\begin{center}
\includegraphics[width=5.2cm,keepaspectratio=true]{../proc/adcp/figuras/adcp_AO_20m.pdf}
\includegraphics[width=5.2cm,keepaspectratio=true]{../proc/hidrografia/figuras/psi_OEII_med_20m.pdf}
\end{center}
}


\frame
{
  \frametitle{$\Psi$ Observada \textcolor{yellow}{1-20m} x $\Psi$ Geostrófica \textcolor{yellow}{200m} - OEII}

\begin{center}
\includegraphics[width=5.2cm,keepaspectratio=true]{../proc/adcp/figuras/adcp_AO_20m.pdf}
\includegraphics[width=5.2cm,keepaspectratio=true]{../proc/hidrografia/figuras/psi_OEII_200m.pdf}
\end{center}
}

\frame
{
  \frametitle{$\Psi$ Climatológica x $\Psi$ Geostrófica - OEII - \textcolor{yellow}{1-20m, lc = 3$^\circ$}}

\begin{center}
\includegraphics[width=5.2cm,keepaspectratio=true]{../proc/climatologia/figuras/psi_woa_fev,mar_20m.pdf}
\includegraphics[width=5.2cm,keepaspectratio=true]{../proc/hidrografia/figuras/psi_OEII_20m_3gr.pdf}
\end{center}
}


\frame
{
  \frametitle{$\Psi$ Rel. a 1000 dbar - \textcolor{yellow}{500 / 800 m}}

\begin{center}
\includegraphics[width=5.2cm,keepaspectratio=true]{../proc/hidrografia/figuras/psi_OEII_500m.pdf}
\includegraphics[width=5.2cm,keepaspectratio=true]{../proc/hidrografia/figuras/psi_OEII_800m.pdf}
\end{center}
}


\frame
{
  \frametitle{$\Psi$ Rel. a 1000 dbar x $\Psi$ Rel. a ADCP - \textcolor{yellow}{Superfície}}

\begin{center}
\includegraphics[width=5.2cm,keepaspectratio=true]{../proc/hidrografia/figuras/psi_OEII_1m.pdf}
\includegraphics[width=5.2cm,keepaspectratio=true]{../proc/ref_adcp/figuras/psi_ref_adcp_1m.pdf}
\end{center}
}

\frame
{
  \frametitle{$\Psi$ Rel. a 1000 dbar x $\Psi$ Rel. a ADCP - \textcolor{yellow}{100 m}}

\begin{center}
\includegraphics[width=5.2cm,keepaspectratio=true]{../proc/hidrografia/figuras/psi_OEII_100m.pdf}
\includegraphics[width=5.2cm,keepaspectratio=true]{../proc/ref_adcp/figuras/psi_ref_adcp_100m.pdf}
\end{center}
}

\frame
{
  \frametitle{$\Psi$ Rel. a 1000 dbar x $\Psi$ Rel. a ADCP - \textcolor{yellow}{200 m}}

\begin{center}
\includegraphics[width=5.2cm,keepaspectratio=true]{../proc/hidrografia/figuras/psi_OEII_200m.pdf}
\includegraphics[width=5.2cm,keepaspectratio=true]{../proc/ref_adcp/figuras/psi_ref_adcp_200m.pdf}
\end{center}
}

\frame
{
  \frametitle{$\Psi$ Rel. a 1000 dbar x $\Psi$ Rel. a ADCP - \textcolor{yellow}{500 m}}

\begin{center}
\includegraphics[width=5.2cm,keepaspectratio=true]{../proc/hidrografia/figuras/psi_OEII_500m.pdf}
\includegraphics[width=5.2cm,keepaspectratio=true]{../proc/ref_adcp/figuras/psi_ref_adcp_500m.pdf}
\end{center}
}

\frame
{
  \frametitle{$\Psi$ Rel. a 1000 dbar x $\Psi$ Rel. a ADCP - \textcolor{yellow}{800 m}}

\begin{center}
\includegraphics[width=5.2cm,keepaspectratio=true]{../proc/hidrografia/figuras/psi_OEII_800m.pdf}
\includegraphics[width=5.2cm,keepaspectratio=true]{../proc/ref_adcp/figuras/psi_ref_adcp_800m.pdf}
\end{center}
}

% \frame
% {
%   \frametitle{Seções Verticais - OEII}
% \begin{center}
% \includegraphics[width=5cm,keepaspectratio=true]{../proc/hidrografia/figuras/vgAO_leste2_sec1.pdf}
% \includegraphics[width=5cm,keepaspectratio=true]{../proc/hidrografia/figuras/vgAO_leste2_sec5.pdf}\\
% \includegraphics[width=5cm,keepaspectratio=true]{../proc/hidrografia/figuras/vgAO_leste2_sec9.pdf}
% \includegraphics[width=5cm,keepaspectratio=true]{../proc/hidrografia/figuras/vgAO_leste2_sec13.pdf}
% \end{center}
% }
% 
% \frame
% {
%   \frametitle{Seções Verticais - OEII}
% \begin{center}
% \includegraphics[width=5cm,keepaspectratio=true]{../proc/hidrografia/figuras/vgAO_leste2_sec17.pdf}
% \includegraphics[width=5cm,keepaspectratio=true]{../proc/hidrografia/figuras/vgAO_leste2_sec21.pdf}\\
% \includegraphics[width=5cm,keepaspectratio=true]{../proc/hidrografia/figuras/vgAO_leste2_sec25.pdf}
% \includegraphics[width=5cm,keepaspectratio=true]{../proc/hidrografia/figuras/vgAO_leste2_sec29.pdf}
% \end{center}
% }
% 
% \frame
% {
%   \frametitle{Seções Verticais - OEII}
% \begin{center}
% \includegraphics[width=5cm,keepaspectratio=true]{../proc/hidrografia/figuras/vgAO_leste2_sec33.pdf}
% \includegraphics[width=5cm,keepaspectratio=true]{../proc/hidrografia/figuras/vgAO_leste2_sec37.pdf}\\
% \includegraphics[width=5cm,keepaspectratio=true]{../proc/hidrografia/figuras/vgAO_leste2_sec41.pdf}
% \includegraphics[width=5cm,keepaspectratio=true]{../proc/hidrografia/figuras/vgAO_leste2_sec45.pdf}
% \end{center}
% }
% 


\end{document}


