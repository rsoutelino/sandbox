\hspace{6mm} The origin of Brazil Current (BC) is associated with the 
bifurcation of the southern branch of the South Equatorial Current (BiSEC). The BC then
flows south bordering the Brazilian continental margin. 
Inferences on the BC origin site as well as its organization as a
western boundary current are based in large scale patterns of the
BiSEC. There is a lack of details on the location of the BC and the possible
mesoscale activity related to its formation. While the BC meandering
off the Southeastern and Southern Brazilian continental margin has
been widely reported in the literature, no information is available
for the BC structure and its meanders off the Brazilian Eastern
margin. The motivation and main goal of this work is to describe the
mesoscale scenario associated with the BC formation and organization
as a boundary current. In order to reach this goal, we opted to first
build a climatological geostrophic picture of the BiSEC and BC origin
for the summer using solely termohaline data. Following such
investigation, we analyzed data from an oceanographic cruise conducted by
the Brazilian Navy that consisted of simultaneous CTD profiling and
vessel-mounted ADCP velocity measurements, also for the summer
period. The quasi-synoptic data analysis was conducted using a method
that has not been employed for flows along the Brazilian continental
margin: the ADCP-referenced dynamic method. Unlike the traditional
dynamic method, ADCP velocities are used to reference the CTD-derived
baroclinic velocities and to obtain a total geostrophic velocity
field. In the climatological data analysis, the relative geostrophic
current patterns exhibited a BiSEC signature at 9$^\circ$S for
surface, at 12$^\circ$S for 200 m, at 20$^\circ$S for 500 m and
23$^\circ$S for 800 m. We thus confirmed information of the literature
about the southward migration of the BiSEC structure with depth. We
also identified the summer mean scenario for the BC and North Brazil
Undercurrent (NBUC) formation. The quasi-synoptic total velocity fields
confirmed the relative climatological fields. At surface, we found no
signature of the BiSEC within the cruise sampled area. The BC was
originated north of 10$^\circ$S and seemed to start flowing, as a weak
current with vertical extension less than 100 m deep, distant from the
western boundary. At 150 m (500 m), the BiSEC signature was evident at
14.5$^\circ$S (20$^\circ$S). We thus suggest that the synoptic BC
origin as a western boundary current was located at around
14.5$^\circ$S. We can summarize our findings describing that the BC
organized itself as a weak and shallow current at around 10$^\circ$S.
At 14.5$^\circ$S, the BC attached itself to the continental margin and
flew following the bathymetric contours. As it flows south, it
meandered vigorously and frontal anticyclones were present off the main
topographic features of the Eastern coast: the Royal Charlotte and the
Abrolhos Banks.  In its path southward, the BC was intensified and
extended vertically. It seemed that NBUC was originated at around
20$^\circ$S with a velocity core centered at 700 m. As this
undercurrent flows northward, it increased transport and its velocity
core became shallower. At the northern portion of the study area, the
NBUC core reached the depth of 250 m.
