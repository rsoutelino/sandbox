\section{Preâmbulo}\label{sec:pre_dados}

\hspace{6mm} Uma vez traçados os objetivos, a busca por atingi-los passa por explorarmos 
um robusto conjunto de dados que nos permita resolver, adequadamente a nossos propósitos, 
as estruturas de interesse. Lembramos aqui que temos em mãos um problema de escalas. 
A investigação sinótica do sítio de origem da CB é diretamente dependente
da localização de uma estrutura de maior escala, ou seja, a BiCSE.

Portanto, para o caso da investigação da BiCSE, buscamos o conjunto de dados climatológicos 
{\it World Ocean Atlas 2001} (WOA2001), que foram apresentados recentemente à comunidade científica através
dos esforços de \cite{boyer_etal2005}. Sempre que nos referirmos aos dados climatológicos, estamos 
considerando-os como um cenário médio de circulação, ideal para representar a estrutura de larga escala
da BiCSE. 

Para investigar os padrões de meso-escala associados a origem da CB, é necessário um conjunto de mais alta 
resolução que abranja uma área mais limitada. Este é o caso da operação Oceano Leste II (OEII),
realizada pela Marinha do Brasil entre fevereiro e março de 2005.

Adiantamos que exploraremos ambos os conjuntos de dados de modo a construir campos horizontais de velocidade
horizontalmente não-divergente em vários níveis, 
baseando-se no conceito e mapeamento de {\bf função de corrente}, que será descrito apropriadamente no Capítulo 
\ref{cap:funccorr}. Entendemos que a obtenção destes campos consistem na melhor estratégia para conduzir as análises
pretendidas e nos subsidiarão na busca pelos objetivos traçados. 

\section{World Ocean Atlas 2001}\label{sec:woa2001}

\hspace{6mm} A origem da base de dados WOA2001 é a mesma de climatologias 
de temperatura e salinidade para os oceanos mundiais já consolidadas, tais como aquelas 
apresentadas como parte do \textsl{Climatological Atlas of the World Ocean} \citep{levitus1982} 
e suas atualizações em 1994 e 1998.

Os campos médios climatológicos de temperatura e salinidade da WOA2001 para os períodos anual, 
sazonal e mensal (com resolução horizontal de 0,25$^\circ$) são calculados com base nos dados do 
\textsl{World Ocean Database 2001} utilizando técnicas de análise objetiva. WOA2001 \textit{anual} 
e \textit{sazonal} apresentam campos em profundidades padrões desde a superfície até 5500 m, 
ao passo que WOA2001 \textit{mensal} atinge somente os primeiros 1500 m de profundidade. 

A resolução de 0,25$^\circ$ é uma melhoria diante daquela de 1$^\circ$ das demais 
climatologias de 1994 e 1998 citadas. O refinamento da resolução espacial permite que feições termohalinas de menor escala 
sejam resolvidas e que gradientes horizontais importantes em determinadas áreas sejam melhor 
preservados, resultando assim, em uma representação mais realista das características médias 
oceanográficas. A título de exemplo, a Figura \ref{fig:ex_woa} retrata 
os campos climatológicos anuais de temperatura e salinidade em superfície para o oceano mundial.

\begin{figure}%[hb]
 \begin{center}
  \includegraphics[width=14cm,keepaspectratio=true]{../figuras/t_global.png}
  \includegraphics[width=14cm,keepaspectratio=true]{../figuras/s_global.png}
 \end{center}
 \vspace{-.25cm}
 \renewcommand{\baselinestretch}{1}
 \caption{\label{fig:ex_woa} \small Campos anuais em superfície da base climatológica WOA2001 
 \citep{boyer_etal2005}: [superior] temperatura; [inferior] salinidade.}
\end{figure}

\newpage

\section{Operação Oceano Leste II}\label{sec:oe2}

\hspace{6mm} No escopo do {\it Plano de Coleta de Dados da Marinha do Brasil}, programa 
este voltado à obtenção de informações ambientais em áreas importantes ao comércio marítimo, 
à economia e à segurança nacional, a OEII foi conduzida a bordo do N.Oc. Antares no período 
de 01 de fevereiro de 2005 a 12 de março de 2005. Os dados amostrados de interesse deste
trabalho, consistem em perfis de temperatura e salinidade desde a superfície até 2500 m,
obtidos via CTD {\it SeaBird SBE 9Plus}
e perfilagem contínua de velocidade alcançada com o auxílio de um 
{\it ADCP RD Instruments Broadband} de 75 kHz. Os perfis termohalinos
foram distribuídos em 12 radiais normais à costa leste brasileira, entre 10$^\circ$S e 20$^\circ$S,
totalizando 112 estações oceanográficas (Figura \ref{fig:grade_leste2}). As perfilagens
de velocidade foram obtidas ao longo de todo o trajeto do navio, enquanto este 
atendia ao planejamento das estações oceanográficas (Figura \ref{fig:grade_leste2}).
Tais perfilagens, de acordo com a capacidade de penetração do equipamento, foram realizadas em média
desde a superfície até 400 m em média. Vale lembrar que a exploração deste
conjunto de dados é de caráter inédito, e sua grade amostral está localizada em um trecho
da costa brasileira que permite atender aos objetivos traçados no Capítulo \ref{cap:intro}.

Para proceder ao mapeamento de função de corrente, urge que consideremos dois procedimentos:
{\bf controle de qualidade e pré-processamento}.
Levando-se em con\-si\-de\-ra\-ção que os dados termohalinos (CTD) e de velocidade (ADCP) são de 
natureza diferente, optamos por descrever tais procedimentos em diferentes seções.

\begin{figure}
 \begin{center}
  \includegraphics[width=8cm,keepaspectratio=true]{../proc/hidrografia/figuras/grade_leste2.pdf}
  \includegraphics[width=7.7cm,keepaspectratio=true]{../proc/adcp/figuras/adcp_trechos.pdf}
 \end{center}
 \vspace{-.25cm}
 \renewcommand{\baselinestretch}{1}
 \caption{\label{fig:grade_leste2} \small Painel esquerdo: Distribuição das estações hidrográficas realizadas 
durante a OEII pela Marinha do Brasil, entre fevereiro e março de 2005. Painel direito: Representação geográfica
dos trechos onde o ADCP obteve perfis de velocidade.}
\end{figure}


\subsection{Dados Termohalinos - Pré-processamento}\label{sec:CTD}

\hspace{6mm} Em operações de coleta de dados termohalinos extensas como a OEII
é comum que os sensores de condutividade do CTD apresentem desvios. 
Estes desvios podem ser sistemáticos
ou aleatórios. No caso de serem sistemáticos, são passíveis de correção posteriormente. 
É de praxe, nas operações realizadas
pela Marinha do Brasil, a coleta de água em garrafas Ninskin, realizada de forma 
concomitante às perfilagens do CTD. Este procedimento permite medir a salinidade
posteriormente, pelo método indutivo. Consta no relatório de bordo fornecido gentilmente 
pela Marinha do Brasil que nenhum desvio do sensor de condutividade foi observado
ao longo dos dias em que a operação foi conduzida.

Tendo então avaliação da qualidade dos dados pela Marinha do Brasil, prosseguimos com a des\-cri\-ção de três 
procedimentos básicos necessários para que estes perfis termohalinos possam ser finalmente
utilizados em nossas análises. 

\subsubsection{Remoção de dados espúrios}\label{sec:spikes}

\hspace{6mm} É comum ocorrerem erros de comunicação entre o CTD e a unidade
de bordo, que armazena os dados em um arquivo. Estes erros são identificados nos arquivos
posteriormente através de códigos, que consistem em números notadamente diferentes dos 
valores de temperatura, condutividade ou pressão, encontrados em qualquer oceano do globo.
Tais erros de comunicação são conhecidos como {\it spikes}. É então necessária uma etapa
na qual estes {\it spikes} são reconhecidos e removidos dos arquivos.  

Pode ainda existir outro tipo de {\it spike}, que não está relacionado com falhas
e\-le\-trô\-ni\-cas e sim com o simples mal funcionamento do sensor em uma amostra isolada. Nesse caso
o valor de temperatura, condutividade ou pressão apresentam valores espúrios, mas dentro da faixa
típica para o oceano. Para detectar e eliminar estes {\it spikes}, adotamos um filtro baseado no
gradiente das propriedades. Como os perfis verticais dessas propriedades já são razoavelmente conhecidos, 
podemos estabelecer uma taxa máxima de variação das propriedades e eliminar qualquer valor que exceda
a essa taxa. Para tal, investigamos os perfis em blocos de 10 m desde a superfície até o fundo.
Em cada bloco referido, foram eliminados os valores que fossem superiores (inferiores) a ele mesmo 
somado (subtraído) de três vezes o desvio padrão do bloco. 

\subsubsection{Equi-espaçamento vertical}\label{sec:binagem}

\hspace{6mm} O perfilador CTD utilizado na OEII trabalhou com uma 
freq\"uência amostral de 15 Hz. Considerando que a velocidade de perfilagem do CTD
foi mantida nos entornos de 1 m s$^{-1}$, entre cada intervalo de 1 m de coluna de água
foram obtidas em torno de 15 amostras de temperatura e condutividade. Porém, nada nos garante
que este número é o mesmo em cada camada de 1 m. Em outras palavras, não há como garantir
um equi-espaçamento vertical entre as amostras no momento da coleta. Com isso, se faz
necessária a etapa de promediação vertical, conhecida como {\it binagem}. Esta etapa 
consiste em calcular médias verticais para cada camada de 1 m de coluna de água, 
resultando em perfis termohalinos equi-espaçados verticalmente. 

\subsubsection{Filtragem por Janela Móvel}\label{sec:janela}

\hspace{6mm} Para finalizar o tratamento básico dos dados de CTD ainda é necessária uma terceira
etapa, a {\it filtragem por janela móvel}. Este procedimento consiste em recalcular
os valores de temperatura e condutividade em cada metro de coluna de água. Os valores
são substituídos por uma média ponderada entre eles mesmos e valores adjacentes. 
A quantidade de valores adjacentes incluídos nesta média corresponde ao {\it tamanho}
da janela utilizada, e a distribuição dos pesos em relação ao valor central é determinada
pelo tipo da janela. Este procedimento é necessário pois os dados de CTD são extremamente ruidosos
para quem pretende realizar cálculos que envolvam derivadas ou derivadas segundas, 
que é o caso do presente trabalho, que pretende inferir velocidades geostróficas
a partir deste conjunto de perfis termohalinos. A eliminação desses ruídos é imprescindível, 
mas deve ser cautelosa, pois não podemos atenuar os importantes gradientes 
destas propriedades. Com isso, optamos por filtrar os diferentes perfis com janelas de tamanhos 
diferentes, dependendo da profundidade local. Para estações mais rasas do que 100 m,
utilizamos uma janela de 5 m. 
Para estações de profundidade entre 100 e 500 m, utilizamos
uma janela de 21 m. No restante das estações aplicamos uma janela de 31 m. Nas três camadas em questão, 
o tipo de janela utilizada foi {\it hanning}, que promove uma distribuição de 
pesos de caráter gaussiano, privilegiando a medida central. A Figura \ref{fig:trat_CTD}
ilustra cada etapa deste pré-processamento, mostrando o exemplo
de um tratamento efetuado em uma estação hidrográfica em particular.

\begin{figure}
 \begin{center}
  \includegraphics[width=15cm,keepaspectratio=true]{../proc/hidrografia/figuras/fig_trat_basico.pdf}
 \end{center}
 \vspace{-.25cm}
 \renewcommand{\baselinestretch}{1}
 \caption{\label{fig:trat_CTD} \small Perfil de temperatura referente aos primeiros 120 m de uma
estação oceânica da OEII, exemplificando duas etapas do tratamento básico dos dados termohalinos.}
\end{figure}

Feito este tratamento básico, os dados estão aptos aos cálculos subseq\"uentes, que serão
devidamente descritos nas próximas seções.   

\subsection{Dados de Velocidade - Pré-processamento}\label{sec:ADCP}

\hspace{6mm} O ADCP é um medidor de correntes que se utiliza de uma peculiaridade da reflexão
de ondas sonoras contra	 um refletor que detém um movimento relativo ao emissor/receptor,
chamada de {\it Efeito Doppler}. Um feixe sonoro, que é emitido com uma freq\"uência 
$\mathcal{F}_1$, ao ser refletido por uma partícula em movimento, retorna com uma
freq\"uência $\mathcal{F}_2$. O equipamento, então, essencialmente emite um feixe
sonoro, que é refletido pelas partículas em suspensão na água. Ele é capaz de quantificar
essa modificação na freq\"uência e associá-la à velocidade e direção de deslocamento
das partículas e, por conseq\"uência, do fluido em que elas estão imersas passivamente.

O ADCP usado no levantamento disponível para este trabalho, de 75 kHz, é capaz de medir velocidades 
na coluna de água desde a superfície até aproximadamente 400 m de profundidade. A perfilagem
vertical da velocidade é dividida em camadas equi-espaçadas pré-configuradas de 8 m. 

Apesar dos perfiladores acústicos de velocidade terem surgido na Oceanografia na década de 80, 
o processamento de seus dados ainda é motivo de discussão na literatura e no meio científico em geral.
Com isso, dedicamos esta seção ao detalhamento e justificativa do uso dos procedimentos implementados ao conjunto de dados de ADCP utilizado neste trabalho. Para montar o protocolo de pré-pro\-ces\-sa\-men\-to dos dados de velocidade aqui utilizados, orientamo-nos
pelos trabalhos do ``{\it Currents Group}'', da Universidade do Havaí, liderado pelo Dr. Eric Firing. Dividiremos o detalhamento dos 
procedimentos necessários em subitens para favorecer o entendimento e a clareza. 

\subsubsection{Promediação Temporal}\label{sec:lta}

\hspace{6mm} Conforme mencionamos anteriormente, o ADCP mede a velocidade do fluido em função 
da profundidade na forma de perfis através do {\it Efeito Doppler}. Estes perfis são obtidos através
de, no mínimo, três emissores e receptores de ondas sonoras denominados transdutores. O conjunto de 
feixes sonoros emitidos pelos transdutores é denominado {\it ping}, que consiste na forma mais 
bruta dos dados de ADCP. Cada {\it ping} resulta em um perfil de velocidade desde a superfície
até a profundidade que os feixes sonoros são capazes de penetrar. O próprio princípio físico 
envolvido por trás da obtenção das velocidades torna o método bastante ruidoso, fazendo com que
sejam necessárias etapas de remoção desses ruídos. 

Para o caso da OEII, o intervalo amostral para a emissão dos
{\it pings} foi de 3 segundos. O software do fabricante, que recebe os {\it pings} diretamente do ADCP ao longo da coleta, dispõe ainda
de opções de promediação temporal. Em suas con\-fi\-gu\-ra\-ções de geração dos arquivos de saída (que contêm os dados em si), 
existem diversas opções de promediação. Em síntese, são três os tipos de arquivos gerados: os brutos, contendo
os {\it pings} inalterados, os de médias curtas (``Short Time Average - STA'') e os de médias
longas (``Long Time Average - LTA''). Conforme consta nas informações de configuração do ADCP na OEII, os
arquivos de médias longas estão separados em intervalos de 10 minutos, e optamos então por utilizá-los apenas.
Livre dos ruídos mais grosseiros, passamos à próxima etapa, que consiste na obtenção dos valores absolutos das correntes, 
oriundos da remoção da velocidade do navio. 

\subsubsection{Cálculo da Velocidade Absoluta}\label{sec:wtrtrack}

\hspace{6mm} Os dados de velocidade são de natureza consideravelmente diferente dos dados obtidos pelo CTD.
Em primeiro lugar, consistem em uma grandeza vetorial, enquanto as propriedades medidas
pelo CTD são grandezas escalares, naturalmente mais fáceis de serem amostradas. 
A amostragem da velocidade se torna particularmente mais complexa para o caso do ADCP de casco, pois
o equipamento não se encontra em um referencial
fixo. Para que seja possível a obtenção da velocidade absoluta do fluido, ou seja, das correntes, é 
necessário o acoplamento de outros equipamentos ao sistema de aquisição via ADCP.  Para que seja feita uma 
estimativa razoável do vetor velocidade do navio, é necessária a presença de um sistema de posicionamento 
por satélite (GPS) e uma agulha giroscópica (GIRO). O GPS fornece as coordenadas geográficas do navio a cada tomada
de perfil que o ADCP executa. Tendo-se o tempo entre cada medida, é possível calcularmos o vetor 
velocidade do navio. A GIRO indica a direção em que a proa do navio está apontada em relação ao norte verdadeiro, indicando, 
se houver, possíveis desvios da mesma em relação a trajetória desenhada pelos pontos obtidos via GPS. 
Tendo-se uma boa estimativa da velocidade do navio, basta uma simples soma vetorial para que obtenhamos
a velocidade absoluta das correntes. A Figura \ref{fig:esqvet} ilustra tal soma. Na referida figura,
$\vec{v_n}$ representa a velocidade do navio, $\vec{v_d}$ representa a velocidade medida pelo ADCP e $\vec{v_f}$
representa a resultante, ou seja, a velocidade absoluta do fluido. A figura também ilustra esquematicamente 
o desalinhamento natural que pode ocorrer entre os eixos de orientação da trajetória do navio e sua proa (ao longo 
do cruzeiro, conforme as condições ambientais) e entre os eixos da proa do navio e dos transdutores do ADCP (dependente da
instalação do equipamento no casco). 

\begin{figure}
 \begin{center}
  \includegraphics[width=10cm,keepaspectratio=true]{../figuras/esqvet.pdf}
  \includegraphics[width=10cm,keepaspectratio=true]{../figuras/trim.pdf}
 \end{center}
 \vspace{-.25cm}
 \renewcommand{\baselinestretch}{1}
 \caption{\label{fig:esqvet} \small Aspectos geométricos e trigonométricos do sistema de aquisição de 
dados via ADCP de casco. Painel superior: plano horizontal. Painel inferior: plano vertical.}
\end{figure}

O método descrito acima seria simples, se não houvesse certas limitações in\-trín\-se\-cas ao sistema ADCP-GPS-GIRO. 
A GIRO usualmente apresenta uma resposta mais lenta do que a necessária para a freq\"uência em que se tomam os 
perfis de velocidade via ADCP. Esta resposta lenta se torna um fator importante em trechos onde o navio 
executa manobras bruscas, ou seja, onde o navio sofre qualquer tipo de aceleração, seja ela linear ou angular.
 Em levantamentos hidrográficos, como a OEII, estes trechos são tipicamente as paradas
para execução de estações hidrográficas e mudanças bruscas de trajetórias associadas aos trechos de navegação 
entre as radiais. Nestes trechos, a GIRO não se comporta de forma adequada, o que pode acarretar em erros na medição da 
magnitude e direção da corrente. A Figura \ref{fig:esq_erros} ilustra os tipos de erros e como eles podem ocorrer.

 \begin{figure}
 \begin{center}
  \includegraphics[width=12cm,keepaspectratio=true]{../figuras/esq_erro.pdf}
 \end{center}
 \vspace{-.25cm}
 \renewcommand{\baselinestretch}{1}
 \caption{\label{fig:esq_erros} \small Representação esquemática da ocorrência de erros durante a remoção do vetor
velocidade do navio, provocados por limitações associadas ao funcionamento da GIRO.}
\end{figure}

Adotamos aqui um procedimento de calibração que visa a eliminação destes desvios associados à GIRO. Este método está 
descrito em detalhes em \cite{joyce1989} e \cite{pollard_read1989} e exibiremos a seguir seus principais aspectos. Em primeiro lugar, enfatizamos
que a calibração é feita apenas nos trechos onde seja suspeito o mal funcionamento da GIRO, que são tipicamente os trechos
citados acima para o caso da OEII. \cite{pollard_read1989} consideram que, no caso de funcionamento inadequado do sistema da GIRO, 
é possível diagnosticar os desvios resultantes, e que os mesmos têm a tendência de serem constantes para cada sistema
ADCP-GPS-GIRO em particular. Antes de mais nada, vamos definir os eixos de coordenadas e as variáveis envolvidas
 no sistema de aquisição de dados via ADCP de casco. 

Como mostramos na Figura \ref{fig:esqvet}, que representa esquematicamente o cenário de levantamento de dados por ADCP de casco,
 o eixo $(x,y)$ representa o eixo cartesiano de coordenadas locais, sob
o qual estão orientados os dados referentes ao GPS e à GIRO. O eixo $(x',y')$ refere-se a orientação dos transdutores 
do ADCP. O ângulo que mede o desvio entre a orientação do ADCP e a proa do navio é denominado $\varphi$, e o ângulo que 
mede o desvio entre o ADCP e o norte verdadeiro é definido como $\alpha$. O ângulo $\gamma$, comumente referido 
como ``heading'', determina o desvio entre a proa do navio e sua trajetória, e pode ser ou não diferente de zero.
Chamemos de $\lambda = \alpha + \varphi$ o ângulo entre a proa do navio e o norte verdadeiro e finalmente de
$\beta$ o desvio entre a orientação dos transdutores do ADCP e a vertical local (Figura \ref{fig:esqvet}, 
painel inferior). O desvio $\beta$ sofre variações caso haja aceleração linear do navio e $\lambda$ sofre variações
 caso haja aceleração angular do mesmo. Os dados de ADCP, segundo \cite{joyce1989}, 
devem ser multiplicados por um fator $(1+\beta)$, onde $\beta$ geralmente é bem pequeno.

O conhecimento dos ângulos $\lambda$ e $\beta$ é então fundamental para conduzir a soma vetorial que resultará
na velocidade absoluta do fluido e os mesmos variam ao longo do cruzeiro conforme o navio muda sua trajetória e 
se submete ao balanço de sentido longitudinal provocado pelas ondas ({\it caturro}).
Os valores desses ângulos devem ser obtidos experimentalmente, utilizando o próprio conjunto de dados, nos trechos
mencionados anteriormente. Antes de prosseguirmos com o desenvolvimento matemático que acarretará em expressões para 
 $\lambda$ e $\beta$, assumimos a seguinte condição: ``{\it o vetor velocidade do fluido ($\vec{V_f}$) não deve variar
diante de qualquer tipo de manobra do navio''}. Essa é a única limitação deste método de calibração, mas consideramos
razoável assumir esta condição, levando-se em consideração a natureza do escoamento que pretendemos investigar. Escolhemos
uma camada específica para que sejam avaliadas as velocidades antes e depois de um evento de manobra brusca, que é denominada
``Camada de Referência''. Como é selecionada uma camada específica, a referida calibração recebe o nome de 
``Watertracking'', em analogia à ``Bottomtracking'', comumente utilizada em águas rasas, que usa como referencial fixo 
o assoalho oceânico. No presente caso, como nos concentramos em correntes oceânicas, que tipicamente fluem sobre
grandes profundidades, praticamente não existem trechos rasos onde o ADCP consegue trabalhar no modo ``Bottomtracking'', que é 
naturalmente mais eficiente. A camada de referência escolhida para o caso deste trabalho foi a décima oitava, que
corresponde a aproximadamente 150 m de profundidade. Tal escolha visa evitar a maior variabilidade espacial das correntes
em níveis mais rasos e ao mesmo tempo evitar regiões muito profundas, onde o sinal do ADCP tem menor capacidade de penetração. 
  
 Passando ao equacionamento do problema, definamos então $(u',v')$ como
as componentes da velocidade quando orientada no eixo de orientação do ADCP $(x',y')$ e $(u,v)$ as componentes de velocidade
orientadas no eixo cartesiano local $(x,y)$. Escrevendo na forma de equação a soma vetorial retratada no painel superior da 
Figura \ref{fig:esqvet}, obtemos: 

\begin{equation}
 u_f = u_n + u_d 
 \label{eq:soma}
\end{equation}
\begin{equation}
 v_f = v_n + v_d
 \label{eq:soma2}
\end{equation}

Se incluírmos nas Eqs. \ref{eq:soma} e \ref{eq:soma2} as definições de  $\lambda$ e $\beta$, chegamos em 

\begin{eqnarray}
 u_f = u_n + (1+\beta)[u'_d cos \lambda - v'_d sen \lambda] 
 \label{eq:soma3}
\end{eqnarray}
\begin{eqnarray}
 v_f = v_n + (1+\beta)[u'_d sen \lambda - v'_d cos \lambda]
 \label{eq:soma4}
\end{eqnarray}

Então, se $u_f$ e $v_f$ não sofrem modificações, conforme assumimos anteriormente, definimos como $d\vec{v}$ a diferença entre
a velocidade após o trecho e a velocidade antes do trecho de calibração e podemos escrever

\begin{eqnarray}
 du_n = u_n^{(2)} - u_n^{(1)} \ \ \ \ \ ; \ \ \ \ \ dv_n = v_n^{(2)} - v_n^{(1)}
 \label{eq:dVn}
\end{eqnarray}

\begin{eqnarray}
 du_d = u_d^{(2)} - u_d^{(1)} \ \ \ \ \ ; \ \ \ \ \ dv_d = v_d^{(2)} - v_d^{(1)}
 \label{eq:dVd}
\end{eqnarray}

\begin{eqnarray}
 du_f = 0 \ \ \ \ \ ; \ \ \ \ \ dv_f = 0
 \label{eq:dVf}
\end{eqnarray}

e usando as definições explicitadas em \ref{eq:dVn}-\ref{eq:dVf} nas Equações \ref{eq:soma3} e \ref{eq:soma4}, temos que 

\begin{eqnarray}
 du_n + (1+\beta)[du_d cos\lambda - dv_d sen \lambda] = 0
 \label{eq:du}
\end{eqnarray}

\begin{eqnarray}
 dv_n + (1+\beta)[du_d sen\lambda + dv_d cos \lambda] = 0.
 \label{eq:dv}
\end{eqnarray}

Finalmente, combinando as Eqs. \ref{eq:du} e \ref{eq:dv}, obtemos as seguintes expressões para $\lambda$ e $\beta$:

\begin{eqnarray}
 tg \lambda = \frac{dv_d.du_n - du_d.dv_n}{dv_d.dv_n + du_d.du_n};
 \label{eq:lambda}
\end{eqnarray}

\begin{eqnarray}
 1+\beta = \left[ \frac{du_n^2 + dv_n^2}{du_d^2 + dv_d^2} \right]^\frac{1}{2}
 \label{eq:beta}
\end{eqnarray}


Conforme dito anteriormente, os parâmetros $\lambda$ e $\beta$ são estimados a partir dos dados em trechos particulares
da trajetória do navio durante o levantamento hidrográfico/cor\-ren\-to\-grá\-fi\-co. Definamos então, mais precisamente,
que trechos são esses:

\begin{enumerate}

 \item {\bf Estações Oceanográficas:} é o momento mais interessante de executar os cál\-cu\-los, pois o navio 
permanece parado durante períodos que variam de 20 minutos até 3 horas. Nesses períodos, assumimos que a 
velocidade das correntes não pode variar e, adicionalmente, que a velocidade depois da estação
oceanográfica deve ser equivalente àquela medida antes da parada. Vale lembrar que a manobra de parada na estação
e retomada da trajetória são momentos onde a GIRO sofre com o problema de resposta lenta.

\item {\bf Guinadas bruscas:} nesse caso, assumimos que antes de uma guinada brusca, a velocidade da corrente deve
ser a mesma que aquela medida após a guinada. 

\item {\bf Trechos repetidos:} caso o navio coincida de navegar o mesmo trecho por duas ou mais vezes em um curto 
espaço de tempo, existe também uma boa oportunidade de fazer a calibração.

\end{enumerate}

Para obtermos valores  $\lambda$ e $\beta$ estatisticamente 
robustos é desejável o maior número pos\-sí\-vel de trechos de calibração.
Considerando essas três oportunidades de estimar as medidas de $\lambda$, imaginamos que um cruzeiro oceanográfico 
como o OEII forneça-nos um número razoável de trechos de calibração. A população de potenciais pontos de calibração é apresentada
na Figura \ref{fig:calib} em termos de sua localização geográfica.
 Para garantir esta robustez, efetuamos a calibração em todos
esses pontos possíveis e conduzimos uma análise estatística criteriosa para avaliar a função densidade de probabilidade dos 
parâmetros estimados. Montamos, então, um histograma usando todos os valores de $\lambda$ estimados e calculamos a média e 
o desvio padrão do mesmo. A Figura \ref{fig:hist_calib} ilustra tal análise e mostra uma boa qualidade estatística para 
os cálculos. Com isso, consideramo-nos aptos a aplicar o valor médio dos parâmetros estimados para obter finalmente os 
vetores corrigidos de velocidade absoluta da corrente. Os valores aplicados aos dados são de
$\lambda = $ 0,94$^\circ$ e $\beta = $ 0,07$^\circ$.

\begin{figure}
 \begin{center}
  \includegraphics[width=10cm,keepaspectratio=true]{../proc/adcp/figuras/calib_codas_OEII.pdf}
 \end{center}
 \vspace{-1cm}
 \renewcommand{\baselinestretch}{1}
 \caption{\label{fig:calib} \small Localização geográfica dos potenciais pontos de calibração ``Watertracking'' encontrados
ao longo da derrota do navio durante a OEII.}
\end{figure}

\begin{figure}
 \begin{center}
  \includegraphics[width=15cm,keepaspectratio=true]{../proc/adcp/figuras/cal2.pdf}
 \end{center}
 \vspace{-1cm}
 \renewcommand{\baselinestretch}{1}
 \caption{\label{fig:hist_calib} \small Representação gráfica da função densidade de probabilidade associada 
a po\-pu\-la\-ção de parâmetros de calibração $\lambda$ estimados ao longo da OEII.}
\end{figure}


\subsubsection{Remoção de perfis espúrios}\label{sec:edit}

\hspace{6mm} Vários são os fatores que podem influenciar a qualidade das medidas efetuadas pelo ADCP:
estado do mar, transparência da água, formação de bolhas nas proximidades dos transdutores, presença de
estruturas ou organismos vivos na coluna de água, falhas mecânicas ou eletrônicas, etc. Existem 
formas de diagnosticar a má qualidade dos dados para a maioria dos problemas exemplificados. Esta etapa
consistirá na eliminação de perfis espúrios, seja automaticamente, através de parâmetros de diagnóstico, seja
manualmente, através da inspeção minuciosa dos perfis. 

Dentre os parâmetros disponíveis para o diagnóstico de deficiências na qualidade dos dados, a velocidade vertical
é o mais utilizado. Assume-se que nos oceanos como um todo a velocidade vertical é muito pequena quando comparada
com a horizontal, diferindo minimamente em uma ordem de grandeza. Adicionalmente, a velocidade vertical
também tem valores absolutos típicos que não costumam ultrapassar uma determinada magnitude. Perfis que tiveram valores 
de velocidade vertical maiores que a horizontal ou apresentaram magnitudes de velocidade vertical maiores que um valor 
pré-determinado, considerado razoável para os escoamentos naturais, foram eliminados.

As velocidades horizontais são também usadas para diagnosticar problemas na qualidade dos dados. 
Para o caso do presente trabalho, onde pretendemos estudar escoamentos oceânicos, é razoável assumir que qualquer vetor de 
velocidade que tenha magnitude maior que 2 m s$^{-1}$ seja espúrio, fazendo com que o perfil seja descartado. 
A intensidade do sinal sonoro, ou intensidade do eco, é também utilizada como parâmetro para eliminar perfis.
Caso a intensidade do eco não atinja valores mínimos satisfatórios, o perfil é eliminado do conjunto. 

Por fim, utilizamos o {\it percent good} para eliminar dados espúrios. Conforme dito anteriormente, para que o 
sistema do ADCP funcione adequadamente, são necessários no mínimo três transdutores. É usual, para os fabricantes,
incluir um quarto transdutor, que pode ser redundante, mas aumenta a robustez estatística das medidas. O {\it 
percent good} é um parâmetro de qualidade que consiste no coeficiente de correlação entre as informações
recolhidas pelos 4 transdutores existentes. Este parâmetro é particularmente importante para detectar
problemas de qualidade associados ao estado do mar. O critério escolhido aqui é o de descartar os perfis que 
não atinjam pelo menos 50 \% de {\it percent good}. 

Passando à remoção manual de dados espúrios, lançamos mão de uma ferramenta gráfica, onde podemos 
visualizar todos os perfis da OEII. Nesta etapa, temos a oportunidade
de analisar os dados sob uma ótica física, ou oceanográfica, fugindo da abordagem puramente quantitativa.
A idéia aqui é detectar problemas de qualidade que passaram por todos os outros filtros, sob argumentos puramente 
associados ao comportamento típico dos escoamentos oceânicos. As correntes oceânicas têm uma forma razoavelmente 
contínua e suave de variar espacialmente. Procuramos aqui por mudanças bruscas e isoladas na direção e/ou
magnitude de velocidade em pequenos trechos, ou evidências sérias de descontinuidades no escoamento, como
grandes divergências ou convergências em áreas limitadas. Esta é a oportunidade de inspecionar os dados de forma 
crítica e minuciosa, melhorando a qualidade do conjunto final.


\subsubsection{Gradeamento Espacial}\label{sec:grid}

\hspace{6mm} Tendo finalmente os dados corrigidos e com garantia de qualidade, acreditamos ser 
interessante a promediação espacial dos dados, acarretando em um equi-espaçamento horizontal dos mesmos. 
Dessa forma, transformamos os dados que estão no domínio {\bf $(x,y,t)$} para o domínio {\bf $(x,y)$}. Esta 
abordagem é a ideal para o presente trabalho, pois pretendemos interpretar os resultados sob o ponto de 
vista sinótico. Consideraremos que o cenário observado durante OEII tem caráter sinótico
para as escalas de movimento que estamos interessados. Para sustentar tal afirmação, será necessária, 
mais adiante, a execução de procedimentos de gradeamento e interpolação que promovam uma filtragem 
espacial nestes dados. Estes procedimentos serão descritos em detalhe no Capítulo \ref{cap:funccorr}. O importante
aqui é deixar os dados preparados para tal, e isso consiste nesta promediação horizontal dos perfis em 
intervalos regulares de latitude e longitude ao longo da derrota do navio. 

Finalmente, após todos estes quatro procedimentos, os dados estão prontos para as etapas subseq\"uentes, de 
acordo com os objetivos propostos na Seção \ref{sec:obj}. Antes de
dar prosseguimento às etapas seguintes, acreditamos ser interessante exibir um trecho específico do levantamento
correntográfico como exemplo para o resultado de todo tratamento conduzido. 

A Figura \ref{fig:ex_calib} retrata mapas horizontais de vetores de velocidade em 80 m de profundidade em
um trecho peculiar da OEII, onde notamos problemas nos dados associados à referida limitação da
GIRO. Nesta figura, observamos no painel superior o trecho antes de qualquer calibração ou tratamento. Neste caso, foi 
apenas calculada a velocidade absoluta da corrente para fins de comparação. O painel inferior consiste nos dados após
todo o tratamento descrito no decorrer da presente seção. Na Figura \ref{fig:ex_calib}, observamos duas mudanças 
bruscas de trajetória do navio e algumas 
oscilações na trajetória correspondentes à realização de estações oceanográficas. Como podemos notar, os vetores brutos
apresentaram mudanças significativas na direção após cada guinada brusca do navio. Nos dados tratados é evidente 
a melhoria de qualidade do campo de velocidade,
fruto da calibração realizada. Vemos aqui que, apesar das limitações e aproximações utilizadas,
o resultado final se mostra com melhor qualidade do que os dados brutos. 

\begin{figure}
 \begin{center}
  \includegraphics[width=12cm,keepaspectratio=true]{../figuras/ex_calib.pdf}
  \includegraphics[width=12cm,keepaspectratio=true]{../figuras/ex_calib2.pdf}
 \end{center}
 \vspace{-.25cm}
 \renewcommand{\baselinestretch}{1}
 \caption{\label{fig:ex_calib} \small Comparação qualitativa entre os dados de ADCP brutos (painel superior)
 e após o pré-processamento (painel inferior).
Estão representados os vetores de velocidade em aproximadamente 80 m de profundidade. 
Os círculos amarelos indicam a posição das estações oceanográficas realizadas durante a OEII.}
\end{figure}

Todos os procedimentos e cálculos efetuados descritos a partir da Seção \ref{sec:wtrtrack} foram conduzidos com o 
auxílio do ``Common Ocean Data Access System (CODAS)''. O CODAS consiste em um pacote de programas desenvolvidos na 
Universidade do Havaí com a finalidade de uniformizar os procedimentos de processamento e armazenamento de dados 
de ADCP de casco e encontra-se disponível para a comunidade científica no website da universidade.

Estando os dados termohalinos e de velocidade prontos para serem trabalhados, prosseguiremos no próximo capítulo
com a descrição detalhada da metodologia ci\-en\-tí\-fi\-ca aplicada para construir os campos que nos fornecerão 
subsídios para responder às perguntas formuladas no Capítulo \ref{cap:intro}.


