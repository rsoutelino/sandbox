\section{Contextualização Teórica}\label{sec:intteo}

\hspace{6mm} Dois terços da superfície do globo terrestre são ocupados pelos oceanos. 
As dimensões das bacias oceânicas são de escala planetária, 
assim como o ar amosférico que está em nossa volta. Sem a atmosfera e os oceanos não haveria
vida em nosso planeta. Os movimentos dos fluidos nesses sistemas são de vital importância e entendê-los
é uma necessidade \citep{cushman1994}. Estes dois grandes compartimentos 
do globo estão em constante movimento, o qual é governado por leis 
físicas que se aplicam a estes fluidos. A {\it Oceanografia 
Física} se destina a estudar os mais
diversos movimentos que ocorrem nos oceanos, desde as ondas de gravidade de superfície que vemos quebrar nas praias até 
as grandes correntes oceânicas que se estendem por milhares de quilômetros. Os movimentos no oceano
são basicamente governados por uma equação de natureza vetorial, a Equação de Navier-Stokes, que traduz a
segunda lei de Newton aplicada aos fluidos.

Devido à elevada diversidade dos
fenômenos que ocorrem nos oceanos, não existe uma solução geral para a Equação de Navier-Stokes.
 Por consequência, para que possamos estudar cada fenômeno oceânico, existe a necessidade de promover
sim\-pli\-fi\-ca\-ções matemáticas/físicas que viabilizem o estudo e o entendimento de cada tipo de movimento em particular. 
Com isso, torna-se comum a divisão dos movimentos oceânicos em diferentes escalas
espaciais e temporais. Na Oceanografia Física atual, tanto para estudos teóricos quanto 
para análises observacionais ou simulações numéricas, é comum dividir os movimentos 
usualmente em quatro escalas fe\-no\-me\-no\-ló\-gi\-cas: {\bf micro, pequena, meso e grande escalas}. 

Os movimentos ditos de {\bf pequena escala} são os que possuem comprimento típico de poucos
metros a algumas dezenas de quilômetros e períodos de variabilidade de poucos segundos
até diária. Neste tipo de movimento, por exemplo, o efeito da rotação da Terra
se torna irrelevante dinamicamente, como é o caso das ondas curtas de gravidade de superfície, anteriormente citadas. 
Os movimentos de {\bf meso-escala} começam a sofrer influência da rotação da terra, dada
sua dimensão e período de variabilidade, que são respectivamente da ordem de centenas
de quilômetros e até centenas de dias. É o caso das correntes de contorno (oeste e leste), meandros, 
vórtices, bifurcações, retroflexões, etc. A terceira é a {\bf grande escala}, que 
contém os movimentos em escala de bacia oceânica, da ordem de milhares de quilômetros de 
comprimento e de caráter temporal superiores a uma centena de dias. 

% É importante se ter em mente que essa é uma divisão meramente didática. O que
% na verdade observamos na natureza é a soma de todas essas escalas se traduzindo em
% um movimento único. Porém, em determinadas regiões do oceano, conseguimos praticamente
% isolar um determinado tipo de movimento a partir de uma escala fenomenológica específica, contanto 
% que suas forçantes sejam as principais geradoras do escoamento, simplificando assim
% o estudo. Através dessas simplificações e o conhecimento das forçantes geradoras de movimento, é
% possível, em determinados casos, entender um determinado padrão de circulação através da aplicação
% das leis e equações da dinâmica de fluidos geofísicos.

Em todas as escalas mencionadas, a circulação atmosférica é uma importante for\-çan\-te
geradora dos movimentos.  Modelos teóricos foram desenvolvidos e puderam 
explicar determinados tipos de movimento nos oceanos. Uma das primeiras teorias importantes
desenvolvidas no tocante à explicação física de movimentos observados no oceano mundial,
 através de simplificações do sistema de equações governantes para movimentos de grande escala, foi a 
de \cite{sverdrup1947}. Este autor mostrou como o oceano responde aos centros de alta pressão atmosférica 
de latitudes subtropicais 
que sopram sobre os Oceanos Atlântico e Pacífico, sul e norte. A resposta, {\bf indiretamente},
acaba sendo a geração de uma circulação com giro anticiclônico no oceano superior,
e são comumente referidos como {\it Giros Subtropicais}. Estes giros de circulação oceânica já tinham sido 
observados antes do desenvolvimento da teoria de \cite{sverdrup1947}. Eles eram identificados
observacionalmente, através dos métodos disponíveis na época. 

Os giros oceânicos, diferentemente dos atmosféricos, são assimétricos, de forma
que as correntes no lado oeste dos oceanos são confinadas em um jato estreito
e mais intenso do que no restante do giro, sendo denominadas {\it Correntes de Contorno
Oeste} (CCOs) \citep{stommel1948,munk1950}. Na seção seguinte revisaremos o conhecimento adquirido até a presente data
sobre a circulação de {\bf grande escala} no Atlântico Sul.

\section{Padrões de Circulação de Grande Escala no Atlântico Sul}\label{sec:largesc}

\hspace{6mm} No Oceano Atlântico Sul, a circulação de grande escala nos primeiros 1000 m é dada por um grande 
giro anticiclônico, que de acordo com \cite{stramma_england1999} é limitado meridionalmente pela Corrente do Atlântico Sul em seu limite sul e a Corrente Sul Equatorial (CSE) em seu limite norte. A borda leste
desse oceano é ocupada pela Corrente de Benguela. A CCO
que completa o giro é a Corrente do Brasil (CB).

Apesar de parecer um padrão simples
de escoamento, a chegada da CSE na margem continental brasileira é complexa. Ao atingir a margem continental, 
esta corrente é forçada a se bifurcar. Tanto o giro subtropical quanto a bifurcação
da CSE (BiCSE) situam-se geograficamente em localidades diferentes em cada nível de profundidade, originando 
um complexo padrão de circulação, com diversas CCOs. A CSE é descrita por \cite{wienders_etal2000} como um fluxo 
para oeste horizontalmente extenso, cruzando o Oceano Atlântico até a costa brasileira. 

De forma a simplificar a descrição desses padrões, dividiremos a porção do oceano sobre a margem continental
brasileira em três camadas principais, 
conforme descrito nos esforços de \cite{stramma_england1999}. Denominaremos camada de  
superfície, os primeiros 150 m de coluna de água; picnoclina, entre 150 e 500 m;
e de camada intermediária, entre 500 e 1000 m. Estes domínios verticais, estão associados 
às principais massas de água do oceano superior no Atlântico
Sul, sendo a Água Tropical (AT) na superfície, a Água Central do Atlântico Sul (ACAS) na camada da
picnoclina e a Água Intermediária Antártica (AIA) na camada intermediária.  

Ainda, é comumente observado na literatura a divisão da CSE em três ramos principais \citep{stramma1991}:
o ramo norte (CSEn), o ramo central (CSEc) e o ramo sul (CSEs). No decorrer da presente dissertação, toda a descrição acerca
da BiCSE referir-se-á à CSEs. Segundo \cite{stramma_england1999}, A porção de superfície
deste ramo atinge a margem continental 
brasileira em torno de 15$^\circ$S. Em nível picnoclínico, isto ocorre em 20$^\circ$S.
Em nível intermediário, aproximadamente em 25$^\circ$S. 

Em todas as camadas, há assinatura de que a CSEs se bifurca em duas CCOs: 
uma seguindo em direção ao equador, e outra seguindo em direção ao pólo sul. Isto 
forma um complexo e estratificado sistema de escoamento. Tal sistema é composto
pela CB, pela Corrente de Contorno Intermediária (CCI), pela Sub-corrente Norte
do Brasil (SNB) e pela Corrente Norte do Brasil (CNB).

A CB, originada através da BiCSE em superfície, em 15$^\circ$S, flui para sul ocupando a camada de superfície
até os entornos de 20$^\circ$S, onde recebe um aporte da BiCSE em nível picnoclínico,
espessando-a verticalmente. Em 25$^\circ$S, a CB recebe um novo aporte, dessa vez da
BiCSE em nível intermediário, e passa a fluir para sul ocupando praticamente toda a 
coluna de água do oceano superior (Figura \ref{fig:CB_3D}). 

\begin{figure}%[ht]
 \begin{center}
  \includegraphics[width=15.5cm,keepaspectratio=true]{../figuras/CB_3D.pdf}
 \end{center}
 \vspace{-.5cm}
 \renewcommand{\baselinestretch}{1}
 \caption{\label{fig:CB_3D} \small Síntese da origem e do escoamento da CB ao longo 
da margem continental brasileira, de acordo com os padrões esquemáticos de grande escala de \cite{stramma_england1999}.}
\end{figure}

A CCI se origina através da BiCSE em nível intermediário ao sul de 25$^\circ$S, flui para norte ao longo do contorno oeste, 
até que em 20$^\circ$S recebe o ramo norte da BiCSE em nível picnoclínico. Com a adição deste escoamento, 
passa a ser denominada SNB.
O ramo norte da BiCSE em superfície soma-se ao fluxo da SNB e forma a CNB ao norte de 10$^\circ$S (Figura \ref{fig:CCI_3D}). 

Existe uma variabilidade temporal relativa à localização deste sistema de bifurcação, o que
pode ter im\-pli\-ca\-ções na variabilidade climática global. Isto se dá porque 
o sistema CSE-SNB-CNB é o principal condutor para o retorno de águas da {\it Célula de Revolvimento Meridional} (CRM) \citep{talley2003,ganachaud2003,lumpkin_speer2003}. A localização precisa da BiCSE pode prover informações
acerca da quantidade de água que é exportada para o Atlântico Norte e recircula no giro subtropical.

\begin{figure}%[ht]
 \begin{center}
  \includegraphics[width=15.5cm,keepaspectratio=true]{../figuras/CCI_3D.pdf}
 \end{center}
 \vspace{-.5cm}
 \renewcommand{\baselinestretch}{1}
 \caption{\label{fig:CCI_3D} \small Síntese da origem e do escoamento de CCI, SNB e CNB ao longo 
da margem continental brasileira, de acordo com os padrões esquemáticos de grande escala de \cite{stramma_england1999}.}
\end{figure}

Existe ainda uma quarta camada que devemos considerar, que ocupa as porções do sopé continental
e parte da planície abissal: é a camada profunda. Esta é composta por três massas de água: 
a Água Circumpolar Superior (ACS), a Água Profunda do Atlântico Norte (APAN) e a Água 
Circumpolar Inferior (ACI). Apesar de ser considerada uma massa de água profunda, a ACS parece
fluir de forma consonante à AIA e consiste no limite inferior do Giro Subtropical 
\citep{stramma_england1999,memery_etal2000}. Devido ao desconhecimento dos padrões associados
à ACS próximo ao contorno oeste, consideraremo-os similares ao da AIA neste trabalho. 

Entre 1500-3000 m, os movimentos da APAN se dão na forma de um escoamento organizado conhecido 
como Corrente de Contorno Profunda (CCP), que flui para o sul ao largo de todo o contorno
oeste abaixo do sistema previamente descrito. Ao sul da BiCSE em nível intermediário, 
observamos que toda a coluna de água entre a superfície e 3000 m flui integralmente 
em direção ao pólo sul \citep{zemba1991,silveira_etal2000A}. Os escoamentos da ACI estão 
associados aos movimentos da Água de Fundo Antártica (AFA) e será considerado neste trabalho
como parte da circulação de fundo. A Figura \ref{fig:brasil_3D}
sintetiza o padrão de grande escala de todo este complexo sistema de correntes de 
contorno oeste, residente na margem continental brasileira.

\begin{figure}%[ht]
 \begin{center}
  \includegraphics[width=15.5cm,keepaspectratio=true]{../figuras/brasil_3D_new.pdf}
 \end{center}
 \vspace{-.5cm}
 \renewcommand{\baselinestretch}{1}
 \caption{\label{fig:brasil_3D} \small Síntese do escoamento do sistema de correntes 
de contorno oeste ao longo da margem continental brasileira, de acordo com os padrões esquemáticos de grande escala de \cite{stramma_england1999}.}
\end{figure}

A BiCSE como uma feição de grande escala foi muito pouco explorada por pesquisadores até a presente data no Atlântico Sul. A primeira 
suspeita que se gera na busca de uma forçante controladora de sua posição é a variabilidade associada à localização da faixa de 
rotacional nulo da tensão de cisalhamento dos ventos de grande escala que forçam o giro subtropical. Essas faixas, segundo a teoria de \cite{sverdrup1947},
delimitam os giros de grande escala forçados pelo vento. Com isso, seria conveniente associar a posição e 
variabilidade da BiCSE com a localização e variabilidade da faixa de rotacional nulo da tensão de cisalhamento do vento. 
Porém, \cite{qu_lukas2003} afirmam que essa faixa de rotacional nulo falha ao explicar a localização e variabilidade 
desta feição, pois além da teoria de \cite{sverdrup1947} considerar um oceano barotrópico, ela se furta a abordar as trocas entre os giros tropical e subtropical. Estes autores analisaram a eficiência da linha de rotacional nulo do vento como
indicador da variabilidade da BiCSE no oceano Pacífico. 

O único trabalho até então publicado que buscou fornecer uma descrição detalhada da 
BiCSE em grande escala e investigar sua variabilidade sazonal deve-se à \cite{rodrigues_etal2006}. Estes autores, através
do cálculo de velocidades geostróficas baseadas em um campo de massa climatológico anual para o Atlântico Sul e posteriores 
experimentos numéricos, confirmaram a migração da BiCSE para sul com o aumento da profundidade (Figura \ref{fig:rodrigues}). Eles dividiram esta migração 
em 7$^\circ$ de latitude nos primeiros 400 m e mais 6$^\circ$ até os 800 m de profundidade, totalizando 14$^\circ$ nos primeiros
1000 m de coluna de água. Os autores observaram que no Pacífico esta migração não excede 8$^\circ$, portanto atribuíram
à presença da CRM a maior migração para sul observada no Atlântico Sul. 

Ainda descrevendo os resultados de \cite{rodrigues_etal2006}, destacamos que
através de suas simulações numéricas os autores puderam constatar que a variabilidade associada à posição da BiCSE se restringe aos 
primeiros 400 m da coluna de água, o que corresponde grosseiramente à camada de mistura somada a região da termoclina
ventilada. Os autores atribuíram como principal mecanismo controlador dessa variabilidade a tensão de cisalhamento do vento remota, 
isto é, o efeito do vento em escala de giro subtropical. 


Os padrões de escoamento descritos até aqui se referem a um cenário de 
grande escala, ou seja, o que seria um padrão médio de circulação oceânica. Para entender a complexa dinâmica desse sistema, 
são necessárias investigações de cunho regional, que explorem em maior riqueza
de detalhe a atividade de meso-escala dessas feições.

\begin{figure}%[hb]
 \begin{center}
  \includegraphics[width=15cm,keepaspectratio=true]{../figuras/rodrigues_fig2.png}
 \end{center}
 \vspace{-.5cm}
 \renewcommand{\baselinestretch}{1}
 \caption{\label{fig:rodrigues} \small Anomalia do geopotencial média anual (x 10$^{-1}$ m$^2$ s$^{-2}$) e fluxo 
geostrófico relativo a 1000 dbar em 0, 100, 200, 400, 600 e 800 m no Atlântico Sul de acordo com \cite{rodrigues_etal2006}. Os círculos pretos representam a localização da Bi\-CSE.}
\end{figure}

\newpage

\section{Padrões Sinóticos de Circulação na Margem Continental Sudeste Brasileira}\label{sec:mesoesc}

\hspace{6mm} A circulação de meso-escala ao longo da margem continental brasileira já foi razoavelmente
investigada ao largo das costas sul \citep{olson_etal1988}, sudeste \citep{silveira_etal2000A} e nordeste 
\citep{silveira_etal2000B}. São encontradas
na literatura informações acerca da estrutura vertical e horizontal, 
meandramentos e até variabilidade temporal das CCOs para as regiões citadas.
Entretanto, a costa leste, sítio de origem da CB, ainda foi muito pouco
explorada e inexistem trabalhos de cunho observacional que abordem a descrição das estruturas de
 meso-escala ligadas à origem e organização da CB.  

Estudos descritivos e de processos dinâmicos, relacionados à importante
atividade de meso-escala no sudeste, sejam através de manipulação de observações ou por
experimentos numéricos, são recentes \citep{schmid_etal1995,lima1997,velhote1998,campos_etal2000,calado2001,fernandes2001,silveira_etal2004,godoi2005,
calado2006,mattos2006,silveira2006}. Através da compilação dos principais resultados obtidos pelos 
autores citados, notamos que há um número razoável de evidências da estrutura
tridimensional do sistema de correntes de contorno oeste ao largo do sudeste,
e que estas vêm a corroborar os padrões médios descritos por \cite{stramma_england1999}.

Acreditamos ser interessante citar os resultados recentes de \cite{mattos2006},
que confirmou através de dados {\it in situ} as hipóteses levantadas por \cite{tsuchiya1985} 
e \cite{vianna_menezes2005}, ou seja, a existência de uma célula de recirculação 
da CB na Bacia de Santos (Figura \ref{fig:subgiro}). O trabalho de \cite{mattos2006} mostrou ainda, através
da formulação de um modelo quase-geostrófico, que esta célula consiste numa estrutura de
grande escala, de caráter quase-estacionário. Este autor sintetiza que o cenário sinótico 
ao largo do sudeste é composto por estruturas de meso-escala associadas a ondas baroclínicas 
de vorticidade, e uma estrutura de grande escala mais robusta, associada a célula de recirculação da CB.

\cite{silveira2006} se dedicou a estudar detalhadamente o Sistema 
Corrente do Brasil na Bacia de Campos, RJ, analisando diversos aspectos.
Este autor mostrou ser predominantemente baroclínica a estrutura do sistema
de correntes, através da comparação de campos de velocidade puramente geostrófica
e de velocidade observada diretamente em uma mesma seção hidrográfica. 

\begin{figure}%[ht]
 \begin{center}
  \includegraphics[width=12cm,keepaspectratio=true]{figuras/regiao_esquema_iso.png}
 \end{center}
 \vspace{-.5cm}
 \renewcommand{\baselinestretch}{1}
 \caption{\label{fig:subgiro} \small Resumo esquemático do cenário oceanográfico 
quase-sinótico, ao largo do sudeste brasileiro, de acordo com \cite{mattos2006}. 
As ondas baroclínicas de vorticidade, tipicamente de meso-escala, são representadas
 pelas estruturas ciclônicas e anticiclônicas. O sinal mais robusto é aquele de grande
escala da célula de recirculação da CB, evidenciando a separação parcial desta do contorno 
oeste.}
\end{figure}

Tal resultado é particularmente importante, pois medidas diretas de velocidade são
difíceis de serem tomadas no oceano, e esta informação permite que inferências
 a partir do campo de massa sejam consideradas uma boa representação do escoamento
geostrófico total. 

Ainda, motivado pelo pequeno número de Rossby encontrado para
o sistema em suas análises, o autor aplicou a aproximação quase-geostrófica ao 
sistema através de uma decomposição modal. Mostrou que na Bacia de Campos, este sistema
é dominado pelo primeiro modo baroclínico, essencialmente caracterizado pela
CB fluindo para sul e a CCI fluindo para norte. O autor explica que a ausência da participação dinâmica da
CCP se deve ao fato da presença  da feição topográfica
conhecida como Platô de São Paulo, afastando-a para o largo.
Isto não ocorre em regiões ao norte de 20$^\circ$S.

A atividade de meso-escala associada à  CB, principalmente, já foi investigada por diversos autores, do ponto 
de vista descritivo e dinâmico no sudeste brasileiro. O vigoroso meandramento da CB é observado desde
a costa de Vitória até a Bacia de Santos, com a descrição de vórtices recorrentes, 
como o Vórtice de Vitória \citep{schmid_etal1995}, o Vórtice de São Tomé \citep{calado_etal2006} 
e o Vórtice de Cabo Frio \citep{calado2006}.  Ademais, \cite{campos_etal1995} e 
\cite{pereira2005} descreveram evidências de formação de um dipolo vortical nas imediações de Cabo Frio
e na Bacia de Santos.

Alguns trabalhos se dedicaram a investigar as causas dinâmicas do meandramento
da CB. Seguindo a vertente de pesquisa voltada à simulação numérica de processos oceânicos,
 podemos destacar alguns poucos trabalhos, como \cite{velhote1998}, \cite{goncalves2000}
e \cite{calado2001}. Esse último buscou, através de simulações numéricas prog\-nós\-ti\-cas 
com o {\it Princeton Ocean Model} (POM) \citep{blumberg_mellor1987}, compreender o papel da mudança 
de orientação da costa e quebra de plataforma nas proximidades de Cabo Frio na geração
e crescimento dos meandros baroclínicos da CB. O autor, então, conduziu simulações
empregando configurações batimétricas distintas, ora realista, ora idealizada onde o talude
continental fora aproximado por uma parede vertical que se estendia até uma bacia abissal
de 2000 m de profundidade. Este autor mostrou que os meandros gerados são 
mais sensíveis à mudança de orientação da costa do que às mudanças de topografia.
Adicionalmente, os experimentos com topografia realista denotaram um
padrão meandrante da CB que em várias ocasiões muito se assemelha aos observados
em imagens de satélite.

\cite{soares2007} conduziu um
experimento idealizado buscando investigar a influência do Banco de Abrolhos
na instabilidade do Sistema Corrente do Brasil ao sul de 20$^\circ$S, e tinha como hipótese que 
esta feição topográfica poderia ser um mecanismo de disparo para o meandramento
subseq\"uente. O autor realizou experimentos prognósticos partindo de uma base de dados
climatológica, adicionando modelos paramétricos para incluir o padrão de meso-escala do sistema de correntes 
de contorno. Seus resultados mostraram que uma topografia semelhante a da região
pode favorecer a ocorrência de instabilidades que geram o meandramento observado no sudeste.

Até esta parte deste capítulo, procuramos sintetizar o conhecimento adquirido 
até a presente data sobre o escoamento de contorno oeste na costa sudeste 
brasileira. Já adiantando que a porção oceânica adjacente à costa leste é nosso objeto de estudo, 
dedicaremo-nos a seguir a detalhar o estado da arte da região de interesse.
  

\section{Padrões Sinóticos de Circulação na Margem Continental Leste Brasileira}\label{sec:cleste}

\hspace{6mm} A margem continental leste brasileira é definida oceanograficamente entre a Cadeia Vitória-Trindade (21$^\circ$S)
e a foz do Rio São Francisco (10,5$^\circ$S). A região abriga um dos ecossistemas mais complexos e biodiversos
do planeta: o sistema coralíneo do Banco de Abrolhos. Geomorfologicamente, 
a região apresenta características peculiares que influem diretamente no escoamento de correntes oceânicas.
A plataforma continental é estreita em sua maior parte. Expande-se e atinge mais de 200 km
na região do Banco de Abrolhos \citep{zembruscki1979}. O talude é bastante mais íngreme se comparado
com a região sudeste. A topografia da bacia abissal se intercala 
com bancos que se estendem até as proximidades da superfície, com declividades 
acentuadas. Essa complexa topografia é ilustrada na Figura \ref{fig:batimetria}.

\begin{figure}[hb]
 \begin{center}
  \includegraphics[width=16cm,keepaspectratio=true]{../figuras/mapa_leste.png}
 \end{center}
 \vspace{-.5cm}
 \renewcommand{\baselinestretch}{1}
 \caption{\label{fig:batimetria} \small Fisiografia da margem continental e bacia abissal adjacente
a costa leste brasileira, extraída da base de dados ETOPO 2.}
\end{figure}
 

Sintetizemos o que consta na literatura a respeito da circulação de meso-escala
nesta região topograficamente complexa. Dentre os poucos trabalhos, podemos
evidenciar o pioneirismo de \cite{miranda_castro1981}, que amostraram uma CB baroclínica mais rasa do que a descrita
no sudeste, através do campo de massa obtido por garrafas de Nansen, atravessando uma
radial hidrográfica localizada em 19$^\circ$S. Estes utilizaram o {\it Método Dinâmico Clássico} \citep{sandstrom_helland1903} com um 
nível de referência situado em 480 dbar em média. A CB amostrada tinha velocidades máximas da 
ordem de 0,72 m s$^{-1}$ em seu núcleo na superfície, e transportava para o sul 6,5 Sv.

Mais recentemente, nos mesmos 19$^\circ$S, \cite{soutelino2005} descreveu uma estrutura da CB muito semelhante (Figura \ref{fig:soutelino19}),
nesta ocasião utilizando um conjunto de dados de alta resolução vertical, obtidos por CTD em 
outubro de 2001. Sua estratégia para o cálculo das velocidades baroclínicas foi o uso da versão 
seccional do {\it Princeton Ocean Model} (POMsec), que calculava as velocidades sem dependência de um nível de referência. 
Foi encontrada uma CB fluindo para sul com núcleo em superfície e espessura
vertical de 300 m. Suas  velocidades máximas eram da ordem de 0,6 m s$^{-1}$ e transportava
5,1 Sv. 

\begin{figure}[hb]
 \begin{center}
  \includegraphics[width=15cm,keepaspectratio=true]{/home/rafaelgs/monografia/radial_13/figuras_rad13/vel_pom_rad13.pdf}
 \end{center}
 \vspace{-.5cm}
 \renewcommand{\baselinestretch}{1}
 \caption{\label{fig:soutelino19} \small Seção vertical de velocidades baroclínicas absolutas em
19$^\circ$S, em outubro de 2001, segundo \cite{soutelino2005}.}
\end{figure}

O autor pôde também observar uma corrente de contorno oeste fluindo logo abaixo, em sentido 
oposto, o qual sugeriu ser a SNB após perceber que esta transportava ACAS em direção ao norte. 
Esta corrente tinha núcleo em aproximadamente 700 m de profundidade e seu fluxo se estendia até
1300 m. As velocidades máximas eram da ordem de 0,25 m s$^{-1}$ e o transporte para norte, de 4,1 Sv. 
Apesar de não ter sido abordado pelo autor em seu trabalho, inspecionando a Figura \ref{fig:soutelino19}
encontramos um fluxo para norte com intensidade comparável a da CB, sugerindo a 
existência de um anticiclone. 

\begin{figure}%[ht]
 \begin{center}
  \includegraphics[width=12cm,keepaspectratio=true]{figuras/psi_abrolhos1.pdf}
  \includegraphics[width=12cm,keepaspectratio=true]{figuras/psi_abrolhos2.pdf}
 \end{center}
 \vspace{-.5cm}
 \renewcommand{\baselinestretch}{1}
 \caption{\label{fig:abrolhos} \small Função de corrente geostrófica em 10 m de 
profundidade, relativa a 1000 dbar, evidenciando
a circulação ao largo do Banco de Abrolhos e adjacências, 
segundo \cite{silveira_etal2006B}. Painel superior: setembro de 2004; painel inferior: março de 2005.}
\end{figure}

Provavelmente, este mesmo anticiclone foi também capturado por \cite{silveira_etal2006B}, através 
de uma análise geostrófica tridimensional com o auxílio de dados hidrográficos
obtidos em uma grade horizontal de alta resolução (18-21$^\circ$S), na região de Abrolhos
no verão de 2005 (Cruzeiro Abrolhos 2). Nesta ocasião,
os autores puderam observar a estrutura completa deste possível vórtice, 
residente ao largo do Banco de Abrolhos (Figura \ref{fig:abrolhos}). Por conseq\"uência, 
denominaram-no de Vórtice de Abrolhos (VAb).
\cite{silveira_etal2006B} ainda 
tiveram a oportunidade de amostrar a região novamente no inverno de 2004 (Cruzeiro Abrolhos 1),
quando encontraram um padrão não-meandrante, propondo o seguinte cenário.
Ao sul de 15$^\circ$S, a CB é uma corrente rasa, estreita, e transporta cerca
de 1,5-3 Sv para o sul, junto à quebra de plataforma e com velocidade máxima da
ordem de 0,5 m s$^{-1}$. A SNB flui em direção oposta, entre 200-1200 m
de profundidade transportando cerca de 12,3 Sv para o norte, com 0,3 m s$^{-1}$
em seu núcleo. Mais ao sul, a CB se trifurca ao atravessar a Cadeia Vitória-Trindade.
Após essa passagem, ocorre a reorganização da CB dentro do Embaiamento do Espírito
Santo, podendo ela então assumir um padrão meandrante (Vórtice de Vitória, \cite{schmid_etal1995}) ou um
padrão de jato \citep{evans_etal1983}.

% Estas não são as únicas evidências acerca da estrutura tridimensional deste anticiclone. 
% Resultados recentes de \cite{soutelino_etal2007}, mostraram através de velocidades observadas
% o mesmo anticiclone ao largo do Banco de Abrolhos.
% A Figura \ref{fig:proab1} mostra o campo de função de corrente
% observada, calculada através de dados de ADCP de casco e posterior mapeamento com remoção da componente divergente 
% do vetor velocidade. Estes dados foram obtidos em setembro de 2007, em uma grade quase-sinótica
% compreendida entre 17-22$^\circ$S (Cruzeiro PRO-ABROLHOS Hidrografia I2007). De acordo com a figura,
%  os autores confirmam os resultados encontrados
% por seus antecessores \citep{silveira_etal2006B}.
% Os resultados mostram uma CB fluindo para sul com intenso meandramento, bifurcando-se ao 
% atravessar a Cadeia Vitória-Trindade, que se reorganiza assumindo um padrão de jato. Considerando a época do ano, 
% constatamos ser coerente a não-formação do Vórtice de Vitória, pois o mesmo só foi encontrado
% em perídos de verão de acordo com os trabalhos anteriores descritos previamente. Além do VAb, 
% \cite{soutelino_etal2007} encontraram, adjacente ao mesmo, um ciclone, que forma junto com o VAb
% um dipolo vortical. Foram encontradas ainda evidências de um outro anticiclone ao norte do 
% Bando de Royal-Charlotte, o qual foi denominado Vórtice de Royal-Charlotte (VRC) pelos autores.
% 
% \begin{figure}%[ht]
%  \begin{center}
%   \includegraphics[width=10cm,keepaspectratio=true]{figuras/psiob_proab1.pdf}
%  \end{center}
%  \vspace{-.5cm}
%  \renewcommand{\baselinestretch}{1}
%  \caption{\label{fig:proab1} \small Em setas vermelhas, desenho esquemático interpretativo da circulação observada 
%  entre 17-22$^\circ$S através de dados de ADCP de casco por \cite{soutelino_etal2007}. Subposto ao esquema
%  encontra-se o campo de função de corrente observada oriundo da remoção da divergência horizontal
%  do campo original de velocidades obtidas via ADCP.}
% \end{figure}


Há uma lacuna de informação na literatura, entre 10-17$^\circ$S. 
Em aproximadamente 10$^\circ$S, quatro trabalhos \citep{silveira_etal1994,stramma_etal1995,soutelino2005,schott_etal2005}
descreveram de forma muito semelhante a estrutura vertical do sistema de correntes,
com uma SNB centrada em 200 m, e aparente ausência da estrutura da CB. Estes autores
inferiram as velocidades através de quatro metodologias diferentes: o Método Dinâmico
Clássico, referenciado em 1000 dbar \citep{silveira_etal1994} (Figura \ref{fig:silveira94}); 
medidas diretas de velocidade via ADCP de casco \citep{stramma_etal1995} 
(Figura \ref{fig:stramma95}); aplicação do POMsec a dados hidrográficos \citep{soutelino2005} (Figura \ref{fig:soutelino10})
e fluxo médio a partir de uma seção de fundeios em 11$^\circ$S entre 2000 e 2004 \citep{schott_etal2005} (Figura \ref{fig:schott_etal2005}).
Destacamos que estes últimos autores tiveram ainda a oportunidade de registrar a presença da CCP, fluindo para sul 
em torno de 2000 m. 

\newpage

\begin{figure}%[ht]
 \begin{center}
  \includegraphics[width=12cm,keepaspectratio=true]{../figuras/secaoCNB_silveira94.pdf}
 \end{center}
 \vspace{-.5cm}
 \renewcommand{\baselinestretch}{1}
 \caption{\label{fig:silveira94} \small Seção vertical de velocidades geostróficas 
relativas a 1000 dbar em 10,5$^\circ$S. Valores positivos para o norte. Adaptado de \cite{silveira_etal1994}.}
\end{figure}

\begin{figure}[hb]
 \begin{center}
  \includegraphics[width=15cm,keepaspectratio=true]{../figuras/secaoCNB_strammaetal95.pdf}
 \end{center}
 \vspace{-.5cm}
 \renewcommand{\baselinestretch}{1}
 \caption{\label{fig:stramma95} \small Seção vertical de velocidades (ADCP de casco) em 10$^\circ$S. 
Valores positivos para o norte. Adaptado de \cite{stramma_etal1995}.}
\end{figure}

\begin{figure}%[hb]
 \begin{center}
  \includegraphics[width=14cm,keepaspectratio=true]{/home/rafaelgs/monografia/radial_2/figuras_rad2/vel_pom_rad2.pdf}
 \end{center}
 \vspace{-.5cm}
 \renewcommand{\baselinestretch}{1}
 \caption{\label{fig:soutelino10} \small Seção vertical de velocidades baroclínicas absolutas em
10$^\circ$S, segundo \cite{soutelino2005}. Valores positivos para o norte.}
\end{figure}

\begin{figure}%[hb]
 \begin{center}
  \includegraphics[width=14cm,keepaspectratio=true]{../figuras/schott_etal2005.jpg}
 \end{center}
 \vspace{-.5cm}
 \renewcommand{\baselinestretch}{1}
 \caption{\label{fig:schott_etal2005} \small Seção vertical de velocidades médias observadas em
11$^\circ$S, segundo \cite{schott_etal2005}. Valores positivos para o norte.}
\end{figure}

\newpage

A revisão apresentada ao longo desta seção, comprovou o quão pouco conhecemos a estrutura
e variabilidade espaço-temporal das CCOs associadas ao 
giro subtropical do Atlântico Sul e ao Giro Equatorial. Mostramos também o quão pouco conhecemos
a respeito das feições de meso-escala em toda costa brasileira. Tecemos uma leitura
mais detalhada acerca dos trabalhos que procuraram descrever os padrões na 
costa leste brasileira. A ausência de trabalhos descritivos exatamente
na lacuna entre 11-17$^\circ$S, que segundo os padrões de grande escala abriga  
a região de origem da CB, motiva-nos a formular as seguintes questões: 

\begin{itemize}

\item[$\checkmark$] qual o padrão, em meso-escala, da chegada da CSE junto à margem continental brasileira leste?

\item[$\checkmark$] como se dá a formação e organização da CB sinoticamente como uma CCO, e em que latitude isto ocorre?

\item[$\checkmark$] ao largo da costa leste a CB exibe intensa atividade de meso-escala como no
sudeste? 

\item[$\checkmark$] como se dá a formação e organização da SNB sinoticamente ao longo da costa leste? 

\end{itemize}

Outro ponto que deve ser enfatizado é que as poucas informações existentes ao sul de 11$^\circ$S na literatura são 
calcadas em velocidades puramente baroclínicas, sejam estas geradas numericamente ou calculadas
pelo Método Dinâmico Clássico. Inferências que incluam a componente barotrópica são inexistentes. 

\section{Objetivos}\label{sec:obj}

\hspace{6mm} O objetivo central desta dissertação é a descrição da circulação ao largo da costa leste brasileira
entre 10$^\circ$S e 20$^\circ$S. Esta descrição será conduzida utilizando conjunto de dados climatológicos e sinóticos.
 Dado que a disponibilidade de dados observacionais sinóticos
 para a realização deste trabalho (a serem descritos no próximo capítulo) se restringe a meses integrantes
do verão austral, concentraremos nossas análises na referida estação do ano. 
Para atender a este objetivo central, formulamos os seguintes objetivos específicos:

\begin{itemize}

\item[$\checkmark$] identificar a posição da BiCSE climatológica em vários níveis e identificar sua taxa de migração para
sul no caso do verão austral;

\item[$\checkmark$] identificar a origem das CCOs CB e SNB do ponto de vista climatológico para o verão;

\item[$\checkmark$] descrever sinoticamente os padrões de escoamento associados ao aporte da 
CSE junto à margem continental brasileira;

\item[$\checkmark$] identificar a latitude onde a CB tem sua origem como uma CCO ao largo da 
margem continental brasileira, e como se dá sinoticamente sua organização;

\item[$\checkmark$] descrever sinoticamente o escoamento da CB ao largo da costa leste brasileira
e a atividade de meso-escala associada; 

\item[$\checkmark$] identificar o sítio de origem da SNB e descrever sinoticamente sua organização e 
escoamento ao largo da costa leste brasileira; 

\end{itemize}


Descreveremos no Capítulo \ref{cap:dados}, os dois conjuntos de dados utilizados
neste trabalho: um conjunto de dados climatológicos termohalinos e um conjunto de dados sinóticos termohalinos e de velocidade.
O Capítulo \ref{cap:funccorr} será dedicado à descrição da metodologia empregada na obtenção de campos  
de velocidade que subsidiam as análises. O Capítulo \ref{cap:resultados} se destinará a apresentar e discutir os 
resultados encontrados. O Capítulo \ref{cap:consideracoes} constará de um sumário dos resultados mais
relevantes e das conclusões encontradas.

































