\hspace{6mm} Este capítulo apresenta os resultados referentes aos padrões de escoamento obtidos
para a região de estudo tanto no âmbito climatológico (via análise geostrófica dos dados do 
WOA2001) quanto sinótico (via análise dinâmica dos dados da OEII). Os resultados obtidos 
são confrontados, comparados e discutidos à luz da literatura disponível para a região. 

\section{Cenário Climatológico} \label{sec:res_climatologico}

\hspace{6mm} Destinamos
esta seção à apresentação e discussão dos campos horizontais de $\psi$ oriundos dos
dados climatológicos provenientes do WOA2001. Lembramos aqui, que os campos apresentados
para o cenário de verão são derivados a partir do  campo termohalino médio obtido para os meses
 de janeiro, fevereiro e março. Enfatizamos ainda, 
que os padrões a serem apresentados se limitam à componente geostrófica baroclínica
do escoamento relativa à superfície isobárica de 1000 dbar (correspondente a um $\mathcal{NR}$ de 1000 m em 
coordenadas cartesianas).
No entanto é a obtenção de uma aproximação razoável da circulação geostrófica, que nos permite entender o cenário médio 
de escoamento do giro subtropical e da BiCSE. 

 Dado que o objetivo desta parte do trabalho
é localizar e descrever a BiCSE e escoamento médio das CCOs ao largo da margem continental brasileira, 
limitaremos a análise e discussão ao escoamento superior, ou seja, em profundidades mais rasas que o $\mathcal{NR}$. Esta 
porção da coluna de água é comumente referida na literatura como o ``Oceano Superior'', e é nela que 
se concentra o escoamento associado ao giro subtropical gerado pelo vento. Selecionamos então, cientes da
dependência vertical que a BiCSE possui, os níveis de 20, 200, 500 e 800 m para apresentação
dos campos de $\psi$. Estes campos correspondem às Figuras \ref{fig:woa_verao20}-\ref{fig:woa_verao800}.

Imediatamente, podemos comprovar através da inspeção das Figuras \ref{fig:woa_verao20}-\ref{fig:woa_verao800}
a dependência que a BiCSE possui em relação ao binômio profundidade - latitude, confirmando 
o cenário já descrito por \cite{stramma_england1999} e mais recentemente por \cite{rodrigues_etal2006} para campos médios anuais.
Entre o campo de 20 m (Figura \ref{fig:woa_verao20}) e o campo de 800 m (Figura \ref{fig:woa_verao800}), 
a localização da BiCSE migrou de aproximadamente 9$^\circ$S para 23$^\circ$S. 
Em 20 m (Figura \ref{fig:woa_verao20}), observamos em linhas gerais que a CSE é um escoamento não confinado. 
Esta corrente tem uma largura extensa, ou seja,  aproximadamente 700 km, enquanto flui
cruzando o oceano Atlântico Sul. Ao se aproximar da margem continental brasileira,
torna-se difícil apontar um ponto exato onde sua bifurcação efetivamente ocorre. O que observamos de fato é que 
seu fluxo se incorpora suavemente ao contorno oeste. Portanto, escolhemos aqui como ponto de sua localização,
 o trecho da costa brasileira onde há uma demarcada divergência nas CCOs. Ou seja, onde 
existe a origem dos escoamentos das CCOs em cada nível analisado. Uma vez que na metodologia discutida no 
Capítulo \ref{cap:funccorr} mostramos que ao aplicarmos o ajuste na $\mathcal{AO}$ para que as condições 
de contorno de \textit{Dirichlet} sejam respeitadas junto a isóbata que escolhemos para representar 
a quebra da plataforma continental, podemos considerar tal isóbata como uma linha de corrente
nos campos de $\psi$. Nesse caso, marcamos o ponto de bifurcação na linha de corrente que se orienta 
perpendicularmente à tal isóbata. 

Para o campo de 20 m este ponto
corresponde ao sítio médio de origem da CB e 
da CNB e se localiza em aproximadamente 9$^\circ$S para o verão climatológico. 
Através da inspeção da Figura \ref{fig:woa_verao20},
podemos constatar que a CB, já a partir dessa latitude, se organiza em um jato de contorno oeste típico. O que 
percebemos também é o ganho em termos de intensidade que esta corrente tem à medida que se dirige para o sul
e vai recebendo continuamente o aporte zonal da extensa CSE. Mais ao sul, nos entornos de 23$^\circ$S e
 38$^\circ$W, encontramos uma robusta assinatura da Célula de Recirculação
Norte da CB. Tal estrutura já foi descrita por \cite{tsuchiya1985} e \cite{mattos2006}
a partir de outros conjuntos de dados.

\begin{figure}%[hb]
 \begin{center}
  \includegraphics[width=16cm,keepaspectratio=true]{/home/rafaelgs/mestrado/proc/climatologia/figuras_larga_escala/psi_woa_jan,fev,mar_20m.pdf}
 \end{center}
 \vspace{-.25cm}
 \renewcommand{\baselinestretch}{1}
 \caption{\label{fig:woa_verao20} \small Função de corrente geostrófica $\psi$, oriunda dos campos 
  climatológicos do WOA\-2001, em 20 m de profundidade, nos meses de verão. O 
  $\mathcal{NR}$ adotado é igual a 1000 m. A região cinza próxima à margem continental denota 
  profundidades menores que o $\mathcal{NR}$ adotado. A envoltória vermelha delimita a área da 
 OEII. O círculo preto representa a posição da BiCSE.}
\end{figure}

\begin{figure}%[ht]
 \begin{center}
  \includegraphics[width=16cm,keepaspectratio=true]{/home/rafaelgs/mestrado/proc/climatologia/figuras_larga_escala/psi_woa_jan,fev,mar_200m.pdf}
 \end{center}
 \vspace{-.25cm}
 \renewcommand{\baselinestretch}{1}
 \caption{\label{fig:woa_verao200} \small Função de corrente geostrófica $\psi$, oriunda dos campos 
  climatológicos do WOA\-2001, em 200 m de profundidade, nos meses de verão. O 
  $\mathcal{NR}$ adotado é igual a 1000 m. A região cinza próxima à margem continental denota 
  profundidades menores que o $\mathcal{NR}$ adotado. A envoltória vermelha delimita a área da 
 OEII. O círculo preto representa a posição da BiCSE.}
\end{figure}

\begin{figure}%[ht]
 \begin{center}
  \includegraphics[width=16cm,keepaspectratio=true]{/home/rafaelgs/mestrado/proc/climatologia/figuras_larga_escala/psi_woa_jan,fev,mar_500m.pdf}
 \end{center}
 \vspace{-.25cm}
 \renewcommand{\baselinestretch}{1}
 \caption{\label{fig:woa_verao500} \small Função de corrente geostrófica $\psi$, oriunda dos campos 
  climatológicos do WOA\-2001, em 500 m de profundidade, nos meses de verão. O 
  $\mathcal{NR}$ adotado é igual a 1000 m. A região cinza próxima à margem continental denota 
  profundidades menores que o $\mathcal{NR}$ adotado. A envoltória vermelha delimita a área da 
 OEII. O círculo preto representa a posição da BiCSE.}
\end{figure}

\begin{figure}%[ht]
 \begin{center}
  \includegraphics[width=16cm,keepaspectratio=true]{/home/rafaelgs/mestrado/proc/climatologia/figuras_larga_escala/psi_woa_jan,fev,mar_800m.pdf}
 \end{center}
 \vspace{-.25cm}
 \renewcommand{\baselinestretch}{1}
 \caption{\label{fig:woa_verao800} \small Função de corrente geostrófica $\psi$, oriunda dos campos 
  climatológicos do WOA\-2001, em 800 m de profundidade, nos meses de verão. O 
  $\mathcal{NR}$ adotado é igual a 1000 m. A região cinza próxima à margem continental denota 
  profundidades menores que o $\mathcal{NR}$ adotado. A envoltória vermelha delimita a área da 
 OEII. O círculo preto representa a posição da BiCSE.}
\end{figure}
 
É notável também em 20 m, que a CB, ao sul de sua origem junto a margem continental, desenvolve um grande 
anticiclone, de aproximadamente 500 km de diâmetro, centrado também em 15$^\circ$S. Especulamos que este anticiclone de
grande escala tenha sua for\-ma\-ção explicada pela complexa topografia associada aos bancos de Royal-Char\-lot\-te e 
Abrolhos (Figura \ref{fig:batimetria}). 

Em 200 m de profundidade (Figura \ref{fig:woa_verao200}), podemos constatar que a BiCSE está localizada consideravelmente mais ao sul, 
em aproximadamente 12$^\circ$S. Esta localidade já se encontra no interior dos domínios geográficos da
OEII. A organização das CCOs parece ocorrer de forma muito semelhante neste nível, quando comparamos 
com os 20 m. Assim como em 20 m, em 200 m observamos o mesmo anticiclone
centrado em  15$^\circ$S. Neste nível, a estrutura apenas exibe intensidades um pouco menores.
Constatamos ainda, que
 a banda de velocidades mais intensas da CSE já se encontra mais ao sul, assim como todo giro subtropical em si. 
Observamos nitidamente a organização da CB e alimentação da já SNB nesses níveis. Se focarmos a análise
para a região de abrangência da OEII, podemos esperar o sítio da BiCSE completamente amostrado. 

Passemos então à análise do campo de 500 m (Figura \ref{fig:woa_verao500}), que já se encontra
na região limítrofe entre o domínio vertical da ACAS e da AIA. 
Este campo nos mostra uma migração mais significativa da CSE para sul. 
Esta corrente, que nos níveis mais rasos descritos cruzava o oceano Atlântico Sul com notada componente meridional em 
direção ao Equador, 
não mais o faz. Em 500 m, o núcleo de velocidades da CSE parece se localizar em aproximadamente 21$^\circ$S,
sendo que todo seu fluxo, que é essencialmente zonal, está delimitado entre os paralelos de 18-25$^\circ$S.
A BiCSE neste nível se localiza 
em aproximadamente 20$^\circ$S, corroborando os padrões de \cite{stramma_england1999} e diferindo 
ligeiramente do cenário médio anual encontrado por \cite{rodrigues_etal2006} (21$^\circ$S). Neste nível, de acordo
com os padrões vistos na literatura (Capítulo \ref{cap:intro}), o que observamos em aproximadamente 20$^\circ$S é
o sítio de origem da SNB. Novamente, se focarmos a atenção na região da OEII, de acordo com estes resultados, 
podemos esperar apenas fluxo para norte junto ao contorno oeste nesta profundidade, associado ao escoamento desta
corrente.  

Em 800 m (Figura \ref{fig:woa_verao800}), ao focarmos atenção na envoltória vermelha
que delimita a região da OEII, podemos esperar apenas fluxo para norte junto ao contorno oeste novamente. 
O núcleo de velocidades
da CSE parece estar localizado em 24$^\circ$S e seu fluxo confinado entre os paralelos de 22$^\circ$S e 
26$^\circ$S. Neste nível não existe uma assinatura robusta da BiCSE próximo ao contorno oeste. Entretanto, 
 observamos que  a divergência do fluxo da CSE ocorre em aproximadamente 23$^\circ$S e 36$^\circ$W.  
Neste nível também ocorre uma concordância significativa entre
estes resultados e aqueles apresentados tanto por \cite{stramma_england1999} quanto por 
\cite{rodrigues_etal2006}. Ambos os trabalhos apontam para
uma BiCSE em aproximadamente 25$^\circ$S em 800 m de profundidade, ou seja, aproximadamente 2$^\circ$ mais ao sul apenas
do que os resultados aqui encontrados. 

\begin{table}
\caption{\label{tab:woa} \small Síntese dos resultados encontrados para a análise dos campos tridimensionais de 
$\psi$ climatológico de verão (WOA2001) no que se refere à localização e migração vertical da BiCSE.}
\begin{center}
% use packages: array
\begin{tabular}{lccc}
\hline 
 & & & \vspace{-0.4cm}\\
 {\bf Autor $\Rightarrow$} & {\bf Stramma \& England} & {\bf Rodrigues et al.} & {\bf Este Trabalho} \\ 
& {\bf (1999)} & {\bf (2006)} & \\
 {\large \ \ \  $\psi$ $\Rightarrow$} & Média Anual  & Média Anual & Média Verão \vspace{0.1cm}\\
\hline 
 & & & \vspace{-0.4cm}\\
BiCSE (0-100 m) & 15$^\circ$S & 14$^\circ$S & 10$^\circ$S \vspace{0.1cm} \\ 
BiCSE (200 m) & - & 18$^\circ$S & 12$^\circ$S \vspace{0.1cm} \\ 
BiCSE (500 m) & 20$^\circ$S & 21$^\circ$S & 20$^\circ$S \vspace{0.1cm} \\ 
BiCSE (800 m) & 25$^\circ$S & 25$^\circ$S & 23$^\circ$S \vspace{0.1cm} \\
Migração Meridional & 10$^\circ$ & 14$^\circ$ & 14$^\circ$ \vspace{-0.2cm}\\
\hspace{10mm} {\footnotesize (0-1000 m)} & & & \\
\hline
\end{tabular}
\end{center}
\end{table}

Com a finalidade de sintetizar os resultados encontrados e compará-los com aqueles encontrados por 
\cite{stramma_england1999} e \cite{rodrigues_etal2006}, construímos a Tabela \ref{tab:woa}.
A inspeção da referida tabela logo nos chama a atenção, pois o posicionamento da BiCSE encontrado 
neste trabalho para o verão climatológico está mais ao norte do que o encontrado por exemplo para 
\cite{stramma_england1999} e \cite{rodrigues_etal2006}, para uma climatologia anual. Esperávamos
o oposto, pois nos meses de verão para o hemisfério sul, todo o sistema de grande escala associado 
aos giros subtropicais se desloca para o sul, seguindo a Zona de Convergência Inter Tropical (ZCIT). 
Com os dados disponíveis para este trabalho, não podemos tirar conclusões a respeito de tal discrepância. 
Podemos apenas especular que as diferenças encontradas podem ser explicadas pela metodologia aplicada
para a identificação do ponto exato de ocorrência da BiCSE. \cite{stramma_england1999} utilizaram
abordagem por camadas, portanto a comparação com os resultados aqui encontrados deve ser conduzida
com cautela. Já \cite{rodrigues_etal2006} utilizaram níveis absolutos, porém estes autores não 
detalharam o método de identificação do ponto exato da feição. 

Os resultados encontrados para a climatologia de verão já nos dão indícios do que esperar dos
dados quase-sinóticos da OEII, que foram coletados nesta mesma estação do ano. Evidentemente, o cenário
a ser disponibilizado através dos campos construídos a partir da OEII, estará longe dos escoamentos
médios aqui estimados. Pretendemos obter  as feições encontradas em muito maior riqueza de detalhe, 
embasados no fato de que a resolução temporal e espacial deste conjunto hidrográfico é suficiente para
tal propósito. Buscaremos também, sempre que desejável, fazer intercomparações entre os cenários 
climatológico e sinótico.


\section{Cenário Sinótico} \label{sec:res_sinotico}

\hspace{6mm} Através da aplicação dos critérios e métodos descritos na Seção \ref{sec:psiref},
 dedicamos esta seção à apresentação e discussão
dos campos horizontais de função de corrente geostrófica absoluta. Estes campos, vale lembrar, são oriundos
de uma combinação entre dados termohalinos e de velocidade para a aplicação de uma metodologia que
julgamos inédita para os escoamentos ao largo da 
costa brasileira: o $\mathcal{MDR}$. Esta metodologia nos possibilita interpretar os campos de $\psi_{tot}$ como
um escoamento geostrófico absoluto, composto por suas componentes baroclínica e barotrópica. Ela oferece
uma grande vantagem em relação ao $\mathcal{MDC}$, pois não constrói campos relativos a uma superfície
isobárica arbitrária ($\mathcal{NR}$) e não se limita à componente baroclínica do escoamento. 

Uma vez feitos estes comentários, e dado que o objeto de estudo desta dissertação é investigar a estrutura tridimensional
da CB em seu sítio de origem e de outras CCOs que fluem ao longo da margem continental brasileira, concentraremos nossa 
análise novamente no oceano superior, ou seja, nos primeiros 1000 m da coluna de água aproximadamente. 

Entretanto, antes de iniciar as análises referentes aos campos de $\psi_{tot}$, acreditamos ser pertinente apresentar
os vetores de velocidade observada em 150 m oriundos de todo o pré-processamento descrito na Seção \ref{sec:ADCP}. Estes
vetores são mostrados então na Figura \ref{fig:adcpvet}, onde nos apoiaremos para indicar as principais feições
notáveis encontradas. É este campo que se submete à aplicação da $\mathcal{AOV}$, originando o campo de 
$\psi_{obs(p_0)}$, que por sua vez serve como referência para finalmente encontrar $\psi_{tot}$ em cada profundidade
desejada.

A primeira feição que nos chama bastante a atenção é o aspecto de divergência de escoamento junto à margem 
continental nos entornos de 14$^\circ$S.
 Tomando como referência este paralelo, 
há, na radial imediatamente ao norte, um evidente fluxo para norte, associado à SNB. Na radial imediatamente ao sul, 
em contraponto, existe um 

\begin{figure}%[hb]
 \begin{center}
  \includegraphics[width=13cm,keepaspectratio=true]{../proc/adcp/figuras/adcp_150m.pdf}
 \end{center}
 \vspace{-.25cm}
 \renewcommand{\baselinestretch}{1}
 \caption{\label{fig:adcpvet} \small Vetores de velocidade observada via ADCP de casco em 150 m de profundidade durante a OEII. 
Estes vetores consistem nas velocidades após a rotina de pré-processamento descrita no Capítulo \ref{cap:dados}.}
\end{figure} 

fluxo para sul. Esta é uma provável evidência da BiCSE em caráter regional,
 que, mesmo analisando apenas os vetores
brutos, sem qualquer tipo de interpolação, nos faz acreditar que esta feição se localiza na referida latitude para o nível de 150 m. 
Oportunamente, o conjunto OEII dispõe de dois trechos de radiais compostas unicamente por perfilagens de ADCP, 
que são quase perfeitamente meridionais ligando estas duas radiais hidrográficas citadas. Estes
trechos mostram claramente a inversão de velocidades, ora para norte (parte norte do trecho), 
ora para sul (parte sul do trecho), 
associada ao sinal da BiCSE junto à margem continental.

Migrando a atenção para a área ao norte do paralelo de 14$^\circ$S, fica
clara também a organização da SNB junto ao contorno oeste, à medida que existem vetores de velocidade junto à margem continental
consideravelmente maiores do que nas porções mais oceânicas das radiais. 
Já que estamos analisando um nível vertical muito próximo do núcleo da
SNB (200 m), de acordo com os padrões revistos no Capítulo \ref{cap:intro}, aproveitemos para comentar seu escoamento. A SNB parece
estar razoavelmente bem representada nas quatro primeiras radiais da fronteira norte da grade amostral. Seu fluxo neste nível parece se intensificar 
a medida que se dirige para o norte. Na primeira radial do desenho amostral, esta corrente exibe velocidades maiores que 
0,5 m s$^{-1}$. Inferimos como largura média desta corrente pouco mais de 100 km neste nível de 150 m.

Já que identificamos a latitude de 14$^\circ$S como a possível localização da BiCSE regionalmente,
 de acordo com os resultados explicitados 
na Figura \ref{fig:adcpvet} apontamos esta latitude como o sítio de origem da CB, em 150 m. A CB não parece se 
organizar tão bem quanto a SNB. A topografia complexa e suas fracas velocidades podem estar mascarando a 
identificação de seu escoamento. Aguardemos a descrição dos campos de $\psi_{tot}$ para inferir 
informações de forma mais detalhada. 

Outras duas feições que identificamos claramente e esperamos encontrar nos campos de $\psi_{tot}$, são duas estruturas vorticais
anticiclônicas, uma aparentemente centrada em  15$^\circ$S - 37,5$^\circ$W e outra em 17$^\circ$S - 38$^\circ$W. Isto pelo menos
é o que sugere o comportamento dos vetores nas radiais que circundam estas localidades. Sem mais para ser explorado na 
Figura \ref{fig:adcpvet}, prosseguimos finalmente com a análise dos campos de $\psi_{tot}$ nos níveis 
verticais de interesse deste trabalho. Selecionamos os níveis verticais de 20, 150, 200, 500, 800 m, 
cujos mapas estão representados nas Figuras \ref{fig:psi_oeii20} - \ref{fig:psi_oeii800}.
 

Primeiramente, voltemos nossa atenção à porção mais rasa da coluna de água, exibida na Figura \ref{fig:psi_oeii20}, 
correspondente aos 20 m de profundidade. Recapitulando o que foi encontrado na Seção \ref{sec:res_climatologico}, climatologicamente, 
a BiCSE não se encontra nos domínios geográficos da OEII para este nível.
Tal fato se confirma sinoticamente, uma vez que não observamos na referida profundidade
nenhum sinal da SNB. O que observamos é uma CB fraca e meandrante, em todo o domínio.
Seu escoamento parece adentrar aproximadamente em 11$^\circ$S - 34,5$^\circ$W dirigindo-se para o sul e logo começa a 
meandrar vigorosamente para oeste, formando um ciclone.
Suas velocidades nesta região são da ordem de 0,25 m s$^{-1}$. 
A corrente meandra anticiclonicamente e volta a se orientar em direção ao sul aproximadamente centrada
 no meridiano de 36$^\circ$W, ainda distante da margem continental. Suas velocidades nesta região 
ainda se mantêm na ordem de 0,25 m s$^{-1}$.

A CB flui para sul até cerca de 13$^\circ$S e meandra novamente para oeste e em aproximadamente 14$^\circ$S, 
finalmente atinge o contorno oeste e 
passa a fluir sobre a margem continental. Suas velocidades crescem e passam, em 13$^\circ$S, a ser da ordem de 0,35 m s$^{-1}$.
Assim esta corrente flui, junto a margem continental, 
ganhando velocidade em sua viagem para o sul, culminando em 0,5 m s$^{-1}$ ao deixar o 
domínio da OEII em aproximadamente 19,5$^\circ$S. 

A estrutura exibida pela CB ao norte de 14$^\circ$S
parece indicar que esta corrente ainda está se organizando como uma CCO. O aspecto meandrante
exibido, parece ser a forma com a qual a CB se integra ao contorno oeste do ponto de vista 
sinótico. O fato de seu escoamento ter início distante da margem continental pode indicar que
a estrutura da BiCSE e sua atividade de meso-escala ocorram já ao largo, fora do contorno
oeste, pelo menos neste nível. 

\begin{figure}%[ht]
 \begin{center}
  \includegraphics[width=14cm,keepaspectratio=true]{../proc/ref_adcp/figuras/psi_ref_adcp_OEII_20m.pdf}
 \end{center}
 \vspace{-.25cm}
 \renewcommand{\baselinestretch}{1}
 \caption{\label{fig:psi_oeii20} \small Campo horizontal de função de corrente geostrófica absoluta $\psi_{tot}$ a 20 m de profundidade, estimado a partir dos dados da OEII. 
 A máscara cinza, junto à costa, representa o contorno dinâmico (isóbata de 100 m). Os vetores de velocidade
 estão sobrepostos ao campo de $\psi_{tot}$.}
\end{figure}

\begin{figure}%[ht]
 \begin{center}
  \includegraphics[width=14cm,keepaspectratio=true]{../proc/ref_adcp/figuras/psi_ref_adcp_OEII_150m.pdf}
 \end{center}
 \vspace{-.25cm}
 \renewcommand{\baselinestretch}{1}
 \caption{\label{fig:psi_oeii150} \small Campo horizontal de função de corrente geostrófica absoluta $\psi_{tot}$ a 150 m de profundidade, estimado a partir dos dados da OEII. 
 A máscara cinza, junto à costa, representa o contorno dinâmico (isóbata de 150 m). Os vetores de velocidade
 estão sobrepostos ao campo de $\psi_{tot}$.}
\end{figure}

\begin{figure}%[ht]
 \begin{center}
  \includegraphics[width=14cm,keepaspectratio=true]{../proc/ref_adcp/figuras/psi_ref_adcp_OEII_200m.pdf}
 \end{center}
 \vspace{-.25cm}
 \renewcommand{\baselinestretch}{1}
 \caption{\label{fig:psi_oeii200} \small Campo horizontal de função de corrente geostrófica absoluta $\psi_{tot}$ a 200 m de profundidade, estimado a partir dos dados da OEII. 
 A máscara cinza, junto à costa, representa o contorno dinâmico (isóbata de 200 m). Os vetores de velocidade
 estão sobrepostos ao campo de $\psi_{tot}$.}
\end{figure}

\begin{figure}%[ht]
 \begin{center}
  \includegraphics[width=14cm,keepaspectratio=true]{../proc/ref_adcp/figuras/psi_ref_adcp_OEII_500m.pdf}
 \end{center}
 \vspace{-.25cm}
 \renewcommand{\baselinestretch}{1}
 \caption{\label{fig:psi_oeii500} \small Campo horizontal de função de corrente geostrófica absoluta $\psi_{tot}$ a 500 m de profundidade, estimado a partir dos dados da OEII. 
 A máscara cinza, junto à costa, representa o contorno dinâmico (isóbata de 500 m). Os vetores de velocidade
 estão sobrepostos ao campo de $\psi_{tot}$.}
\end{figure}

\begin{figure}%[ht]
 \begin{center}
  \includegraphics[width=14cm,keepaspectratio=true]{../proc/ref_adcp/figuras/psi_ref_adcp_OEII_800m.pdf}
 \end{center}
 \vspace{-.25cm}
 \renewcommand{\baselinestretch}{1}
 \caption{\label{fig:psi_oeii800} \small Campo horizontal de função de corrente geostrófica absoluta $\psi_{tot}$ a 800 m de profundidade, estimado a partir dos dados da OEII. 
 A máscara cinza, junto à costa, representa o contorno dinâmico (isóbata de 800 m). Os vetores de velocidade
 estão sobrepostos ao campo de $\psi_{tot}$.}
\end{figure}

Ao sul de 14$^\circ$S, onde a CB já está completamente integrada ao contorno oeste, a corrente
já exibe velocidades, larguras e escoamento típico de uma CCO. A CB parece, ao sul de 14$^\circ$S,
seguir os contornos topográficos da margem continental, como que tentando conservar
 vorticidade potencial. O Banco de Royal-Charlotte e o Banco de Abrolhos
parecem forçar um intenso meandramento no escoamento da CB, dando origem a três estruturas
anticiclônicas: uma logo ao sul do paralelo de 14$^\circ$S, ao largo de Ilhéus, outra entre
o Banco de Royal-Charlotte e o Banco de Abrolhos e outra ao largo do Banco de Abrolhos. 
Voltemos agora nossa atenção então para os meandros e vórtices da CB. 

 A partir deste momento, definimos nomenclaturas para estas estruturas, pois
as mesmas se estendem até outros níveis. A estrutura anticiclônica localizada ao norte do 
Banco Royal-Charlotte denominaremos {\bf Vórtice de Ilhéus (VI)}. Nos referiremos ao meandro
localizado entre o Banco Royal-Charlotte e o Banco de Abrolhos como {\bf Vórtice de Royal-Charlotte
(VRC)}.
 Ao anticiclone de maior dimensão horizontal, localizado ao largo do Banco de Abrolhos, 
seguiremos também a nomenclatura primeiramente sugerida por \cite{silveira_etal2006B}, ou seja, 
chamaremos de {\bf Vórtice de Abrolhos (VAb)}. 

O VI nestes níveis parece ter a estrutura de
velocidades mais bem definida que os demais. Ele possui uma forma bastante simétrica e tem, como todo vórtice, 
velocidades menores em seu centro. Seu diâmetro é de aproximadamente 200 km, característico de vórtices
de meso-escala. Ao longo de sua estrutura, este vórtice exibe velocidades da ordem de 0,5 m s$^{-1}$.

O VRC tem intensidade semelhante ao VI, porém sua simetria não é tão marcada.
Este vórtice se alonga em sua porção leste, deformando-se e dificultando qualquer
tentativa de estimativa de seu diâmetro nestes níveis. 

Propositalmente, deixamos por último o VAb,
que é o maior deles e já foi 
anteriormente observado do ponto de vista baroclínico, através da aplicação do $\mathcal{MDC}$ 
por \cite{silveira_etal2006B}, e cujos esforços já foram relatados no Capítulo \ref{cap:intro}.
Vale frisar que os dados explorados por estes são também no período de verão. 
 Os resultados destes autores são confrontados com aqueles aqui 
obtidos com o auxílio da Figura \ref{fig:VAb}. 

Nesta figura, selecionamos 
uma sub-região inserida na presente área de estudo, representativa da região investigada pelos autores, 
no nível de 20 m.
Conforme a figura nos mostra, o VAb trata-se 
de um vórtice de diâmetro visivelmente maior do que os demais encontrados aqui, ou mesmo os que ocorrem
ao largo do sudeste brasileiro, muitas vezes descritos na literatura. Apesar de aparentemente
o VAb não ter sido completamente capturado pela OEII, sua localização geográfica é 
extremamente semelhante com a da estrutura encontrada nos esforços de \cite{silveira_etal2006B}.
A menos das intensidades envolvidas e da exata simetria do vórtice, o padrão de escoamento é bem similar
 entre os campos deste e do trabalho supracitado. A maior diferença observada talvez esteja associada à extensão 
zonal desta estrutura, que parece ser significativamente maior para o caso deste trabalho. 
Enquanto o campo obtido por \cite{silveira_etal2006B} mostra uma extensão zonal 

\begin{figure}%[ht]
 \begin{center}
  \includegraphics[width=13cm,keepaspectratio=true]{figuras/psi_abrolhos2_20m.pdf}
  \includegraphics[width=13cm,keepaspectratio=true]{figuras/comp_OEII_abrolhos2_20m.pdf}
 \end{center}
 \vspace{-.25cm}
 \renewcommand{\baselinestretch}{1}
 \caption{\label{fig:VAb} \small Campos horizontais de $\psi$ em 20 m de profundidade.
Painel superior: $\psi$ calculado através do $\mathcal{MDC}$, relativo a 1000 dbar, 
segundo os resultados de \cite{silveira_etal2006B} para o Cruzeiro Abrolhos 2 (verão de 2005).
Painel inferior: $\psi$ calculado através do $\mathcal{MDR}$, oriundo dos dados da OEII (verão de 2005).}
\end{figure}

\hspace{-7mm} de aproximadamente 220 km, os resultados encontrados aqui nos remetem a uma extensão de aproximadamente 280 km. 
A metodologia empregada neste trabalho, baseada em velocidades observadas, nos permite
confirmar a existência do VAb, pois em nossos cálculos não nos furtamos de considerar
a componente barotrópica do escoamento. Com isso, complementamos os esforços destes autores e 
confirmamos que seus resultados, contendo apenas a componente baroclínica, são uma representação 
razoável do campo total. A assinatura do VAb na análise dos dados da OEII corrobora
\cite{silveira_etal2006B} acerca desta feição ser uma estrutura quase-permanente.

Passemos agora ao nível de 150 m (Figura \ref{fig:psi_oeii150}).
Lembremos que este campo, uma vez que usamos 150 m como referência 
para os cálculos segundo o $\mathcal{MDR}$, se trata exatamente das velocidades oriundas do ADCP de casco.
Assim, este é o campo de função de corrente observada oriundo da interpolação dos vetores de velocidade 
expostos na Figura \ref{fig:adcpvet}. 
Neste nível já é nítida a ocorrência da BiCSE em aproximadamente 14$^\circ$S. Há um
escoamento zonal para oeste adentrando o domínio em 14,5$^\circ$S, provavelmente associado a CSE, que 
se bifurca nitidamente ao atingir a margem continental brasileira. A CSE penetra zonalmente o domínio como um jato
de aproximadamente 100 km de largura, exibindo velocidades da ordem de 0,15 m s$^{-1}$. Esta bifurcação, naturalmente
dá origem a um fluxo para sul (CB) e outro para norte (SNB), que tendem a seguir o contorno oeste
com destinos opostos. Apesar de no campo de 20 m não termos observado a BiCSE, o padrão
exposto nos sugere que a mesma ocorre 
ao largo da costa. Já para os campos de 150 m fica claro que
ela ocorre junto à margem continental. 

Por este nível (150 m) não ser ainda o correspondente ao 
núcleo da SNB, de acordo com a revisão da literatura exposta no Capítulo \ref{cap:intro}, 
esta corrente ainda exibe baixas velocidades. Logo que a SNB se organiza junto ao contorno oeste
exibe velocidades da ordem de 0,2 m s$^{-1}$. Sua velocidade aumenta conforme se dirige para o 
norte, deixando o domínio com a velocidade máxima de 0,25 m s$^{-1}$. Seu escoamento é bem 
organizado em forma de jato, com largura média de aproximadamente 100 km. A corrente segue
fielmente a margem continental, sem exibir qualquer tipo de meandramento, fato este favorecido
pela topografia nessa região, que é extremamente diferente daquela ao sul de 15$^\circ$S (Figura \ref{fig:batimetria}).

A CB tem em sua origem, aproximadamente 0,25 m s$^{-1}$ e assim se mantém rumo ao sul até 
deixar o domínio em 19,5$^\circ$S. Ressaltamos aqui a menor intensidade do escoamento
da CB de forma geral, quando comparamos com o campo de 20 m. Isto nos mostra o quão
esta corrente é rasa perto de seu sítio de origem, e que seu núcleo de velocidades máximas se encontra
em superfície. Sua largura média neste nível corresponde a aproximadamente 100 km, assim como a SNB. 

No nível de 150 m novamente observamos um vigoroso meandramento associado ao escoamento da CB. As estruturas
vorticais descritas para 20 m são detectadas ainda neste nível. A estrutura do VI permanece praticamente 
inalterada. O VRC parece ter perdido um pouco de intensidade, uma vez que exibe velocidades menores (0,17 m s$^{-1}$)
nesta profundidade. O VAb, também parece apresentar uma estrutura mais tênue do que
no nível de 20 m, assim como o VRC e a CB propriamente dita. 

Ao descrevermos a porção da coluna de água, equivalente aos 200 m de profundidade (Figura \ref{fig:psi_oeii200}), aparentemente
não existem grandes mudanças no padrão de escoamento. Entretanto, há alguns comentários 
interessantes que se fazem pertinentes no tocante a este nível. 

Conforme explicitado no Capítulo \ref{cap:intro}, este é exatamente o nível onde a SNB tem seu 
núcleo, de acordo com trabalhos anteriores presentes na literatura. É exatamente nesse escopo
que fazemos a primeira ressalva em relação a este campo. É evidente que a SNB se mostra mais 
intensa e bem organizada em 200 m. Seu escoamento se inicia nos mesmos 14,5$^\circ$S, e
segue sua viagem para o norte junto à margem continental. Neste nível, a SNB atinge velocidades
da ordem de 0,45 m s$^{-1}$, em 10,5$^\circ$S, onde se encontra plenamente formada.
Este valor é comparável ao encontrado por \cite{silveira_etal1994} (0,5 m s$^{-1}$),
 \cite{stramma_etal1995} (0,5 m s$^{-1}$), e \cite{soutelino2005} (0,34 m s$^{-1}$). 

A segunda observação interessante se refere à estrutura dos vórtices VI e VRC. Estes parecem 
ter suas estruturas de velocidade mais bem organizadas, ou seja, apresentam uma configuração mais simétrica
do que aquela observada nos níveis mais rasos. As intensidades de corrente
associadas ao VI não diferem de níveis mais rasos, sugerindo que sua estrutura 
atinge níveis ainda mais profundos.  

O mesmo não pode ser afirmado em relação ao VAb. No campo de 200 m, este vórtice parece se 
 esvaecer. Sua estrutura, mais fraca em termos de velocidade e dimensionalmente
menor do que aquela descrita nos níveis mais superiores, está agora limitada à porção 
sul do Banco de Abrolhos.  

Intrigantemente o mesmo foi observado nos campos de \cite{silveira_etal2006B}, apesar de tal 
fato não ter sido explorado pelos autores. Para tornar clara esta constatação, repetimos
para 200 m a construção de uma figura que exponha o campo de $\psi_{tot}$ oriundo da OEII limitado 
na área geográfica correspondente a grade hidrográfica explorada por estes autores. Este campo, 
juntamente com o campo original de  \cite{silveira_etal2006B} em 200 m são mostrados na Figura
\ref{fig:VAb2}. Exceto pelas velocidades para sul associadas às bordas oeste dos vórtices VI, VRC e VAb, a CB parece
não mais fazer parte do cenário sinótico em 200 m de profundidade. Aparentemente, o aporte zonal 
de velocidades para oeste observado em 19$^\circ$S - 35$^\circ$W já é a assinatura da CSE atingindo 
o contorno oeste neste nível. Isto pode indicar o sítio de formação da SNB, porém não façamos 
mais observações antes de inspecionar o campo de 500 m. 

Ao passarmos a análise ao campo de 500 m, lembramos que 
de acordo com padrões de larga escala observados no Capítulo \ref{cap:intro}, 
a CSE adentra a margem continental brasileira em aproximadamente 20$^\circ$S, onde a Cadeia Vitória-Trindade
 funciona como um divisor de águas para o seu fluxo. Os resultados encontrados aqui para o 
cenário climatológico de verão (Figura \ref{fig:woa_verao500}) confirmam tal fato.
De acordo com estes resultados, descritos na Seção
\ref{sec:res_climatologico}, não há mais evidências da CB neste nível.

De acordo com a Figura \ref{fig:psi_oeii500}, parece também não haver mais sinais da presença da CB sinoticamente, 
mostrando novamente o quão rasa esta corrente é ao largo da margem continental leste brasileira. Os padrões da 
literatura que afirmam que a BiCSE na região da picnoclina ocorre em 20$^\circ$S 
\citep{stramma_england1999,rodrigues_etal2006} parecem ser corroborados pelos resultados 
aqui obtidos.  

O que vemos em 500 m é provavelmente a organização da SNB como CCO ao largo da costa leste brasileira. Este
indício é sustentado pela presença de um fluxo zonal adentrando o domínio no extremo sul e leste
da grade amostral da OEII (já observado em 200 m). Este fluxo parece ser a chegada da CSE nestes níveis, que, meandrando, 
se alinha ao contorno oeste junto ao Banco de Abrolhos e finalmente se constitui como uma CCO
(SNB), que segue para o norte exibindo intensa atividade de meso-escala. O fluxo da CSE tem aproximadamente
150 km de largura e exibe velocidades da ordem

\begin{figure}%[hb]
 \begin{center}
  \includegraphics[width=13cm,keepaspectratio=true]{figuras/psi_abrolhos2_200m.pdf}
  \includegraphics[width=13cm,keepaspectratio=true]{figuras/comp_OEII_abrolhos2_200m.pdf}
 \end{center}
 \vspace{-.25cm}
 \renewcommand{\baselinestretch}{1}
 \caption{\label{fig:VAb2} \small Campos horizontais de $\psi$ em 200 m de profundidade.
Painel superior: $\psi$ calculado através do $\mathcal{MDC}$, relativo a 1000 dbar, 
segundo os resultados de \cite{silveira_etal2006B} para o Cruzeiro Abrolhos 2 (verão de 2005).
Painel inferior: $\psi$ calculado através do $\mathcal{MDR}$, oriundo dos dados da OEII (verão de 2005).}
\end{figure}

\hspace{-7mm} de 0,2 m s$^{-1}$ no momento em que adentra o domínio da 
grade amostral da OEII. 

A SNB em 500 m, origina-se então ao sul de 19$^\circ$S, exibindo velocidades da ordem de 0,2 m s$^{-1}$, 
com uma largura de aproximadamente 80 km. Esta corrente segue sua viagem para norte meandrando, ora junto 
a margem continental, ora mais ao largo. Este é o caso do paralelo de 17,5$^\circ$S, onde logo ao norte do 
Banco de Abrolhos a corrente deixa de seguir a isóbata de 500 m, e se afasta para o largo, fluindo ao longo do 
meridiano de 36,5$^\circ$W. Nesta latitude, seu fluxo é ligeiramente mais intenso, com velocidades da 
ordem de 0,25 m s$^{-1}$. Ao atingir o paralelo de 15$^\circ$S, seu fluxo se torna consideravelmente mais
intenso, atingindo valores de velocidade da ordem de 0,5 m s$^{-1}$. Nesta latitude, a SNB integra a borda
leste do VI, o qual ainda permanece vigoroso neste nível. Imediatamente ao norte de 14$^\circ$S, seu fluxo parece
se bifurcar, onde parte da corrente se dirige ao contorno oeste e parte segue para norte ao longo do meridiano 
de 37$^\circ$W. O ramo que se aproxima da margem continental flui com velocidades da ordem de 0,15 m s$^{-1}$, enquanto 
o que flui ao largo, exibe velocidades de 0,25 m s$^{-1}$. Em torno de 12$^\circ$S estes ramos voltam a se unificar,
reorganizando a SNB e fazendo com que assim ela escoe até deixar o domínio, com velocidades de aproximadamente
0,35 m s$^{-1}$. 

Constatamos também para o campo de 500 m a assinatura do VI. Este vórtice parece
ter a mesma estrutura de velocidades desde o campo de 150 m até os 500 m. Sua estrutura apenas difere sutilmente
daquela observada no campo de 20 m. Neste nível, as velocidades associadas a ele são menores.  
Observamos ainda que o centro do VRC parece ter se deslocado para leste. 
Em seu lugar, junto a margem continental notamos em 500 m 
uma outra estrutura vortical, girando no sentido oposto, ou seja, ciclônica. Este ciclone
parece estar confinado entre o banco Royal-Charlotte e o Banco de Abrolhos. Seu diâmetro é menor
do que o das demais estruturas vorticais observadas em todos os campos, atingindo apenas 100 km. O VRC 
e este ciclone parecem formar uma estrutura bipolar de velocidades. O VAb esvaeceu por completo neste nível,
dando lugar a assinatura da chegada da CSE e estrutura de origem da SNB. 
 
Passando ao campo mais profundo escolhido para análise nesta seção, comentemos as principais
constatações acerca do campo de 800 m (Figura \ref{fig:psi_oeii800}). Novamente não observamos mudanças importantes
nos padrões de escoamento quando comparamos com o nível de 500 m. Em geral as velocidades se mostraram 
menores que no nível citado. Destacamos que em 800 m o VRC parece ter praticamente desaparecido, 
diferentemente do VI, que permanece ainda bem estruturado e fazendo parte da estrutura bipolar antes descrita. 
O fluxo da SNB não mais aparenta ter a robustez apresentada em níveis mais rasos, provavelmente indicando que 
esta corrente já está enfraquecendo com o aumento da profundidade. 

