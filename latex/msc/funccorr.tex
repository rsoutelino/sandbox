\section{Preâmbulo}\label{sec:pre_funccorr}

\hspace{6mm} Após a apresentação dos dois conjuntos (climatológico e sinótico) 
de dados que utilizamos, é natural que apresentemos a metodologia empregada para a 
construção dos campos de velocidade horizontalmente não-divergente 
citados no capítulo anterior. Cada conjunto, 
em particular, possibilitará a obtenção de diferentes campos. 

Antes de discutirmos tais campos, abordaremos a fundamentação teórica na qual está
calcada a presente dissertação. Através dessa fundamentação teórica e da metodologia 
aplicada, conseguiremos construir mapas horizontais de velocidade a partir dos 
conjuntos de dados citados. 

O tipo de escoamento que pretendemos investigar tem escalas temporais e espaciais
que se aproximam da geostrofia. O movimento geostrófico é um movimento 
estacionário não-divergente horizontalmente (se avaliado no plano $f$), que consiste no resultado de 
um equilíbrio entre as forças de gradiente
de pressão e Coriolis. As duas modalidades conhecidas da força do gradiente de 
pressão são a força do gradiente barotrópico de pressão e a força do gradiente baroclínico de
pressão. A primeira depende unicamente da elevação da superfície livre
do mar em relação a um nível base. Ela é responsável pela geração das correntes
geostróficas barotrópicas, que não apresentam variações verticais.
A segunda depende dos gradientes horizontais de densidade, 
ou seja, dos gradientes horizontais das propriedades termohalinas. Esta é responsável
pela geração de correntes geostróficas baroclínicas, que tem por definição o cisalhamento
vertical de velocidades e transporte nulo na coluna de água. 

As correntes geostróficas do oceano real, como as CCOs, tem na realidade a con\-tri\-bu\-i\-ção
dos dois tipos de força do gradiente de pressão. É comum então dividirmos as correntes
geostróficas em duas {\bf componentes}: a {\bf componente barotrópica} e a {\bf componente baroclínica}.
De uma forma geral as correntes oceânicas como as CCOs em um oceano tropical, têm em sua 
componente baroclínica a maior parte da energia, em parte devido a forte estratificação
da coluna de água observada nestas latitudes. É devido a este motivo que, como pudemos
perceber através do Capítulo \ref{cap:intro}, numerosos são os trabalhos que descrevem os escoamentos, 
quantitativamente ou qualitativamente, com base apenas na componente baroclínica dos mesmos. 
Por mais que esta componente seja predominante e ajude a responder e entender as lacunas
ainda existentes no conhecimento do escoamento ao largo da margem continental brasileira, 
a disponibilidade de traçar estimativas da velocidade total torna mais completo o entendimento
da circulação. 

Em geral, obter estimativas da velocidade total é consideravelmente mais dispendioso do que 
apenas a da componente baroclínica. O motivo é que é muito mais difícil, por exemplo, obter
dados de velocidade ou altura da superfície livre do mar, do que da estrutura termohalina
tridimensional. 

Neste trabalho, temos a disponibilidade dos dois tipos de variáveis: dados termohalinos
(climatológicos e sinóticos) e dados de velocidade (sinóticos). Os dados termohalinos, 
sejam eles climatológicos ou sinóticos, nos possibilitam a estimativa de velocidade 
geostrófica baroclínica. Os dados de velocidade, possibilitam a estimativa da velocidade
geostrófica total. Na próxima
seção, detalharemos o embasamento teórico que permite, através de dados termohalinos, 
estimar velocidades geostróficas baroclínicas relativas, e sob que ótica isto permite
a construção de campos horizontais desta grandeza.    

\section{Função de Corrente Geostrófica Relativa} \label{sec:psigeo}

\subsection{Fundamentos Teóricos} \label{sec:psigeo_teo}

\hspace{6mm} O movimento geostrófico, por ter aproximadamente natureza 
horizontalmente não-divergente ($\nabla_H . \vec{v} = 0$), nos permite definir uma grandeza 
denominada {\bf função de corrente ($\psi$)}, que tem como
definição matemática

\begin{equation}
u = -\frac{\partial \psi}{\partial y} \ \ \ \ \ \ \ \ \  e  \ \ \ \ \ \ \ \ \  v = \frac{\partial \psi}{\partial x}.
\vspace{0.5cm}
\label{eq:uvpsi}
\end{equation}

Se partimos das componentes zonal e meridional da equação do movimento geostrófico,
em coordenadas isobáricas, temos que 

\begin{equation}
u = -\frac{1}{f_0} \frac{\partial \Delta \Phi}{\partial y} \ \ \ \ \ \ \ \ \  e  \ \ \ \ \ \ \ \ \  v = \frac{1}{f_0} \frac{\partial \Delta \Phi}{\partial x},
\vspace{0.5cm}
\label{eq:uvgpan}
\end{equation}
 
onde $f_0$ é o parâmetro de Coriolis médio para a região de estudo. $\Delta \Phi$ é anomalia do geopotencial, definida por 

\begin{equation}
\Delta \Phi \: = \: \int_{p_0}^{p} \delta_\alpha dp,
\vspace{0.5cm}
\label{eq:gpan}
\end{equation}

que por sua vez depende unicamente dos dados termohalinos, pois 
$\delta_\alpha$ consiste na anomalia do volume específico.  Portanto, vemos que para cada perfil hidrográfico,
seja ele oriundo dos dados termohalinos climatológicos provenientes da base WOA2001, seja ele oriundo das
estações hidrográficas da OEII, podemos calcular perfis de $\Delta \Phi$ relativos a um nível 
$p_0$ de referência, que daqui em diante chamaremos de $\mathcal{NR}$. 
Combinando as Equações \ref{eq:uvpsi} e \ref{eq:uvgpan}, chegamos facilmente a 

\begin{equation}
\psi_{(p_0/p)} \: = \: \frac{\Delta \Phi}{f_0},
\vspace{0.5cm}
\label{eq:psigpan}
\end{equation}

de onde podemos calcular as componentes zonal e meridional do movimento geostrófico, através da Equação 
\ref{eq:uvpsi}. Entretanto, calcular o gradiente explicitado na referida equação requer um mapeamento
horizontal da quantidade $\psi_{(p_0/p)}$. 

Das Equações \ref{eq:psigpan} e \ref{eq:gpan}, podemos apontar algumas limitações inerentes ao método quando da obtenção 
de $\psi_{(p_0/p)}$ através de dados hidrográficos apenas.
Uma vez que a determinação de $\psi_{(p_0/p)}$ passa por adotarmos um valor representativo do $\mathcal{NR}$, 
este deve ser tal que melhor reproduza o campo de velocidades baroclínicas absolutas 
ao largo da costa leste brasileira. Em outras palavras, está associado às velocidades geostróficas 
baroclínicas relativas (ao $\mathcal{NR}$) e não absolutas. A forma mais comum de lidar com o $\mathcal{NR}$
é assumir um valor nulo e constante para o mesmo, em toda a área de estudo. Esta abordagem que permite a obtenção
de velocidades geostróficas através dos dados termohalinos é denominada Método Dinâmico Clássico ($\mathcal{MDC}$),
e foi desenvolvido originalmente por \cite{sandstrom_helland1903}.

As técnicas utilizadas para o mapeamento de $\psi_{(p_0/p)}$, e conseq\"uentemente $u_{(p_0/p)}$ e $v_{(p_0/p)}$ 
geostróficas serão detalhadas na próxima seção.
Entretanto, já adiantamos aqui que para as diferentes abordagens previstas nesta dissertação
(climatológica e sinótica), técnicas diferentes serão utilizadas para este mapeamento, em 
função dos tipos de variáveis observacionais disponíveis para cada uma delas. Para o cenário 
climatológico, por dispormos apenas de dados termohalinos, utilizaremos exatamente a abordagem 
apresentada nesta seção. Para o cenário sinótico, utilizaremos uma com\-bi\-na\-ção de dados termohalinos
e de velocidade para nos isentarmos da dependência de um $\mathcal{NR}$. 

\subsection{Mapeamento Objetivo} \label{sec:psigeo_AO}

\hspace{6mm} Fazendo uso dos dados termohalinos climatológicos disponíveis e usufruindo do método analítico 
detalhado na seção anterior, comecemos a descrição da metodologia empregada
na construção dos campos de $\psi_{(p_0/p)}$ geostrófico climatológico para o verão austral.

Inicialmente, como buscamos a representação de um cenário 
médio que possa nos prover uma descrição adequada da feição da BiCSE, adotaremos o $\mathcal{NR}$ de 1000 dbar, 
já utilizado por \cite{rodrigues_etal2006}. De acordo com estes autores este é o nível médio em que o escoamento associado ao 
giro subtropical se esvanece. 

Na construção dos campos horizontais de $\psi_{(p_0/p)}$, primeiramente calculamos um perfil de $\psi_{(p_0/p)}$ a partir de 
cada perfil termohalino da base de dados WOA2001, limitados ao Oceano Atlântico Sul, e usando 
a Equação \ref{eq:psigpan}. Posteriormente, interpolamos os valores de $\psi_{(p_0/p)}$ encontrados para uma 
grade limitada pelos paralelos de 2$^\circ$-30$^\circ$S e pelos meridianos de 30$^\circ$-50$^\circ$W, em cada
nível vertical. Para tanto
utilizamos uma técnica de interpolação chamada 
\textit{Análise Objetiva} ($\mathcal{AO}$). De acordo com 
\cite{bretherton_etal1976}, a $\mathcal{AO}$ é baseada no Teorema 
de Gauss-Markov. Este teorema fornece uma expressão para a estimativa de erros mínimos 
quadráticos de variáveis físicas (temperatura, salinidade, velocidade, função de corrente, 
vorticidade, etc), sendo a estatística do campo estimada na forma de espectro espaço-temporal. 
Mapas de erros são obtidos através de estimativas do erro médio quadrático de interpolação. 

\cite{carter_robinson1987} mostraram que a $\mathcal{AO}$ de dados oceânicos pode ser pensada 
como um ajuste por mínimos quadrados onde as funções-peso dependem da correlação entre os dados. 
Logo, sua implementação requer um conhecimento prévio da função de correlação das 
variáveis, $C$, e da variância do erro amostral aleatório, $\epsilon^2$. Seguindo 
\cite{emery_thomson1998}, embora a especificação da matriz de correlação deva ser determinada na 
estrutura observada das variáveis do oceano, há a opção de se utilizar uma forma matemática teórica 
ajustada e baseada nesta estrutura. É o caso sugerido quando os dados amostrados são esparsos, 
como acontece com o conjunto WOA2001 e a OEII (já adiantando que o mesmo método de 
interpolação será adotado para este conjunto).

Para o mapeamento horizontal de escoamentos geofísicos, buscamos utilizar uma função 
de correlação dada por:

\begin{equation}
C(\Delta x,\Delta y) = (1-\epsilon^{2})e^{-\left [ (\Delta x)^2/l_x^2 + (\Delta y)^2/l_y^2 \right ]},
\vspace{0.25cm}
\label{func_cor_1}
\end{equation}
onde $\Delta x$ e $\Delta y$ representam os incrementos espaciais horizontais nas 
direções zonal e meridional, respectivamente, $l_x$ e $l_y$ são os chamados comprimentos de 
correlação e $\epsilon^{2}$ é a variância do erro amostral aleatório \citep{carter_robinson1987}.

Para essa natureza de escoamentos, as diferenças entre os comprimentos de correla\-ção 
horizontais $l_x$ e $l_y$ costumam ser sutis como conseq\"uência da notável isotropia horizontal dos 
mesmos. Em muitos casos e principalmente naqueles envolvendo mapeamento de quantidades 
com estruturas de caráter radial, tal como característico das feições de larga escala, como a BiCSE,
e as feições ligadas à atividade de meso-escala da CB (meandros e vórtices),
é convidativo utilizar a forma isotrópica gaussiana da Equação \ref{func_cor_1}:

\begin{equation}
C(r) = (1-\epsilon^{2})e^{-{r^2/l_c^2}},
\vspace{0.25cm}
\label{func_cor_2}
\end{equation}
onde $r = \sqrt{x^2 + y^2}$ e $l_c$ é o comprimento de correlação horizontal na 
direção radial \citep{silveira_etal2000B}.

O comprimento de correlação e a variância do erro amostral podem ser determinados 
de duas maneiras conhecidas: (1) a partir do ajuste não-linear da função de correlação amostral 
dos dados à forma teórica (Eq. \ref{func_cor_2}) ou (2) através do conhecimento prévio da 
estrutura do campo investigado e das principais feições que se quer realçar. Nesta dissertação, 
a determinação destes parâmetros é realizada através do caso (1), seguindo \cite{silveira_etal2000B}. 
Fica claro o grau de subjetividade implícito no caso (2). Em muitas ocasiões, isto pode resultar 
em campos estimados ``contaminados'' com estruturas as quais não conseguem ser resolvidas pela 
grade de observações e resultam em feições artificais. Assim como no caso (2), há também uma 
limitação associada ao caso (1), não de caráter 
subjetivo, mas sim metodológico. Nem sempre a correlação apresentada entre os dados observados e 
aqueles estimados respeita aquela assumida como gaussiana pela Equação \ref{func_cor_1} ou 
\ref{func_cor_2}. Logo, há uma implicação em determinarmos qual a forma da função de correlação, 
o que pode acarretar em um problema de solução a longo prazo.

\begin{figure}%[ht]
 \begin{center}
  \includegraphics[width=14cm,keepaspectratio=true]{/home/rafaelgs/mestrado/proc/climatologia/figuras_larga_escala/corr_woa_larga.pdf}
 \end{center}
 \vspace{-.25cm}
 \renewcommand{\baselinestretch}{1}
 \caption{\label{fig:func_corr} \small Correlação amostral da função de corrente 
 geostrófica calculada para os dados do WOA2001. Conforme ajuste não-linear para a Equação 
 \ref{func_cor_2} (linha vermelha), obtemos um comprimento de correlação $l_c = 488$ km 
 $\approx 4,4^\circ$ e uma variância do erro amostral aleatório $\epsilon^2 =$ 0,105.}
\end{figure}

Calculamos a correlação dos valores de $\psi_{(p_0/p)}$ para uma profundidade em que existam valores em 
todos os pontos de grade do WOA2001. A correlação amostral é apresentada na Figura \ref{fig:func_corr}, 
graficada em função da distância entre as estações (\textit{lag}). É fácil notar o caráter 
aproximadamente gaussiano apresentado pela correlação, sugerindo que a forma teórica proposta 
(Eq. \ref{func_cor_2}) é uma boa aproximação. Seu ajuste não-linear nos fornece um comprimento de 
correlação \textbf{$l_c =$ 488 km $\approx$ 4,4$^\circ$} e uma variância do erro amostral 
aleatório \textbf{$\epsilon^2 =$ 0,105}. 

% Consideramos, conforme sugerido 
% por \cite{denman_freeland1985}, que qualquer valor de correlação de $\psi_{(p_0/p)}$ que fuja das 
% proximidades de zero a uma distância muito maior que o primeiro raio de deformação 
% baroclínico típico da região, não possuem um embasamento na física envolvida e, portanto, 
% podem ser desconsiderados. Logo, de posse do valor médio aproximado do raio de deformação igual a 
% 80 km \citep{houry_etal1987}, limitamos a correlação em aproximadamente 1000 km, como 
% apresentado na Figura \ref{fig:func_corr}.

Em resumo, o emprego da $\mathcal{AO}$ como técnica de interpolação na construção dos campos 
horizontais de $\psi_{(p_0/p)}$ estima o valor desta quantidade para cada ponto pertencente à grade de 
interpolação de escolha. Os parâmetros $l_c =$ 4,4$^\circ$ e $\epsilon^2 =$ 0,105 são então 
utilizados. Como subproduto, a $\mathcal{AO}$ nos fornece o campo de erros, estimado a partir do erro 
médio quadrático, associado ao processo de interpolação. A Figura \ref{fig:grades} 
apresenta tal campo de erros de interpolação (percentual) considerando os parâmetros $l_c$ e 
$\epsilon^2$ determinados. 
A título de informação, a Figura \ref{fig:grades} ilustra a grade construída para 
interpolação, caracterizada por uma resolução de $60 \times 60$ pontos espaçados, em média, 
de 50 km.

\begin{figure}[hb]
 \begin{center}
  \includegraphics[width=16cm,keepaspectratio=true]{/home/rafaelgs/mestrado/proc/climatologia/figuras_larga_escala/er_woa.pdf}
 \end{center}
 \vspace{-.25cm}
 \renewcommand{\baselinestretch}{1}
 \caption{\label{fig:grades} \small Representação da grade para interpolação dos campos de $\psi_{(p_0/p)}$ 
 da climatologia WOA2001 sobreposta ao mapa do erro médio quadrático percentual de interpolação por 
 $\mathcal{AO}$: $l_c =$ 4,4$^\circ$ e $\epsilon^{2} =$ 0,105. A área correspondente aos domínios geográficos da
OEII está envolta pela caixa vermelha.}
\end{figure}

Há ainda um último aspecto importante a ser frisado sobre à construção dos campos de $\psi_{(p_0/p)}$. 
Determinadas estruturas espúrias presentes nos campos podem ter sua origem na forma como 
$\psi_{(p_0/p)}$ foi calculado nas proximidades da costa, onde são mais esparsos os perfis 
do WOA2001 e, adicionalmente, há 
a complicação de como lidar com as condições de contorno no lado costeiro do domínio. É necessário 
satisfazer as condições de contorno junto à borda oeste, no caso. Em particular, as 
condições de contorno de \textit{Dirichlet} são de implementação mais simples. Estas consistem 
simplesmente em estabelecer $\psi = constante$ no contorno sólido, em outras palavras, fluxo 
nulo normal ao contorno.

\cite{silveira_etal2000B} propõem que a condição de \textit{Dirichlet} seja aplicada durante o 
processo de mapeamento através da inclusão do que chamam de ``pseudo-dados'' no esquema de 
interpolação por $\mathcal{AO}$. Para tanto, a matriz com as coordenadas da isóbata da 
profundidade que se deseja mapear e valores de $\psi = 0$ no contorno são adicionados à 
matriz contendo as coordenadas das estações e os valores de $\psi_{(p_0/p)}$ calculados relativamente 
ao $\mathcal{NR}$.


\section{Função de Corrente Geostrófica Absoluta} \label{sec:psiref}

\subsection{Fundamentos Teóricos} \label{sec:psiref_teo}

\hspace{6mm} O arcabouço teórico utilizado aqui é o mesmo apresentado na seção precedente, ou seja, a geostrofia. 
Com isso, a formulação matemática para o cálculo das velocidades geostróficas permanece
inalterada. O que muda é que substituiremos o {\it nível de movimento nulo} ($\mathcal{NR}$) por um {\it nível de velocidade
conhecida}, oriunda dos dados de ADCP. Ao fazermos isto, estaremos migrando do {\bf Método Dinâmico Clássico} ($\mathcal{MDC}$) para 
o {\bf Método Dinâmico Referenciado} ($\mathcal{MDR}$). De agora em diante, quando nos referirmos ao $\mathcal{MDR}$
fazemos alusão àquele que utiliza-se das velocidades observadas por ADCP de casco. Neste caso, para nos fazermos mais claros, expandiremos a abordagem 
matemática. 

Na Seção \ref{sec:psigeo_teo}, mostramos como chegar à grandeza escalar $\psi_{(p_0/p)}$ através dos 
dados termohalinos, através das Equações \ref{eq:uvgpan} e \ref{eq:psigpan}. O que buscaremos agora
é uma equação que relacione uma componente qualquer da velocidade (meridional por exemplo) aos dados
termohalinos diretamente. Se combinamos as Equações \ref{eq:uvgpan} e \ref{eq:psigpan} e as escrevemos para  
a componente meridional, obtemos 

\begin{equation}
v(p) - v(p_0) = - \frac{1}{f_0} \int_{p_0}^p \frac{\partial \delta_\alpha}{\partial x} dp,
\vspace{0.5cm}
\label{eq:vgeo}
\end{equation}

que nada mais é do que uma forma diferente de representar o já exposto na Seção \ref{sec:psigeo_teo}, 
porém torna-se fundamental para entender o cerne do processo de referenciamento do Método Dinâmico. 

Na Equação \ref{eq:vgeo}, $v(p)$ é a velocidade em um nível isobárico qualquer de interesse e $v(p_0)$ corresponderia à 
velocidade no $\mathcal{NR}$. Ou seja, segundo o $\mathcal{MDC}$, $v(p_0) = 0$, mas para a aplicação do $\mathcal{MDR}$, simplesmente
$v(p_0)$ será a velocidade medida pelo ADCP de casco. A Figura \ref{fig:ex_metdinref} ilustra como é feito este cálculo
através da representação de um perfil de velocidade idealizado calculado por ambos os métodos: o $\mathcal{MDC}$, 
que se baseia em um $\mathcal{NR}$ e o $\mathcal{MDR}$, que se baseia em um nível de 
velocidades medidas diretamente. Supondo ser uma representação idealizada do sistema CB-SNB por exemplo, note
o quão inadequada pode se tornar a representação do perfil de velocidades caso a escolha do $\mathcal{NR}$ seja mal sucedida. 
À medida que utilizamos dados observados de velocidade, a representação fica mais próxima do cenário real, por nos
tornarmos livres da imposição de um nível {\bf constante} de escoamento nulo. No nosso caso, teremos inúmeros perfis, 
e em cada um deles, o valor de velocidade observada é único.

\begin{figure}%[ht]
 \begin{center}
  \includegraphics[width=16cm,keepaspectratio=true]{/home/rafaelgs/mestrado/figuras/ex_metdinref.pdf}
 \end{center}
 \vspace{-.25cm}
 \renewcommand{\baselinestretch}{1}
 \caption{\label{fig:ex_metdinref} \small Representação idealizada de um perfil de velocidade típico da 
costa brasileira evidenciando as diferenças entre métodos $\mathcal{MDC}$ e $\mathcal{MDR}$. O painel esquerdo
representa um perfil de velocidade geostrófica calculada através do $\mathcal{MDC}$ com NR = 1000 dbar, o painel central
exemplifica o caso do NR = 150 dbar e o painel direito mostra como se dá a aplicação do $\mathcal{MDR}$.}
\end{figure}

A próxima e crucial etapa consiste em selecionar um $v(p_0)$ que possibilite isolar o movimento geostrófico, tendo em vista
que as velocidades diretamente observadas são o resultado da ação de um conjunto de movimentos
diversos ocorrentes no oceano. Isto significa que devemos buscar apenas a 
componente geostrófica dos vetores de velocidade oriundos do ADCP, para que utilizemos a Equação \ref{eq:vgeo}
adequadamente. A melhor estratégia, segundo \cite{sutton_chereskin2002} e \cite{pickard_lindstrom1993},
está calcada na escolha de $p_0$, garantindo que este nível esteja
sob mínima influência possível das componentes ageostróficas da velocidade.
Segundo estes autores, existem três principais fontes de 
velocidades ageostróficas para regiões oceânicas, como esta que é objeto de estudo nesta dissertação.
São estas: 

\begin{itemize}

\item[$\checkmark$] a deriva de Ekman, provocada pela tensão de cisalhamento do vento;

\item[$\checkmark$] as correntes quase-inerciais;

\item[$\checkmark$] as correntes de maré.   

\end{itemize}

Dos três itens, as correntes quase-inerciais e as correntes de maré serão filtradas naturalmente através 
da interpolação por $\mathcal{AO}$ que aplicaremos aos dados, por terem freq\"uências relativamente altas quando comparadas às
correntes geostróficas. Para evitarmos a deriva de Ekman, pretendemos, de acordo com \cite{sutton_chereskin2002},
buscar níveis de profundidade onde o efeito direto da tensão de cisalhamento do vento é praticamente nulo.
Para tanto, conduzimos uma estimativa da profundidade média da camada de Ekman durante o período de duração da 
OEII. 

De acordo com \cite{cushman1994}, a profundidade da camada de Ekman pode ser determinada empiricamente 
através de 

\begin{equation}
h_E = \frac{\gamma}{\rho_0}u_*, 
\vspace{0.5cm}
\label{eq:hek}
\end{equation}

onde $\gamma$ é a constante de {\it Von Karman} e tem como valor 0,4. A variável $u_*$ é a {\it velocidade
friccional} que depende diretamente da tensão de cisalhamento do vento, através de

\begin{equation}
u_* = \sqrt{\frac{| \vec{\tau} |}{f_0}}.
\vspace{0.5cm}
\label{eq:ufric}
\end{equation}
  
Para obtermos uma estimativa do vento, retiramos os valores de tensão de cisalhamento ($\vec{\tau}$)
de campos oriundos do escaterômetro da NASA (QuickSCAT), recolhidos
para todo o período de duração da OEII. Trata-se de um sensor remoto montado em um satélite que mede
a tensão de cisalhamento do vento junto a superfície do mar através da rugosidade da mesma. Estes dados 
são conteúdo de um programa de sensoriamento remoto internacional que data desde o ano de 1998 até os dias atuais, 
e estão disponíveis gratuitamente na rede através do {\it website} da NASA. 
De posse de todos os valores de $h_E$ calculados através do uso destes dados na Equação \ref{eq:hek}, 
obtivemos seu valor médio, que foi de aproximadamente 77 m. Utilizamos os vetores de velocidade de corrente no nível
equivalente à aproximadamente o dobro da profundidade média da camada de Ekman calculada (150 m),
que estão certamente livres deste efeito.
Este nível de 150 m adicionalmente localiza-se em uma profundidade onde os dados de ADCP
ainda apresentam uma boa consistência, considerando sua 
capacidade de penetração.

Como nas investigações deste trabalho estamos construindo campos horizontais, na prática não usaremos a Equação 
\ref{eq:vgeo}, portanto voltemos a abordagem que 
considera $\psi$ em detrimento de tratar separadamente as componentes do vetor velocidade. Esta abordagem é 
mais elegante e confortável, pois carrega a informação dos dois componentes do vetor velocidade em uma só
grandeza escalar ($\psi$). Esta é também a abordagem utilizada por \cite{sutton_chereskin2002} que partem
do princípio que 

\begin{equation}
\psi_{tot} = \psi_{(p_0/p)} + \psi_{obs(p_0)}.
\vspace{0.5cm}
\label{eq:psiref}
\end{equation}

O procedimento de cálculo representado através da Eq. \ref{eq:psiref} consiste na soma
escalar de dois campos horizontais de $\psi$, e pode ser resumido em três etapas: 

\begin{enumerate}

 \item {\bf Cálculo de $\psi_{(p_0/p)}$:} consiste na construção de um campo de função de corrente
geostrófica segundo o $\mathcal{MDC}$, baseando-se unicamente na estrutura termohalina da OEII. 
O $p_0$ utilizado será convenientemente o de 150 m.
 
 \item {\bf Cálculo de $\psi_{obs(p_0)}$:} consiste na construção de um campo de função de corrente
observada baseando-se nos dados de velocidade observada via ADCP de casco durante a OEII. Este
campo será construído no nível de 150 m, de acordo com as razões apresentadas nos últimos parágrafos. 

 \item {\bf Cálculo de $\psi_{tot}$:} consiste na construção de um campo de função de corrente geostrófica
absoluta. Isto se dá com a soma dos demais campos construídos, de acordo com a expressão
apresentada na Eq. \ref{eq:psiref}. Notemos que a etapa 1 utiliza-se de um $p_0 = 150 m$, pois o campo
de função de corrente geostrófica absoluta neste nível deve ser idêntico ao campo de função de corrente
observada. Neste nível, somaremos o campo de $\psi_{obs(p_0)}$ com um campo de $\psi_{(p_0/p)}$ {\bf nulo}. 
O resultado destes cálculos em um panorama vertical é semelhante ao exposto na Figura \ref{fig:ex_metdinref}.

\end{enumerate}

\subsection{Mapeamento Objetivo} \label{sec:psiref_AO}

\hspace{6mm} Evidentemente, é necessário tecermos maiores esclarecimentos acerca da etapa 2, ou seja, do cálculo 
de $\psi_{obs(p_0)}$, uma vez que 
esta é oriunda da interpolação das componentes zonal ($u_{obs(p_0)}$) e meridional ($v_{obs(p_0)}$) do vetor velocidade fornecidas pelo 
ADCP, utilizando uma 
modalidade diferente da $\mathcal{AO}$: a {\bf Análise Objetiva Vetorial} ($\mathcal{AOV}$) \citep{bretherton_etal1976}.
Antes de esclarecermos
tal método, lembramos que os campos de $\psi_{(p_0/p)}$ são calculados exatamente aos moldes do que foi detalhado na Seção  
\ref{sec:psigeo_teo}, e dependem unicamente dos dados termohalinos. De acordo com esta abordagem ($\mathcal{MDC}$), o interpolador 
$\mathcal{AO}$
recebe como entrada $\psi_{(p_0/p)}$, e disponibiliza
como saída a mesma grandeza $\psi_{(p_0/p)}$ interpolada otimamente. Já para o caso de $\psi_{obs(p_0)}$ é necessária
a utilização da $\mathcal{AOV}$, pois as grandezas de entrada e saída são de natureza dimensional diferente. Temos
como entrada as componentes $u_{obs(p_0)}$ e $v_{obs(p_0)}$ e como saída $\psi_{obs(p_0)}$. Este procedimento é executado
de forma semelhante nos cálculos de \cite{silveira_etal2000B}, que fez uso de dados de velocidade observada diretamente via
perfilador PEGASUS. 

O princípio físico que torna possível tal cálculo está calcado em garantir a não-divergência do campo de velocidade 
após a interpolação, que é o caso dos cálculos realizados pela $\mathcal{AOV}$.
Naturalmente, mesmo em níveis profundos como o de 150 m,
longe da influência da tensão de cisalhamento do vento, ainda resta nos campos alguma divergência horizontal, 
oriunda de movimentos ageostróficos de mais baixa freq\"uência. Com isso, a $\mathcal{AOV}$ além de necessária
para transformar $u_{obs(p_0)}$ e $v_{obs(p_0)}$ em $\psi_{obs(p_0)}$, favorece ainda mais a eliminação
dos ruídos ageostróficos. 

Em síntese, o esquema de interpolação por $\mathcal{AOV}$ utiliza o princípio de que todo e qualquer campo
de velocidade, pode ser escrito segundo a decomposição de Euler, que consiste em 

\begin{eqnarray}
\vec{v} = (u,v) = \vec{k} \times \bigtriangledown \psi - \bigtriangledown \chi,
\label{eq:cauchyriemman}
\end{eqnarray}

onde $\chi$ é o potencial de velocidade, que representa a parte divergente e irrotacional do campo de velocidade,
e $\psi$ a função de corrente, que representa a parte  
rotacional e não-divergente. Para podermos obter um campo de função de corrente observada $\psi_{obs(p_0)}$, de acordo com sua definição \citep{kundu1990}, obrigatoriamente devemos
remover a divergência.  Os cálculos efetuados durante a $\mathcal{AOV}$ dão uma estimativa de $\psi_{obs(p_0)}$, assumindo que
o fluxo divergente é negligenciável, seguindo o princípio:
{\it ``Se existe uma relação funcional entre a velocidade não-divergente e $\psi_{obs(p_0)}$, existe também uma relação funcional 
entre suas funções de correlação, assumindo isotropia para o campo de velocidade.''} 

Assim como a $\mathcal{AO}$, a $\mathcal{AOV}$ tem a mesma propriedade de promover uma filtragem espaço-temporal
nos dados, de forma a eliminar as estruturas que estejam embebidas em uma escala não resolvida pelo desenho
amostral da OEII. Para tanto, devemos repetir os cálculos referentes à estimativa dos parâmetros de interpolação
{\bf $l_c$} e {\bf $\epsilon^2$}, de modo a manter o critério metodológico que permite utilizar a técnica de forma
ótima. Portanto, repetimos os cálculos descritos na Seção \ref{sec:psigeo_AO} para o caso de $\psi$ climatológico,
dessa vez utilizando $\psi$ sinótico, seja $\psi_{(p_0/p)}$ ou $\psi_{obs(p_0)}$. 
Escolhemos aqui o $\psi_{(p_0/p)}$ para esta estimativa, e assumimos que 
tanto para esta grandeza quanto para as velocidades observadas pelo ADCP, as escalas horizontais das estruturas
de interesse são as mesmas. 

Conforme esperávamos, o  {\bf $l_c$} estimado para os dados sinóticos é significativamente menor do que aquele
estimado para a climatologia, pois a mesma representa um cenário médio, onde as estruturas são de maior escala 
espacial. A Figura \ref{fig:corr_oeii} mostra a curva de 
auto-correlação e os valores dos parâmetros encontrados são de comprimento de 
correlação \textbf{$l_c =$ 95 km $\approx$ 1$^\circ$} e variância do erro amostral 
aleatório \textbf{$\epsilon^2 =$ 0,19}. Este valor menor de $l_c$, já nos dá indícios do quão maior será o nível 
de detalhamento contido no escoamento encontrado. A título de informação, a Figura \ref{fig:grades_oeii}
ilustra a grade curvilinear construída para 
interpolação, caracterizada por uma resolução de $50 \times 30$ pontos espaçados, em média, 
de 15 km. O mapa de erro percentual de interpolação está subposto à grade curvilinear na Figura \ref{fig:grades_oeii}.

A metodologia empregada para o cálculo de $l_c$ e $\epsilon^2$, apesar de ser meramente estatística, 
remete a um significado físico suficientemente robusto: estamos minimizando {\it aliasing} por estruturas
não resolvidas adequadamente pela amostragem. 


\begin{figure}%[hb]
 \begin{center}
  \includegraphics[width=12cm,keepaspectratio=true]{/home/rafaelgs/mestrado/proc/hidrografia/figuras/comp_corr_leste2.pdf}
 \end{center}
 \vspace{-.25cm}
 \renewcommand{\baselinestretch}{1}
 \caption{\label{fig:corr_oeii} \small Correlação amostral da função de corrente 
 geostrófica calculada para os dados da OEII. Conforme ajuste não-linear para a Equação 
 \ref{func_cor_2} (linha vermelha), obtemos um comprimento de correlação $l_c = 95$ km 
 $\approx 1^\circ$ e uma variância do erro amostral aleatório $\epsilon^2 =$ 0,19.}
\end{figure}

\begin{figure}%[ht]
 \begin{center}
  \includegraphics[width=12cm,keepaspectratio=true]{/home/rafaelgs/mestrado/proc/hidrografia/figuras/er_OEII_20m.pdf}
 \end{center}
 \vspace{-.25cm}
 \renewcommand{\baselinestretch}{1}
 \caption{\label{fig:grades_oeii} \small Representação da grade para interpolação dos campos da 
 da OEII sobreposta ao mapa do erro médio quadrático percentual de interpolação por 
 $\mathcal{AO}$: $l_c =$ 1$^\circ$ e $\epsilon^{2} =$ 0,19.}
\end{figure}

% O sensoriamento remoto é uma ferramenta poderosíssima de amostragem indireta dos oceanos. Imagens de satélite
% são obtidas diariamente em todo o globo, em poucos minutos. Dentro do escopo da circulação oceânica, podemos
% citar alguns tipos de produtos oriundos do sensoriamento remoto que nos auxiliam na identificação e interpretação
% de feições de escoamento. Imagens da temperatura da superfície do mar são excelentes traçadoras de frentes térmicas
% associadas à correntes de contorno, meandros, vórtices e ressurgências, pois tais feições são caracterizadas por
% gradientes térmicos marcantes. Na região sudeste brasileira é muito comum a ocorrência de ressurgências costeiras  e
% de plataforma, que fazem com que na maior parte do tempo, as águas na plataforma continental sejam mais frias do que 
% na porção oceânica adjacente. Com isso, o meandramento de uma corrente de contorno como a CB é capaz de advectar estas
% águas mais frias para a porção oceânica, ou injetar águas oceânicas para a plataforma continental. Ambos os mecanismos
% fazem com que os meandros sejam reconhecidos através de uma imagem térmica. 
%   
% Porém, em nossa região de interesse, os meandros e vórtices não possuem assinatura
% térmica, devido à ausência de águas mais frias junto a costa e a plataforma continental. Com isso, devemos procurar
% outro traçador para identificar estas feições. Um meandro ou vórtice é conhecido pela sua capacidade de 
% bombear água em direção à superfície (ciclônico) ou em direção à termoclina (anticiclônico). Em uma região oceânica, 
% caracterizada por águas pobres em nutrientes, um bombeamento de águas provenientes da termoclina para o interior
% da camada de mistura, por disponibilizar nutrientes na zona eufótica, pode promover ``blooms'' de fitoplâncton.
% Estas microalgas são ricas em clorofila, que convenientemente para este estudo, pode ser medida indiretamente
% por sensores remotos de satélite. 
% 
% Buscamos então imagens de clorofila disponíveis na rede, oriundas do satélite MODIS, que estivessem livre
% de contaminação por nuvens e que tenham sido obtidas dentro do período de realização da OEII. Escolhida uma
% imagem com essas características, comparamo-a com um campo de função de corrente geostrófica calculada através
% da metodologia descrita. Esta comparação está retratada na Figura \ref{fig:modis}. Ao inspecionarmos a figura, 
% imediatamente remetemos a atenção ao vórtice anticiclônico centrado em aproximadamente 11.7$^\circ$S e 35$^\circ$W.
% Fica evidente a correspondência entre as dimensões da estrutura nos dois métodos. Considerando que a imagem de 
% clorofila é instantânea e que os vetores de velocidade são calculados sobre um conjunto termohalino que demorou
% mais de um mês para ser obtido, temos aqui mais uma confirmação de que o $l_c$ calculado está fazendo com que as 
% estruturas estejam sendo representadas adequadamente. É mais um indício de que não existe o risco de uma 
% filtragem excessiva e tampouco uma contaminação por ``aliasing''.  
% 
% \begin{figure}%[ht]
%  \begin{center}
%   \includegraphics[width=10cm,keepaspectratio=true]{../proc/clorofila/figuras/clorofila_vgeo.pdf}
%  \end{center}
%  \vspace{-.25cm}
%  \renewcommand{\baselinestretch}{1}
%  \caption{\label{fig:modis} \small Sobreposição de vetores de velocidade geostrófica em superfície a
% um campo de concentração de clorofila oriundo do satélite MODIS.}
% \end{figure}

Antes de prosseguir, para melhorar o entendimento dos cálculos executados, expandiremos o exemplo idealizado exposto
na Figura \ref{fig:ex_metdinref2} para o espaço horizontal, que
será a forma de apresentação que nos possibilitará interpretação dos resultados.
Lembremos então, que para todo vetor $\vec{v} = u\vec{i} + v\vec{j}$ existe 
um escalar $\psi$, e vice-versa. Escolhemos a abordagem vetorial para este exemplo por esta permitir uma melhor visualização
dos cálculos executados. Para tanto, construímos campos idealizados das quantidades
$\psi_{(p_0/p)}$ e $\psi_{obs(p_0)}$ e somamo-os com a finalidade de obter um $\psi_{tot}$ também idealizado. 

A Figura \ref{fig:ex_metdinref2} apresenta os campos idealizados onde são representados vetores de velocidade não-divergente.
Estes vetores foram obtidos através do cálculo do gradiente horizontal das quantidades explicitadas no parágrafo anterior,
conforme a Equação \ref{eq:uvpsi}.
No painel superior esquerdo, é mostrado o campo idealizado de vetores de velocidade geostrófica em 10 m relativa a 150 m:
$\vec{v}_{(p_0/p)}$. O painel superior direito representa os vetores de velocidade observada por ADCP em 150 m isentos de sua componente
divergente: $\vec{v}_{obs(p_0)}$. O painel inferior representa o resultado da soma explicitada na Equação \ref{eq:psiref}, com
as devidas adaptações para uso vetorial: $\vec{v}_{tot} = \vec{v}_{(p_0/p)} + \vec{v}_{obs(p_0)}$. Esta é a forma com a qual
calcularíamos o campo de velocidades em 10 m, por exemplo.  Procedemos exatamente dessa 
forma com os dados reais observados durante a OEII para os vários níveis verticais de interesse deste trabalho.    

Após o esclarecimento e detalhamento da metodologia empregada nesta etapa do trabalho, nos resta
dar início a apresentação dos resultados. 


\begin{figure}%[ht]
 \begin{center}
  \includegraphics[width=7.5cm,keepaspectratio=true]{figuras/ex_metdinref2_1.pdf}
  \includegraphics[width=7.5cm,keepaspectratio=true]{figuras/ex_metdinref2_2.pdf}
  \includegraphics[width=7.5cm,keepaspectratio=true]{figuras/ex_metdinref2_3.pdf}
 \end{center}
 \vspace{-.25cm}
 \renewcommand{\baselinestretch}{1}
 \caption{\label{fig:ex_metdinref2} \small Campos idealizados de velocidade não-divergente ilustrando os cálculos correspondentes ao
 $\mathcal{MDR}$. O painel superior esquerdo idealiza a velocidade geostrófica em 10m calculada com $\mathcal{NR}$ = 150m. O
painel superior direito idealiza a velocidade observada via ADCP de casco em 150m, após remoção da componente divergente. O
painel inferior mostra o campo final, resultado da soma vetorial dos anteriores.}
\end{figure}














