
\hspace{6mm} A Corrente do Brasil (CB) se origina na bifurcação do ramo sul da  
Corrente Sul Equatorial (BiCSE), e flui para o sul ao longo da margem
continental brasileira. Inferências sobre a origem e organização da CB
são baseadas em padrões de grande escala associados à estrutura da
BiCSE. Há carência de detalhes no que tange à rica atividade de
meso-escala possivelmente envolvida em seu processo de formação.
São fartas as evidências de meandros e vórtices da CB na literatura ao
largo das costas sudeste e sul do Brasil.  A costa leste, que abriga
seu sítio de origem, permanece ainda bastante desconhecida neste ponto
de vista.  Motivamo-nos por esta lacuna de informações a conduzir este
trabalho. Optamos primeiramente por estimar um cenário médio do
escoamento geostrófico associado a BiCSE e origem da CB no período de
verão, através de um conjunto de dados termohalinos
climatológicos. Objetivamos por fim, obter uma descrição sinótica do
escoamento geostrófico associado ao sítio de origem da CB. Para tanto,
dispomos de um conjunto de dados composto por amostragem
quase-sinótica de dados termohalinos (via CTD) e de velocidade (via
ADCP de casco) simultaneamente, entre os paralelos de 10$^\circ$S e
20$^\circ$S.  Tais observações são oriundas da Operação Oceano Leste
II (OEII), realizada pela Marinha do Brasil em março de 2005.
Conduzimos a investigação a partir de um método científico que
entendemos não ter sido ainda empregado para os escoamentos ao largo
da costa brasileira. Tal método consiste no Método Dinâmico
Referenciado, que combina dados termohalinos e de velocidades
observadas para estimar velocidades geostróficas absolutas, que são
livres da dependência de um {\it nível de movimento nulo} imposto pelo
Método Dinâmico Clássico.  Este método substitui o {\it nível de
movimento nulo} por um {\it nível de velocidades conhecidas}, que
correspondem às velocidades medidas diretamente pelo ADCP de casco.
Encontramos para o cenário climatológico de verão, a assinatura da
BiCSE em 9$^\circ$S para superfície, em 12$^\circ$S para 200 m, em
20$^\circ$S para 500 m e em 23$^\circ$S para 800 m. Confirmamos
informações da literatura sobre sua migração meridional com o aumento
da profundidade. Identificamos também o cenário médio de origem e
organização da CB e da Sub-corrente Norte do Brasil (SNB) nos níveis
citados, os quais subsidiarão a análise dos campos sinóticos.  Os
campos sinóticos confirmaram os resultados médios da climatologia. Em
superfície, não observamos sinais da BiCSE no interior da área de estudo, e
mostramos que na ocasião da OEII, a CB se origina ao norte de
10$^\circ$S e começa sua organização ainda distante da margem
continental. Em 150 m, identificamos a assinatura da BiCSE em
14,5$^\circ$S, e em 500 m, em 20$^\circ$S. Sugerimos então que a
origem da CB enquanto corrente de contorno oeste, fluindo junto à
margem continental leste brasileira, se localiza em
14,5$^\circ$S. Concluindo, a CB se organiza ao norte de 10$^\circ$S e
flui distante da margem continental como uma corrente fraca e rasa,
atingindo no máximo os 100 primeiros metros de profundidade.  Esta
corrente atinge o contorno oeste em 14,5$^\circ$S e flui em direção ao
sul seguindo as isobatimétricas. Ao longo deste caminho, apresenta
meandros vigorosos, em particular, anticiclones. Estes anticiclones
parecem estar associados a feições topográficas relevantes da área,
tais quais os Bancos de Royal Charlotte e de Abrolhos. As velocidades
da CB aumentam em intensidade e a corrente ganha espessura vertical
até 20$^\circ$S. A SNB parece se originar em 20$^\circ$S, com núcleo
situado em 700 m. Esta corrente segue viagem rumo ao norte de forma
que seu núcleo se torna cada vez mais raso, e suas velocidades cada
vez maiores, até deixar o domínio em 10$^\circ$S. Nesta latitude, seu
núcleo se localiza a 250 m de profundidade.


