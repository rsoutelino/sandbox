\hspace{6mm} Nenhum agradecimento deve ser tecido antes daquele dirigido a 
minha família, em especial aos meus pais e meus avós. Sem uma base familiar e o 
ambiente por eles proporcionado durante a fase inicial de minha vida, sequer
estaria enfrentando um curso de pós-graduação desse quilate. Sua incondicional confiança em
minha capacidade permitiu que eu escolhesse este caminho sem incertezas ou inseguranças. 

Agradeço muito meu orientador Ilson da Silveira, que sempre compartilhou tam\-bém desta 
incondicional credibilidade, seja no âmbito profissional ou pessoal. Sempre um modelo
profissional para mim, é capaz de proporcionar um ambiente e interação de trabalho extremamente
agradáveis, fazendo-me acreditar sempre que a vida a\-ca\-dê\-mi\-ca é fascinante e ainda reserva para mim muitas
realizações. Agradeço também a Professora Sueli, pessoa de coração inestimavelmente grande, sempre
disposta a ajudar e oferecer conselhos valiosos. 

Fui acolhido neste laboratório do Instituto Oceanográfico antes do mestrado e lá também pude conhecer alunos
que foram ícones pra mim durante minha formação. Devo muito de meu aprendizado à paciência e 
boa vontade do quinteto Leandro Calado, André, Filipe, Rafael (Heyjow) e Cayo. Guardo ótimas
recordações de todos e espero continuar em contato com estes grandes exemplos pra mim. Alguns tive
mais contato, como o Filipe e o Heyjow, com os quais pude compartilhar dias de mar e vivenciar 
etapas práticas importantes para minha formação. Outros, tive menos contato direto, como André,
Cayo e Leandro, porém seus relatos de experiência e votos de incentivo foram de valor imensurável.

Agradeço aos atuais ``novatos'' do laboratório, Wellington (Doutorando), Leandro Ponsoni e Juliana (Mestrandos), César, Patrícia, 
Hermínio e Franco (Graduandos) pela confiança em mim sempre depositada.
Um agradecimento especial à Juliana, que devagarinho foi me impressionando e conquistando, e hoje é minha fiel companheira.
Ju, só posso dizer que é um luxo ter uma namorada tão linda, carinhosa, muito inteligente e ainda por cima
trabalhando comigo. Obrigado por segurar as pontas e me aturar nesse finalzinho bastante cansativo.
Wellington, você é um exemplo de vida, sempre, 
sob todas as óticas. Sua força de vontade é um exemplo a ser seguido, e suas sábias palavras, um combustível. Ponsoni, sempre
criterioso e competente, e lógico, bem humorado, excelente companheiro de mar. 
Não posso deixar de lado o Thiago (Paquito), que agora está no mercado exercendo sua profissão, 
obrigado por proporcionar inúmeros momentos de descontração.

Ao longo deste mestrado tive oportunidade de dividir meu lar com pessoas in\-crí\-veis. Os velhos Diogo e Paulo
que me acompanharam durante o primeiro ano deste curso, tenho muito a agradecer. Estiveram ao meu lado em
um dos momentos mais difíceis da minha vida. A calma e equilíbrio de ambos, além do companheirismo, fazem-me
recordar desta época com muitas lembranças boas, apesar das adversidades. Estes escolheram outros caminhos, mas 
continuam grandes amigos. Ainda como parte desta turma, não posso deixar de agradecer ao Sandro. Companheiro pra todas as horas,
exemplo de disciplina, dedicação e tenacidade, com certeza uma amizade para sempre. Fabrício, sem comentários, ninguém
fica triste perto dele, muito menos estressado. Obrigado pelos intermináveis momentos de descontração. 
Continuaremos por aqui para o doutorado... 
 
No segundo ano, em casa nova, novamente fui surpreendido com pessoas maravilhosas. Felipe e Marin chegaram
para seus cursos a procura de uma casa, e eu coincidentemente, precisava de novos integrantes para a minha. 
Foi assim que surgiu o ``2ndfloor'', que depois de ser base de estudos intensos durante o período de disciplinas, 
veio a se tornar base oficial para os momentos de descontração do pessoal da pós-graduação. 
Dois exemplos de dedicação, bom humor, companheirismo, integridade. Obrigado a vocês por proporcionarem
um ambiente excelente para se morar. Sim estou longe dos pais (com quem passei a maior parte de minha vida até hoje),  
mas sinto-me igualmente seguro de que sempre terei apoio quando necessário.

Obrigado aos meus amigos da pós, que 
apesar de não terem participado mais diretamente da minha vida, com certeza constituíram minha grande família
aqui em São Paulo. Só conheci pessoas sensacionais, como Raquel, Marina, Piero, Cássia, Sebastian, Melissa, Márcio, Fabíola, 
Huaringa, Dri, Janini, Bruno, Marcos, Gustavo e desculpem-me os que não me lembrei no momento desta redação.

Agradeço a todos os professores que tive ao longo do mestrado. Belmiro, Ilson, Paulo, Edmo, Sueli, obrigado
pelas horas de aprendizado durante as aulas, provas, intermináveis listas, etc. Saí do ciclo básico seguro de que
tinha finalmente uma formação básica teórica em Oceanografia Física. 

Agradeço especialmente a um professor que não fez parte deste curso formalmente, porém é e sempre será responsável 
pelo meu êxito. Professor Luiz Carlos, da UERJ, muito obrigado pelos inúmeros incentivos, confiança total, 
apoio incondicional. Tenha em mente que você sempre fará parte de minha carreira, e sempre lembrarei 
de você nos momentos fáceis, difíceis, nas vitórias e nas derrotas, procurando por incentivo.

A todos os amigos que fiz na UERJ, minha turma inesquecível, que tanto me amadureceu socialmente e 
continua até hoje fazendo parte da minha vida. Cito os mais chegados, como João, Dudu, Tatiana, Totonha,
Tiagão, Tomás, Saulo, Carol, e outros tantos. 

Agradeço, é claro, à Marinha do Brasil, pela disponibilização gratuita do valioso conjunto de dados 
utilizado nesta dissertação. Agradeço também ao CNPq por financiar meus estudos, num país com oportunidades
conhecidamente escassas. \\

Obrigado a todos! 






 
