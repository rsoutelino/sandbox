\section{Sumário e Conclusões}\label{sec:sum_conclu}

\hspace{6mm} Tendo finalmente descrito os principais padrões de escoamento em vários níveis com o auxílio dos campos de
$\psi$ construídos, urge uma organização e compilação das informações obtidas através da interpretação dos resultados.
Primeiramente, façamos uma breve síntese dos resultados encontrados para o escoamento climatológico de verão.

Ao longo da Seção \ref{sec:res_climatologico}, constatamos que em superfície, a BiCSE, e por conseq\"uência 
a origem da CB do ponto de vista climatológico, ocorrem em aproximadamente 10$^\circ$S. 
Em 200 m, encontramos um evidente sinal da BiCSE
em 12$^\circ$S, de onde partem os escoamentos climatológicos da CB e da SNB, para sul e para norte respectivamente.
Para os níveis de 500 m e 800 m, observamos que todo o escoamento na costa leste brasileira é composto
essencialmente pelo fluxo para norte da SNB. A BiCSE encontra-se em 20$^\circ$S e 23$^\circ$S, respectivamente
para os referidos níveis. Entendemos que a origem climatológica da SNB se localiza em aproximadamente 20$^\circ$S, 
pela inspeção do nível de 500 m. 

Passando à circulação sinótica, sintetizemos os resultados encontrados na Seção \ref{sec:res_sinotico}. 
Podemos por exemplo, dividir os fluxos das principais CCOs investigadas (CB e SNB) em três níveis verticais
principais. Tais níveis serão divididos de acordo com as principais diferenças observadas na estrutura tridimensional das
correntes em todos os níveis analisados. Serão então próximo a {\bf superfície} (20 m), {\bf topo da picnoclina} (150 m) 
e {\bf base da picnoclina} (500 m). Construímos para cada nível referido um desenho esquemático do escoamento, sobreposto
aos campos de $\psi$ já descritos. Apresentamos estes esquemas nas Figuras \ref{fig:esq20}, \ref{fig:esq150}
e \ref{fig:esq500}. 

Sintetizando o escoamento em nível de superfície, ou seja, de 0 - 20 m, apoiando-se na Figura \ref{fig:esq20}, 
reafirmamos que não há sinais do sítio da BiCSE. O que observamos é um fluxo para sul organizado e meandrante em 
todo o domínio investigado.
Atribuímos este fluxo à CB, que adentra no domínio afastada da margem continental e meandra vigorosamente
até se juntar à mesma em 14$^\circ$S. A partir desta latitude, o padrão meandrante é delineado pelos contornos topográficos,
e a CB ganha velocidade até deixar o domínio.
Também descrevemos as estruturas vorticais mais importantes e confrontamo-as com resultados
prévios para a mesma região. 

\begin{figure}%[ht]
 \begin{center}
  \includegraphics[width=15cm,keepaspectratio=true]{/home/rafaelgs/mestrado/proc/ref_adcp/figuras/esq_OEII_20m.pdf}
 \end{center}
 \vspace{-.5cm}
 \renewcommand{\baselinestretch}{1}
 \caption{\label{fig:esq20} \small Interpretação esquemática do cenário oceanográfico 
quase-sinótico ao largo da costa leste brasileira, em 20 m de profundidade, sobreposto ao campo de $\psi_{tot}$.
 Observamos o escoamento da CB em todo 
o domínio investigado, associado a uma intensa atividade de meso-escala. Destacamos que esta corrente se une à 
margem continental em aproximadamente 14,5$^\circ$S.
São evidenciadas as estruturas dos principais vórtices encontrados. São eles o 
Vórtice de Ilhéus (VI), Vórtice de Royal-Charlotte (VRC) e Vórtice de Abrolhos (VAb).}
\end{figure}

\begin{figure}%[ht]
 \begin{center}
  \includegraphics[width=15cm,keepaspectratio=true]{/home/rafaelgs/mestrado/proc/ref_adcp/figuras/esq_OEII_150m.pdf}
 \end{center}
 \vspace{-.5cm}
 \renewcommand{\baselinestretch}{1}
 \caption{\label{fig:esq150} \small Interpretação esquemática do cenário oceanográfico 
quase-sinótico, ao largo da costa leste brasileira, em 150 m de profundidade, sobreposto ao campo de $\psi_{tot}$.
Observamos a assinatura sinótica da chegada da CSE e de sua bifurcação aproximadamente no paralelo de 14,5$^\circ$S.
São evidenciados também os escoamentos da CB e SNB neste nível. Destacamos ainda as estruturas dos principais vórtices encontrados. São eles o Vórtice de Ilhéus (VI), Vórtice de Royal-Charlotte (VRC) e Vórtice de Abrolhos (VAb).}
\end{figure}

\begin{figure}%[ht]
 \begin{center}
  \includegraphics[width=15cm,keepaspectratio=true]{/home/rafaelgs/mestrado/proc/ref_adcp/figuras/esq_OEII_500m.pdf}
 \end{center}
 \vspace{-.5cm}
 \renewcommand{\baselinestretch}{1}
 \caption{\label{fig:esq500} \small Interpretação esquemática do cenário oceanográfico 
quase-sinótico, ao largo da costa leste brasileira, em 500 m de profundidade, sobreposto ao campo de $\psi_{tot}$.
Observamos a assinatura sinótica da origem, organização e escoamento da SNB. Destacamos ainda as
estruturas dos principais vórtices encontrados. São eles o Vórtice de Ilhéus (VI) e o Vórtice de Royal-Charlotte (VRC).}
\end{figure}

O cenário referente ao topo da termoclina (Figura \ref{fig:esq150}) já apresenta um robusto sinal da BiCSE em 
14,5$^\circ$S. Ao norte desta latitude temos a SNB se organizando em um jato junto a margem continental. Ao sul desta
latitude temos a CB também junto à margem continental exibindo uma intensa atividade de meso-escala, aparentemente
forçada pela complexa topografia local. Os mesmos vórtices descritos para a superfície têm também robusta estrutura 
no topo da termoclina. 

Atribuímos à base da picnoclina (500 m) o sítio de origem da SNB junto à costa leste brasileira.
A Figura \ref{fig:esq500} mostra
o que acreditamos ser a chegada da CSE em nível picnoclínico, fazendo com que a CCI receba este aporte e passe a ser
denominada SNB. A ausência do sinal da CB mostra que esta é uma corrente bastante rasa na costa leste. O escoamento
da SNB passa afastado do contorno oeste integrando a borda leste dos anticiclones VRC e VI, desde sua origem até 
aproximadamente 14$^\circ$S. Nesta latitude, junta-se à margem continental e segue seu percurso em direção ao norte.  

Considerando as CCOs presentes em nossas análises, ou seja, a SNB e a CB, faremos agora uma breve síntese sobre
seus respectivos escoamentos ao longo da costa leste. As Tabelas \ref{tab:cb} e \ref{tab:snb} sintetizam as principais características
de cada CCO em diversas latitudes e as compara com àquelas obtidas na literatura até a presente data. 

Aparentemente a origem da CB em nível de superfície não foi capturada pela OEII. Esta corrente já aparece
organizada em 10,5$^\circ$S, que corresponde à fronteira norte da grade amostral. Suas velocidades e espessura
vertical aumentam ao longo de sua viagem para o sul, passando de 0,25 m s$^{-1}$ e 50 m na fronteira norte para
0,5 m s$^{-1}$ e 200 m na fronteira sul. Os resultados encontrados aqui se comparam de forma razoavelmente favorável
com os resultados de pesquisas anteriores \citep{miranda_castro1981,soutelino2005,silveira_etal2006B},
conforme exposto na Tabela \ref{tab:cb}.

A origem da SNB parece ter sido capturada na profundidade de 500 m, uma vez que observamos um marcado 
fluxo zonal adentrando o domínio em sua fronteira sul/leste. A este fluxo zonal atribuímos a assinatura
sinótica de um ramo da CSE neste nível. Esta corrente então segue para o norte, porém nem sempre bordejando 
a margem continental brasileira. Até aproximadamente 14$^\circ$S, seu fluxo se distancia 
do contorno oeste. O escoamento da SNB tem seu núcleo cada vez mais raso, com velocidades 
cada vez maiores conforme diminui a latitude. Sua espessura vertical permanece praticamente inalterada, o que ocorre é apenas um
deslocamento de seu fluxo em direção a porções mais rasas da coluna de água. Seu núcleo que está em 700 m de profundidade
na fronteira sul do domínio, passa a ocupar os 250 m na fronteira norte. As velocidades no núcleo são de 0,45 m s$^{-1}$
em sua origem e passam a 0,58 m s$^{-1}$ ao deixar o domínio. 

Concluindo, acreditamos ser pertinente separar a costa leste brasileira em três principais setores diferentes.
 Chamamos de setor superior aquele associado a região
onde a SNB se encontra plenamente formada e fluindo junto ao contorno oeste. Podemos restringir este 
setor às três primeiras radiais de nossa análise ($\approx$ 10-13$^\circ$S), por exemplo.
 Escolhemos mostrar uma seção de velocidades geostróficas absolutas para 
a Radial 1, que se localiza em aproximadamente 10,5$^\circ$S, por ser representativa da estrutura vertical deste setor da costa leste (Figura \ref{fig:sec_vertical11S}). Como podemos
constatar na figura, apenas a presença da SNB se confirma, sem nenhum sinal da CB. Novamente vemos que a SNB se estende desde
a superfície até aproximadamente 1000 m. Estes resultados são bastante similares aos encontrados por \cite{schott_etal2005} 
(já mencionados no Capítulo \ref{cap:intro}, apresentados na Figura \ref{fig:schott_etal2005}), a menos do
transporte de volume, que é consideravalmente maior para o caso dos referidos autores. Reforçamos entretanto, que os 
resultados de \cite{schott_etal2005} se referem a uma média calculada entre 2000 e 2004, enquanto os resultados deste trabalho são 
considerados um cenário instantâneo em março de 2005.

\begin{figure}%[hb]
 \begin{center}
  \includegraphics[width=16cm,keepaspectratio=true]{/home/rafaelgs/mestrado/proc/ref_adcp/figuras/sec_adcp_objmap_rad1.pdf}
 \end{center}
 \vspace{-.5cm}
 \renewcommand{\baselinestretch}{1}
 \caption{\label{fig:sec_vertical11S} \small Seção \  vertical \ de \ velocidades \ geostróficas \ absolutas, baseadas no cálculo do 
$\mathcal{MDR}$, em 10,5$^\circ$S (painel superior). A máscara preta representa grosseiramente o perfil topográfico, com 
base na profundidade máxima alcançada pelo perfilador CTD em cada estação hidrográfica da radial em questão. O painel
inferior representa graficamente a localização geográfica da seção.}
\end{figure}

% \begin{figure}%[hb]
%  \begin{center}
%   \includegraphics[width=14cm,keepaspectratio=true]{../figuras/schott_etal2005.jpg}
%  \end{center}
%  \vspace{-.5cm}
%  \renewcommand{\baselinestretch}{1}
%  \caption{\label{fig:schott_etal2005-2} \small Distribuição vertical de velocidades médias observadas em
% 11$^\circ$S, segundo \cite{schott_etal2005}. Valores positivos para o norte.}
% \end{figure}

O setor central da costa leste se trata de uma zona de transição entre os setores inferior e superior. Podemos limitá-lo
entre as Radiais 4-8 da OEII ($\approx$ 13-17$^\circ$S). É neste setor que observamos a CB junto ao contorno oeste, a partir de 
14,5$^\circ$S. Como representativa da estrutura vertical neste setor, escolhemos exibir a seção vertical de 
velocidades geostróficas absolutas referente à Radial 8, conforme mostra a Figura \ref{fig:sec_vertical17S}, 
localizada em aproximadamente 17$^\circ$S. Nesta região, a CB encontra-se bem rasa, atingindo no máximo
200 m de profundidade e transportando 2,8 Sv para o sul, enquanto a SNB flui logo abaixo, robusta, 
com núcleo mais profundo do que aquele observado no setor superior, e transportando aproximadamente 13,8 Sv para o 
norte. Este valor é ligeiramente menor do que aquele observado no setor superior. 

\begin{figure}%[hb]
 \begin{center}
  \includegraphics[width=16cm,keepaspectratio=true]{/home/rafaelgs/mestrado/proc/ref_adcp/figuras/sec_adcp_objmap_rad8.pdf}
 \end{center}
 \vspace{-.5cm}
 \renewcommand{\baselinestretch}{1}
 \caption{\label{fig:sec_vertical17S} \small Seção \  vertical \ de \ velocidades \ geostróficas \ absolutas, baseadas no cálculo do 
$\mathcal{MDR}$, em 17$^\circ$S (painel superior). A máscara preta representa grosseiramente o perfil topográfico, com 
base na profundidade máxima alcançada pelo perfilador CTD em cada estação hidrográfica da radial em questão. O painel
inferior representa graficamente a localização geográfica da seção.}
\end{figure}

O setor inferior da costa leste será considerado aquele onde a CB já se encontra plenamente formada e organizada, fluindo junto
ao contorno oeste sobre o fluxo da SNB ($\approx$ 17-20$^\circ$S), 
que por sua vez ocupa níveis mais profundos do que no setor superior. Para
representar este setor, escolhemos exibir a seção vertical de velocidades geostróficas absolutas
referente à Radial 11 (Figura \ref{fig:sec_vertical19S}), que se localiza em aproximadamente 19$^\circ$S. Esta
seção nos mostra um padrão muito similar aos observados por outros autores na mesma latitude, como é o caso dos
resultados de \cite{soutelino2005}, citados no Capítulo \ref{cap:intro}, e exibidos através
da Figura \ref{fig:soutelino19}. A CB flui confinada nos primeiros 250 m de profundidade e 
transportando para sul aproximadamente 2,8 Sv. A SNB flui logo abaixo, exibindo um transporte de 13,1 Sv, 
 menor do que aqueles calculados para os setores central e superior. 

\begin{figure}%[hb]
 \begin{center}
  \includegraphics[width=16cm,keepaspectratio=true]{/home/rafaelgs/mestrado/proc/ref_adcp/figuras/sec_adcp_objmap_rad11.pdf}
 \end{center}
 \vspace{-.5cm}
 \renewcommand{\baselinestretch}{1}
 \caption{\label{fig:sec_vertical19S} \small Seção \  vertical \ de \ velocidades \ geostróficas \ absolutas, baseadas no cálculo do 
$\mathcal{MDR}$, em 19$^\circ$S (painel superior). A máscara preta representa grosseiramente o perfil topográfico, com 
base na profundidade máxima alcançada pelo perfilador CTD em cada estação hidrográfica da radial em questão. O painel
inferior representa graficamente a localização geográfica da seção.}
\end{figure}

% \begin{figure}%[hb]
%  \begin{center}
%   \includegraphics[width=14cm,keepaspectratio=true]{/home/rafaelgs/monografia/radial_13/figuras_rad13/vel_pom_rad13.pdf}
%  \end{center}
%  \vspace{-.5cm}
%  \renewcommand{\baselinestretch}{1}
%  \caption{\label{fig:soutelino_etal2005-2} \small Distribuição vertical de velocidades baroclínicas absolutas em
% 10$^\circ$S, segundo \cite{soutelino2005}. Valores positivos para o norte.}
% \end{figure}


\begin{table}
\caption{\label{tab:cb} \small Síntese dos resultados encontrados para a análise dos campos tridimensionais de 
$\psi$ no que se refere às diferentes características da CB ao longo da costa leste brasileira. Adicionalmente, 
comparamos os resultados aqui encontrados com àqueles oriundos de trabalhos anteriores.}
\begin{center}
\begin{sideways}
\renewcommand{\baselinestretch}{2}
% use packages: array
\begin{tabular}{|lcccccc|}
\hline 
%  & & & & &\vspace{-0.4cm}\\
 & & & {\bf CB em 10,5$^\circ$S} & & &\\ 
% \hline 
 & Espessura & Largura  & Núcleo & Vmáx & Transporte &  Método \vspace{-0.1cm}\\
 & (m) & (km) & (m) & (m s$^{-1}$) & (Sv) & Utilizado\\
\hline 
 & & & & & &\vspace{-0.4cm}\\
 Este Trabalho  & 0-50 & 70 & 0 & 0,25 & 1,7 & $\mathcal{MDR}$ \vspace{0.1cm} \\ 
\hline
 & & & & & &\vspace{-0.4cm}\\
 & & & {\bf CB em 17$^\circ$S} & & &\\  
% \hline 
& Espessura & Largura  & Núcleo & Vmáx & Transporte & Método \vspace{-0.1cm}\\
& (m) & (km) & (m) & (m s$^{-1}$) & (Sv) & Utilizado\\
\hline 
 & & & & & &\vspace{-0.4cm}\\
 Este Trabalho  & 0-160 & 100 & 0 & 0,58 & 2,8 & $\mathcal{MDR}$ \vspace{0.1cm} \\
\hline
 & & & & & &\vspace{-0.4cm}\\
 & & & {\bf CB em 19$^\circ$S} & & &\\  
% \hline 
& Espessura & Largura  & Núcleo & Vmáx & Transporte & Método \vspace{-0.1cm}\\
& (m) & (km) & (m) & (m s$^{-1}$) & (Sv) & Utilizado\\
\hline 
 & & & & & &\vspace{-0.4cm}\\
 \cite{miranda_castro1981} & - & 74 & 0 & 0,72 & 6,5 & $\mathcal{MDC}$ \vspace{0.1cm} \\
 \cite{soutelino2005}  & 0-300 & 100 & 0 & 0,6 & 5,1 & POMsec \vspace{0.1cm} \\
\cite{silveira_etal2006B}  & 0-200 & 60 & 0 & 0,5 & 3 & $\mathcal{MDC}$ \vspace{0.1cm} \\
Este Trabalho  & 0-200 & 100 & 0 & 0,45 & 2,8 & $\mathcal{MDR}$ \vspace{0.1cm} \\
\hline
\end{tabular}
\end{sideways}
\end{center}
\end{table}


\begin{table}
\caption{\label{tab:snb} \small Síntese dos resultados encontrados para a análise dos campos tridimensionais de 
$\psi$ no que se refere às diferentes características da SNB ao longo da costa leste brasileira. Adicionalmente, 
comparamos os resultados aqui encontrados com àqueles oriundos de trabalhos anteriores. }
\begin{center}
\begin{sideways}
\renewcommand{\baselinestretch}{2}
% use packages: array
\begin{tabular}{|lcccccc|}
\hline 
%  & & & & &\vspace{-0.4cm}\\
 & & & {\bf SNB em 10,5$^\circ$S} & & & \\ 
% \hline 
 & Espessura & Largura  & Núcleo & Vmáx & Transporte & Método \vspace{-0.1cm} \\
 & (m) & (km) & (m) & (m s$^{-1}$) & (Sv) & Utilizado\\
\hline 
 & & & & & &\vspace{-0.4cm} \\
\cite{silveira_etal1994}  & 50-900 & 120 & 150 & 0,5 & 23,7 & $\mathcal{MDC}$ \vspace{0.1cm} \\ 
\cite{stramma_etal1995}  & - & 70 & 200 & 0,5 & 22 & ADCP \vspace{0.1cm} \\ 
\cite{soutelino2005}  & 100-950 & 80 & 200 & 0,34 & 12,5 & POMsec \vspace{0.1cm} \\ 
\cite{schott_etal2005}  & 0-1500 & 170 & 250 & 0,6 & 23,8 &  Fundeio \vspace{0.1cm} \\ 
Este Trabalho  & 0-1000 & 100 & 250 & 0,58 & 14 & $\mathcal{MDR}$ \vspace{0.1cm} \\ 
\hline
 & & & & & &\vspace{-0.4cm} \\
 & & & {\bf SNB em 17$^\circ$S} & & & \\  
% \hline
& Espessura & Largura  & Núcleo & Vmáx & Transporte & Método \vspace{-0.1cm} \\
& (m) & (km) & (m) & (m s$^{-1}$) & (Sv) & Utilizado\\
\hline 
 & & & & & &\vspace{-0.4cm} \\
Este Trabalho  & 170-1400 & 120 & 700 & 0,5 & 13,8 & $\mathcal{MDR}$ \vspace{0.1cm} \\
\hline
 & & & & & &\vspace{-0.4cm} \\
 & & & {\bf SNB em 19$^\circ$S} & & & \\  
% \hline
& Espessura & Largura  & Núcleo & Vmáx & Transporte & Método \vspace{-0.1cm} \\
& (m) & (km) & (m) & (m s$^{-1}$) & (Sv) & Utilizado\\
\hline 
 & & & & & &\vspace{-0.4cm} \\
\cite{soutelino2005}  & 300-1300 & 70 & 700 & 0,25 & 4,3 & POMsec \vspace{0.1cm} \\
\cite{silveira_etal2006B}  & 200-1200 & - & 700 & 0,3  & 12,3 &  $\mathcal{MDC}$ \vspace{0.1cm} \\
Este Trabalho  & 200-1200 & 90 & 700 & 0,45 & 13,1 & $\mathcal{MDR}$ \vspace{0.1cm} \\
\hline
\end{tabular}
\end{sideways}
\end{center}
\end{table}


\newpage

\section{Sugestões para Trabalhos Futuros}\label{sec:trab_fut}

\hspace{6mm} A presente dissertação, através do mapeamento de função de corrente geostrófica absoluta, 
buscou esclarecer diversos aspectos da circulação sinótica ao longo da pouco conhecida costa leste brasileira. 
Inferimos as informações através de uma metodologia que combina dados hidrográficos e de 
velocidade via ADCP de casco para estimativa de um escoamento geostrófico absoluto. Tal metodologia revela-se
como uma ferramenta poderosa para o conhecimento da circulação geostrófica. Quando existe 
a disponibilidade de dados hidrográficos e de velocidade coletados via ADCP de casco simultaneamente, acreditamos
ser a forma mais indicada para se trabalhar. 

Entretanto, para avançar ainda mais no conhecimento da complexa circulação na costa leste, 
é imprescindível que medições diretas de velocidade sejam tomadas desde a superfície até o fundo.
Sugerimos a execução de levantamentos semelhantes à OEII, onde tais perfilagens sejam realizadas através
do uso de ``Lowered-ADCP'' (L-ADCP). 

Uma vez dispondo de dados desta natureza, acreditamos ser interessante avançar no estudo da dinâmica da
região, procurando inferir informações sobre a estacionaridade (ou não) das estruturas de meso-escala
aqui descritas. Sugerimos também estudos que procurem investigar a dinâmica de formação das estruturas
vorticais mais importantes, observadas ao longo desta dissertação. 

Várias iniciativas acerca de estudos numéricos de processos oceânicos e de caráter de previsão
 das correntes oceânicas estão presentemente 
em andamento. Os resultados aqui apresentados sugerem que experimentos prognósticos acerca da 
dinâmica da BiCSE, da CB e das demais CCOs presentes na costa leste, incluam modelos de feição
das principais estruturas descritas nesta dissertação, bem como assimilação de dados oriundos de imagens de satélite. 




 
