\documentclass{beamer}
% \documentclass[draft]{beamer}
% \documentclass[handout]{beamer}
\mode<handout>{\usepackage{pgf,pgfpages}}
\mode<beamer>{\usetheme{AnnArbor}}
\usepackage[utf8x]{inputenc}
\usepackage{bookman} 
\usepackage[T1]{fontenc} 
\usepackage{textcomp}
\usepackage{graphics}
\usepackage{animate}

\mode<presentation>

\setbeamertemplate{navigation symbols}{}

\definecolor{darkblue}{RGB}{22,52,136}
\definecolor{midblue}{RGB}{72,118,162}
\definecolor{midgray}{RGB}{127,127,127}
\definecolor{lightgray}{RGB}{204,204,204}
\definecolor{yell}{RGB}{120,10,2}

\setbeamercolor{alerted text}{fg=yell}
\setbeamercolor*{palette primary}{fg=black,bg=midblue}
\setbeamercolor*{palette secondary}{fg=darkblue,bg=lightgray}
\setbeamercolor*{palette tertiary}{bg=darkblue,fg=white}
\setbeamercolor*{palette quaternary}{fg=darkblue,fg=midgray}

\setbeamercolor*{sidebar}{fg=darkblue,bg=orange!75!white}

\setbeamercolor*{palette sidebar primary}{fg=darkblue!10!black}
\setbeamercolor*{palette sidebar secondary}{fg=white}
\setbeamercolor*{palette sidebar tertiary}{fg=darkblue!50!black}
\setbeamercolor*{palette sidebar quaternary}{fg=yellow!10!orange}

%\setbeamercolor*{titlelike}{parent=palette primary}
\setbeamercolor{titlelike}{bg=lightgray,fg=darkblue}
\setbeamercolor{frametitle}{bg=lightgray,fg=darkblue}
\setbeamercolor{frametitle right}{bg=midgray,fg=black}

\setbeamercolor*{separation line}{}
\setbeamercolor*{fine separation line}{}

%\usebackgroundtemplate{\includegraphics[width=\paperwidth]{logoio.png}}

\mode
<all>


% ------------------------------------------------------------
% comandos para fazer animações
% ffmpeg -i quake.flv -pix_fmt rgb24 -r 2 lixo.gif
% convert -antialias tsu.gif tsuj_%d.png
% se ficar ruim abra o .gif no gim, use animate -> unoptimize -> save e repita o passo anterior 
% ------------------------------------------------------------

% \mode<handout>{\pgfpagesuselayout{4 on 1}[a4paper,border
%   shrink=5mm,landscape]}

\title{Oceanografia Física dos Tsunamis} 
\subtitle{IFUSP 2011} 
\author[Paulo S. Polito]{Paulo S. Polito, Ph.D.  \hfill
  \texttt{polito@usp.br}}
\institute[IOUSP]{Instituto Oceanográfico da USP\\
\includegraphics[height=1.5cm]{logolos_310.png}}

\date{}
\mode<beamer>{\logo{\includegraphics[height=.5cm]{logolos_310.png}}}

\AtBeginSubsection[] {
  \begin{frame}<beamer> \frametitle{Roteiro}
    \tableofcontents[currentsubsection]
  \end{frame}
} \setbeamercovered{highly dynamic}

\begin{document}

% \includeonlyframes{a,b,c,d,e,f,g,h}

% ------------------------------------------------------------
\usebackgroundtemplate{\includegraphics[width=\paperwidth]{header.png}}
\begin{frame}
\vspace{1.5cm}
  \titlepage
\end{frame}
\usebackgroundtemplate{}

% ------------------------------------------------------------
\begin{frame}
  \frametitle{Roteiro}
  \tableofcontents
\end{frame}
% ------------------------------------------------------------
\section{Introdução}
% ------------------------------------------------------------
\subsection{Porque Estudar Oceanografia Física?}
% ------------------------------------------------------------
\begin{frame}{Alguns Fatos Sobre os Oceanos}
  \begin{columns}
    \begin{column}{5.5cm}   
      \begin{itemize}\setlength{\itemsep}{2ex}
      \item A Terra(?) é coberta por 70\% de água.
      \item 97.2\% da água está nos oceanos, 
      \item 1,8\% é gelo, 0,9\% é subterrânea, 
      \item \alert{Rios são só 0,02\%} e vapor 0.001\%.
      \item 1m$^3$ de água armazena mais de 4000 vezes mais calor que
        1m$^3$ de ar.
      \end{itemize}
    \end{column}
    \begin{column}{6.5cm}
      \centerline{\includegraphics[width=6cm]{earth.jpg}}
    \end{column}
  \end{columns}
\end{frame}
% ------------------------------------------------------------
\begin{frame}{As Pessoas e os Oceanos}
  \only<1>{  \begin{itemize}\setlength{\itemsep}{2ex}
    \item Mais de metade dos humanos vive a < 100km do mar.
    \item A largura típica dos oceanos é da ordem de 5000km.
    \item Pouquíssimas pessoas foram mais de 1 km mar adentro.
    \end{itemize}
    \hrulefill
    \begin{itemize}\setlength{\itemsep}{2ex}
    \item A profundidade média dos oceanos é da ordem de 5km.
    \item Pouquíssimos desceram mais de 10m abaixo da superfície.
    \item Mais gente já foi à Lua que à Fossa das Marianas (-10.911m).
    \end{itemize}
    \hrulefill
    \begin{itemize}\setlength{\itemsep}{2ex}
    \item \alert{Proporcionalmente, o Atlântico seria uma poça d'água
        com 5m de comprimento e 5mm de profundidade!}
    \end{itemize} }
  \only<2>{Proporcionalmente, seríamos bactérias na borda da poça... \\
    \centerline{\includegraphics[height=0.7\paperheight]{bacteria.jpg}}}
\end{frame}
% ------------------------------------------------------------
\begin{frame}{Oceanos $\times$ Pessoas}
  \begin{columns}
    \begin{column}{0.475\textwidth}
      \includegraphics[width=\textwidth]{banda_aceh.jpg}
    \end{column}
    \begin{column}{0.5\textwidth}
      \begin{itemize}\setlength{\itemsep}{1.5ex}
      \item São Vicente, 1541, 620 réis;
      \item Lisboa, 1755, 100.000 $\dagger$
      \item Messina, 1908 100.000 $\dagger$
      \item \alert{Indonésia, 2004, 300.000 $\dagger$}
      \item Chile, 2010, 700 $\dagger$
      \item Japão, 2011, ~30.000 $\dagger$
      \end{itemize}
      \centerline{\includegraphics[height=0.45\textwidth]{indonesia2004_2.jpg}}
    \end{column}
  \end{columns}
\end{frame}
% ------------------------------------------------------------
\subsection{Ondas Mecânicas}
% ------------------------------------------------------------
\begin{frame}[label=esquema]{O Vocabulário Correto}
  \centerline{\includegraphics[width=12cm]{waveparm.jpg}}
\end{frame}
% ------------------------------------------------------------
\begin{frame}
  \frametitle{Características Genéricas}
  \begin{columns}
    \begin{column}{4cm}   
      \centerline{\includegraphics[height=7.5cm,width=4cm]{bird.jpg}}
    \end{column}
    \begin{column}{8cm}
      \begin{itemize}\setlength{\itemsep}{2ex}
      \item Transferem \alert{uma perturbação} através de um meio (como no estádio).
      \item A perturbação é transmitida sem muito movimento de matéria.
      \item A transmissão não modifica muito o formato da perturbação.
      \item A velocidade de transmissão é geralmente constante.
      \item \alert{Ondas transportam energia sem efetivamente
          transportar matéria (vai--e--vem). }
      \end{itemize}
    \end{column}    
  \end{columns}  
\end{frame}
% ------------------------------------------------------------
\subsection{O Espectro do Oceano}
\begin{frame}[label=espectro]{A Energia de Cada Tipo de Onda}
  \centerline{\includegraphics[height=7.75cm]{spectrum2d.jpg}}
\end{frame}
% ------------------------------------------------------------
\section{Tsunami Como Onda}
% ------------------------------------------------------------
\begin{frame}
  \frametitle{Características Específicas}
  \begin{columns}
    \begin{column}{3.5cm}   
      \tiny{Fotos: National Geographic}

      \includegraphics[width=3.7cm]{tsu1.jpg}

      \includegraphics[width=3.7cm]{tsu2.jpg}

      \includegraphics[width=3.7cm]{tsu3.jpg}
    \end{column}
    \begin{column}{8.5cm}
      \begin{itemize}\setlength{\itemsep}{2ex}
      \item A onda tem amplitude pequena (<1m) enquanto
        em oceano profundo.
      \item A velocidade é constante apenas em
        mar profundo. 
      \item \alert{Na costa a interação com o fundo desacelera, aumenta a
          amplitude e muda a forma da onda}.
      \item O movimento de vai--e--vem aliado ao \alert{comprimento da onda
          longo e amplitude elevada} é justamente a causa da destruição. 
      \end{itemize}
    \end{column}    
  \end{columns}  
\end{frame}
% ------------------------------------------------------------
\subsection{Ondas de Gravidade}
% ------------------------------------------------------------
\begin{frame}
  \frametitle{Tsunamis são Ondas}
  \begin{itemize}\setlength{\itemsep}{2.5ex}
  \item Forçadas por terremotos, meteoros, vulcões, queda de barreiras submarinas etc..
  \item \alert{Difíceis de detectar e de prever a tempo.}
  \end{itemize}
  \centerline{\includegraphics[width=0.45\paperwidth]{tsunami_gen.png}\animategraphics[autoplay,loop,width=0.5\paperwidth]{8}{quake/quake_}{0}{49}}
\bigskip
\end{frame}
% ------------------------------------------------------------
\begin{frame}{Ondas Lineares de Gravidade}
    \begin{itemize}
    \item Períodos de $\sim$10~s a 1 dia e $\lambda$ de $\sim$10~cm a 1000~km.
    \item \alert{Em águas ``rasas'':}
      \begin{itemize}
      \item A velocidade {\bf não} depende do comprimento.
      \item {\bf Não} mudam de forma, {\bf não} se dispersam e vão longe.  
      \end{itemize}
    \item \alert{Em águas ``profundas'':}
      \begin{itemize}
      \item A velocidade depende do comprimento.
      \item Mudam de forma, se dispersam e {\bf não} vão longe.  
      \end{itemize}
    \item \alert{Rasa ou profunda em relação ao comprimento da onda:}
    \end{itemize}
    \centerline{\includegraphics[width=0.8\paperwidth]{aguasrasas.png}}  
\end{frame}
% ------------------------------------------------------------
\begin{frame}{Comprimento $\times$ Profundidade}
  \begin{itemize}
  \item \alert{Tsunamis são ondas longas ou de águas rasas:}
  \end{itemize}
  \animategraphics[autoplay,loop,width=12cm]{35}{sww/sww_}{0}{69} 
  \begin{columns}
    \begin{column}{6cm} 
      \begin{itemize}
      \item Ondas curtas ou de águas profundas:
      \end{itemize}
      \animategraphics[autoplay,loop,width=6cm]{35}{dww/dww_}{0}{69} 
    \end{column}
    \begin{column}{6cm}
      \begin{itemize}
      \item Na superfície as trajetórias se assemelham. 
      \item No fundo as ondas curtas não apresentam movimento.
      \item No fundo as ondas longas apresentam movimento horizontal.
      \end{itemize}
    \end{column}
  \end{columns}
\end{frame}
% ------------------------------------------------------------
\begin{frame}
  \frametitle{Tsunamis são Peculiares}
  \begin{columns}
    \begin{column}{0.2\paperwidth} 
      \includegraphics[width=0.2\paperwidth]{tsu_thai.jpg}
    \end{column}
    \begin{column}{0.8\paperwidth} 
      \begin{itemize}\setlength{\itemsep}{2.5ex}
      \item Em mar profundo (linear):
        \begin{itemize}\setlength{\itemsep}{1.5ex}
        \item Período da ordem de 30~min;
        \item Comprimento de onda da ordem de 300~km;
        \item Amplitude de apenas 30~cm--1m.
        \item \alert{Velocidade da ordem de 600~km/h;}
        \end{itemize}
      \item Na costa ocorre interação com o fundo: 
        \begin{itemize}\setlength{\itemsep}{1.5ex}
        \item Comprimento de onda encolhe para 20~km;
        \item Velocidade cai para 80~km/h;
        \item Formato muda para de senóide para degrau; 
        \item \alert{Amplitude aumenta para 5--30~m.}
        \end{itemize}
      \end{itemize}
    \end{column}
  \end{columns}
\end{frame}
% ------------------------------------------------------------
\againframe{espectro}
% ------------------------------------------------------------
\subsection{Resultados da Teoria Linear}
% ------------------------------------------------------------
\begin{frame}
  \frametitle{Velocidade de Fase (c)}
  \begin{columns}
    \begin{column}{2.5cm}   
      \centerline{\includegraphics[width=2.4cm]{hypfun.png}}
    \end{column}
    \begin{column}{9.5cm}
      \begin{itemize}\setlength{\itemsep}{1.5ex}
      \item Solução geral:
        \[c=\sqrt{\frac{g\lambda}{2\pi}\tanh\left(\frac{2\pi H}{\lambda}\right)}\]
      \item Se $H >> \frac{\lambda}{2}$ ondas ``curtas'',
        $\tanh\left(\frac{2\pi H}{\lambda}\right) \rightarrow 1$:
        \[c=\sqrt{\frac{g\lambda}{2\pi}}\]
      \item Se $H << \frac{\lambda}{2}$ ondas ``longas'',
        $\tanh\left(\frac{2\pi H}{\lambda}\right) \rightarrow \frac{2\pi
          H}{\lambda}$:\alert{
          \[c=\sqrt{gH}
          \quad \mbox{ com } g = 10\, ms^{-2} \mbox{ e } H = 5000\, m\]
          \[ \mbox{ temos } c=\sqrt{50.000} = 223\, ms^{-1} = 803\, km/h \]}
      \end{itemize}
    \end{column}    
  \end{columns}  
\end{frame}
% ------------------------------------------------------------
\begin{frame}
  \frametitle{Mudança de Forma}
  \begin{itemize}\setlength{\itemsep}{1.5ex}
  \item Tsunamis mudam de forma quando a \alert{amplitude} da onda
    tem a mesma ordem de grandeza da \alert{profundidade}.
  \item Em oceano profundo não acontece: $A\sim 1\,m$ e
    $H\sim 5\,km$.
  \item Na praia uma onda de 1~m pode representar 5\% ou mais da
    profundidade. A parte superior da onda se propaga mais rápido.
  \end{itemize} 
  \includegraphics[width=7cm]{steep.png}\includegraphics[width=4.5cm]{wave.jpg}
\end{frame}
% ------------------------------------------------------------
\begin{frame}
  \frametitle{Energia e Momentum}
  \begin{itemize}\setlength{\itemsep}{1.5ex}
  \item O momentum por área é dado por:
    \[ M \; = \; \frac{\frac{1}{2}\rho g A^2}{\sqrt{gH}} 
    \quad kg\,\frac{m}{s}\,\frac{1}{m^2}\]
  \item Por conservação, à medida que $H$ diminui, a velocidade 
    de fase diminui e \alert{a amplitude $A$ aumenta.}
  \item Para ondas longas a energia mecânica por área é:
    \[ E \;=\; Mc \;=\;\frac{1}{2}\rho g A^2 \quad \frac{J}{m^2} \]
  \item Observe que a energia depende da \alert{amplitude ao
      quadrado}.
  \end{itemize}
\end{frame}
% ------------------------------------------------------------
\section{Observações e Previsões}
% ------------------------------------------------------------
\subsection{Modelos}
\begin{frame}
  \frametitle{Teoria $\Rightarrow$ Modelo (propagação da fase)}
  \begin{columns}
    \begin{column}{7.5cm}
      \animategraphics[autoplay,loop,width=8cm]{5}{tsuj/tsuj_}{0}{99}
    \end{column}
    \begin{column}{4.5cm}
      \begin{itemize}\setlength{\itemsep}{2.5ex}
      \item Propagação de \alert{fase}.
      \item Os quatro pontos amarelos (alto, esq.) são bóias.
      \item A estrela vermelha é o epicentro.
      \item Há um gráfico para cada bóia.
      \item Note a forma da onda.
      \end{itemize}
    \end{column}
  \end{columns}
\end{frame}
% ------------------------------------------------------------
\begin{frame}
  \frametitle{Teoria $\Rightarrow$ Modelo (propagação da energia)}
  \begin{columns}
    \begin{column}{9cm}
      \includegraphics[width=9.5cm,height=7.75cm]{Energy_plot20110311-1000.png}
    \end{column}
    \begin{column}{3cm}
      \begin{itemize}\setlength{\itemsep}{3ex}
      \item $\triangle = $ bóia.
      \item Linhas = 1 hora.
      \item Cor = Amplitude
      \item \alert{Energia $\propto$ Amplitude$^2$}
      \end{itemize}
    \end{column}
  \end{columns}
\end{frame}
% ------------------------------------------------------------
\subsection{Observações}
% ------------------------------------------------------------
\begin{frame}
  \frametitle{Medidas por Satélites Altimétricos}
  \centerline{\includegraphics[width=0.475\paperwidth]{tsunami_envisat.png}\includegraphics[width=0.475\paperwidth]{tsunami_jason.png}}
  \begin{itemize}\setlength{\itemsep}{1ex}
  \item Usam tempo de retorno de radar para medir a anomalia do nível do mar numa área de 30~km de diâmetro.
  \item Uma medida a cada 7~km = 1~s, \alert{com precisão de 2~cm}.
  \item Cobrem o planeta a cada 10 dias (Jason) ou 35 dias (Envisat) $\leadsto$ não servem para monitoramento ou alerta.
  \item \alert{Comprimento da ordem de 4--5$^\circ$ e a amplitude de 20~cm.}
  \end{itemize}
\end{frame}
% ------------------------------------------------------------
\begin{frame}
  \frametitle{Medidas In--Situ}
  \begin{columns}
    \begin{column}{0.58\paperwidth}
      \centerline{\includegraphics[width=0.58\paperwidth]{boia.png}}
    \end{column}
    \hspace{0.01\paperwidth}
    \begin{column}{0.25\paperwidth}
      \centerline{\includegraphics[width=0.25\paperwidth]{dart-buoy-ndbc-2006c.jpg}}

      \centerline{\includegraphics[width=0.25\paperwidth]{boia2.jpg}}
    \end{column}
  \end{columns}
\bigskip
\end{frame}
% ------------------------------------------------------------
\begin{frame}
  \frametitle{Medidas $\times$ Modelo}
  \begin{itemize}\setlength{\itemsep}{0ex}
  \item As 8 bóias estão em mar profundo no Pacífico (Japão 2011).
  \end{itemize}
  \begin{columns}
    \begin{column}{0.6\paperwidth}
      \includegraphics[width=0.65\paperwidth]{20110311_darts_comp.png}
    \end{column}
    \hspace{0.01\paperwidth}
    \begin{column}{0.3\paperwidth}
      \begin{itemize}\setlength{\itemsep}{2ex}
      \item \alert{As previsões são boas.}
      \item 4 de cima estão a menos de 2h, amplitude $\simeq$ 1m.
      \item 4 de baixo estão a 3--6h, amplitude $\simeq$ 20cm.
      \end{itemize}
    \end{column}
  \end{columns}
\end{frame}
% ------------------------------------------------------------
\section{Sumário}
% ------------------------------------------------------------
\begin{frame}
  \frametitle{Em resumo...}
  \begin{itemize}\setlength{\itemsep}{2ex}
  \item Tsunamis são ondas de gravidade longas (águas rasas).
  \item São gerados, na maioria dos casos, por terremotos.
  \item A fase se propaga rápido, o tempo é curto para que se acione o alarme, a defesa civil etc.
  \item A amplitude é pequena e cresce junto à costa.
  \item A energia depende do quadrado da amplitude.
  \item Satélites detectam tsunamis {\em a posteriori}.
  \item As observações baseadas em sensores de pressão validam
    as simulações numéricas e podem salvar vidas.
  \end{itemize}
\end{frame}
% ------------------------------------------------------------
\begin{frame}{Sobre Tsunamis Acabou...}
\LARGE
Mas se me permite perguntar...

\vspace{4cm}

\Large
\hfill...você se interessou por oceanografia física?
\normalsize
\end{frame}
% ------------------------------------------------------------
\begin{frame}[allowframebreaks]{\normalsize Sobre Oceanografia Física na USP}
  \begin{itemize}\setlength{\itemsep}{2ex}
  \item Oceanografia se divide em 4 áreas:
    \begin{itemize}
    \item \alert{Física,}
    \item Química,
    \item Biologia e
    \item Geologia.
    \end{itemize}
  \item O programa de pós--graduação em Oceanografia
    \begin{itemize}
    \item Oferece vagas de Mestrado e Doutorado;
    \item Tem nível 5$^\dagger$ na Capes;
    \item Tem nível internacional e
    \item Admite alunos de outras áreas:
      \begin{itemize}
      \item Meteorologia,
      \item Física,
      \item Engenharia,
      \item Matemática Aplicada,
      \item Computação etc.
      \end{itemize} 
    \end{itemize}
  \item Que tipo de problemas se discute?
    \begin{itemize}\setlength{\itemsep}{1.5ex}
    \item Ondas, todo o espectro,teoria e observações;
    \item Correntes, fluxo de massa e momentum;
    \item Nível do mar, 
    \item Fluxo de calor e sal, relação com clima,
    \item Modelagem numérica, (e.g. derramamento de óleo, IPCC), 
    \item Técnicas estatísticas de análise de dados,
    \item Influência de correntes e ondas na biologia,
    \end{itemize}
  \item A Oceanografia Física no IOUSP tem duas linhas de pesquisa:
    \begin{itemize}
    \item Hidrodinâmica costeira e estuarina e
    \item Circulação oceânica de meso e larga escala.
    \end{itemize}
  \item Essas pesquisas se desenvolvem nos laboratórios de:
    \begin{itemize}
    \item \alert{Oceanografia por Satélites},
    \item Dinâmica Oceânica,
    \item Hidrodinâmica Costeira,
    \item Simulação Hidrodinâmica,
    \item Modelagem dos Oceanos e
    \item Oceanografia, Clima e Criosfera.
    \end{itemize}
  \item Por fim, \alert{onde vão parar nossos egressos}?
    \begin{itemize}
    \item Institutos de pesquisa: INPE, CPTEC, Marinha, NIWA (Nova
      Zelândia), NOAA (EUA) etc.
    \item Universidades (docência): USP, UFABC, UFBA, UERJ, UNESP, UFSC etc.
    \item Universidades (Ph.D., pos-doc):  MIT/WHOI, Caltech/NASA, URI, OSU,
      UMiami, URI, UMass, Rutgers (EUA), LOCEAN (França) etc.
    \item Iniciativa privada: petróleo e meio--ambiente (consultoria).   
    \end{itemize}
  \end{itemize}
\end{frame}
% ------------------------------------------------------------
% ------------------------------------------------------------
\begin{frame}
  \centerline{\includegraphics[width=\paperwidth]{obrigado.jpg}}
  \begin{itemize}
  \item Saiba mais em \alert{www.io.usp.br} e/ou converse com os
    professores do IOUSP. É uma opção de futuro.
  \end{itemize}
\end{frame}
% ------------------------------------------------------------
\end{document}
