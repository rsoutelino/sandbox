
\documentclass[12pt,portuguese,a4paper,pdftex]{article}
%\usepackage{epstopdf}
\usepackage[pdftex]{hyperref}
%\documentclass[12pt,a4paper]{article}
\usepackage[brazil]{babel}
%\usepackage[latin1]{inputenc} %traduz acentos
\usepackage{natbib}
%\usepackage{epsf,psfig}
\usepackage[dvips]{graphicx}
%\usepackage[square]{natbib}
\usepackage{subfigure}
\usepackage{rotating}
\usepackage{amssymb}
\usepackage{ucs}
\usepackage[utf8x]{inputenc}

\renewcommand{\baselinestretch}{1.5}
\hoffset=-1cm %regula a margem esquerda 
\topmargin -1cm %regula a margem superior
\textwidth 16cm %largura da pagina
\textheight 24.5cm %altura da pagina

\begin{document}



\begin{titlepage}

%\voffset=-90pt

\begin{center}
{\large \bf UNIVERSIDADE DO ESTADO DO RIO DE JANEIRO\\
\vspace{-0.3cm}
CENTRO DE TECNOLOGIA E CI\^ENCIAS\\
\vspace{-0.3cm}
INSTITUTO DE GEOCI\^ENCIAS\\
\vspace{-0.3cm}
DEPARTAMENTO DE OCEANOGRAFIA\\} 
\end{center}

\vspace{3cm}

\begin{center}
{\large \bf Rafael Guarino Soutelino\\} 
\end{center}

\vspace{2cm}

\begin{center}
%\raisebox{0pt}[0pt][0pt]

{\LARGE \bf \sc Caracteriza\c c\~ao da Estrutura Barocl\'inica do Sistema de Correntes de Contorno Oeste 
ao Largo da Costa Leste Brasileira}\\

\end{center}
\renewcommand{\baselinestretch}{1.2}

\vspace{2cm}

\hspace{2.5in}
\parbox[t][2.0in][t]{3.7in}{\small Monografia apresentada $\nearrow{k}$ ao Curso de Oceanografia do Instituto de Geociências
da Universidade do Estado do Rio de Janeiro, como requisito final para a
obtenção do grau de Bacharel em Oceanografia.}\\

\vspace{-2cm}

\begin{center}
{ORIENTADOR:\\
Prof. Dr. Ilson Carlos Almeida da Silveira}
\end{center}

\vspace{2cm}

\begin{center}
{\small RIO DE JANEIRO, \\
Dezembro de 2005}
\end{center}
\end{titlepage}


\newpage
%\pagestyle{empty}
%\pagestyle{headings}
%\pagenumbering{roman}
\tableofcontents

\newpage
\pagestyle{plain}
\addcontentsline{toc}{section}{Lista de Figuras}

\begin{center}
\section*{Lista de Figuras}
\end{center}
\listoffigures

\newpage
\pagestyle{plain}
\addcontentsline{toc}{section}{Lista de Tabelas}

\begin{center}
\section*{Lista de Tabelas}
\end{center}
\listoftables

\newpage
\pagenumbering{arabic}
\renewcommand{\baselinestretch}{1}

\pagestyle{plain}
\setcounter{page}{1}

\section{Introdução}

%%%%%%%% EXEMPLO DE INCLUSAO DE FIGURAS
\vspace{-0.5cm}

\begin{figure}[ht]
\begin{center}
\includegraphics[width=10cm,keepaspectratio=true]{figuras/streng99_ACAS_nova.pdf}
\end{center}
\vspace{-0.5cm}
\renewcommand{\baselinestretch}{.5}
\caption{\label{fig:streng2} \small{Padr\~oes de circula\c c\~ao de larga-escala ao n\'ivel de 200 metros para o Atl\^antico Sul. 
Adaptado de \cite{stramma_england1999}.}}
\end{figure}


    \subsection{Área de Estudo}
    \subsection{Objetivos}

\section{Metodologia e Resultados}
    \subsection{Meteorologia}
    \subsection{Maré}
    \subsection{Correntografia}
    \subsection{Hidrografia}
\section{Síntese dos Resultados}
\section{Conclusões}


\newpage

%%%%%%%%%% BIBLIOGRAFIA %%%%%%%%%%%%%%%%%%%%%%%
\addcontentsline{toc}{section}{Referências}

\bibliography{bibliografia}

\bibliographystyle{tese}
%\bibliographystyle{apalike}
%\bibliographystyle{abbrvnat}
%%%%%%%%%%%%%%%%%%%%%%%%%%%%%%%%%%%%%%%%%%%%%%%

%\input{refbib.tex} % bibliografia antes do bibtex (antiga)

\end{document}
