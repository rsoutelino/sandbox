\documentclass[12pt,portuguese,a4paper,pdftex]{article}

\usepackage[square]{natbib}
\usepackage[brazil]{babel}
\usepackage[pdftex]{hyperref}
\usepackage{natbib}
\usepackage[dvips]{graphicx}
\usepackage{figure}
\usepackage{rotating}
\usepackage{amssymb,amsmath}
\usepackage{ucs}
\usepackage[utf8x]{inputenc}
\usepackage{nicefrac}
\usepackage{fancyhdr}
\usepackage{multirow}
% \usepackage{palatino}
\usepackage{color}
\usepackage{cancel}
\usepackage{shadow}
\usepackage{helvet} %família de fontes helvet, fonte helvetica = arial 
\renewcommand{\rmdefault}{phv} % necessario para colocar fonte arial


\renewcommand{\baselinestretch}{1.5}
% \hoffset -12.8mm %regula a margem esquerda
% % \topmargin -0.5in %regula a margem superior
% \textwidth 160mm %largura da pagina
% % \textheight 8.5in %altura da pagina

\topmargin -6mm %regula a margem superior
\hoffset -11mm %regula a margem esquerda
\textwidth 160mm %largura da pagina
\textheight 240mm %altura da pagina

\renewcommand{\headrulewidth}{0.0pt}
\renewcommand{\footrulewidth}{0.0pt}

\begin{document}

%%% CAPA %%%
\begin{titlepage}

%\voffset=-90pt


\vspace{2cm}

\begin{center}  
%\raisebox{0pt}[0pt][0pt]

\LARGE
\textbf{Descrição Sinótica da Origem da Corrente do Brasil}\\

\end{center}
\renewcommand{\baselinestretch}{1.2}

\vspace{2cm}


\begin{center}
{\large Aluno:\\
\textbf{Rafael Guarino Soutelino}}
\end{center}


\begin{center}
{\large Orientador:\\
\textbf{Ilson Carlos Almeida da Silveira}}
\end{center}

\vspace{2cm}

\begin{center}
{\large Laboratório de Dinâmica Oceânica - LaDO\\
Departamento de Oceanografia Física\\
Instituto Oceanográfico da Universidade de São Paulo - IOUSP}
\end{center}

\vspace{2cm}

\begin{center}
{\large \textbf{Fevereiro de 2008}}
\end{center}

\vspace{2cm}

\begin{center}
{\large Relatório Científico III - Mestrado}
\end{center}

\end{titlepage}

% FOLHA DE ROSTO %%%%%%%%%%%%%%%%%%%%%%%%%%%%%%%%%%%%%%%%%%%%%%%%%%
\newpage
\pagestyle{empty}

\hspace{-7.5mm} {\bf Aluno:} Rafael Guarino Soutelino\\
{\bf Modalidade:} Mestrado \\
{\bf Orientador:} Prof. Dr. Ilson Carlos Almeida da Silveira\\
{\bf Área de Atuação:} Oceanografia Física\\
{\bf Projeto:} ``Descrição Sinótica da Origem da Corrente do Brasil''\\
{\bf Período de Atividades:} agosto/2007 à fevereiro/2008\\
{\bf Relatório Número:} 03

\vspace{6cm}

\begin{center}

\underline{\hspace{10cm}}\vspace{0.3cm}\\
Ilson Carlos Almeida da Silveira

\vspace{3cm}

\underline{\hspace{10cm}}\vspace{0.3cm}\\
Rafael Guarino Soutelino

\end{center}


% INDICE %%%%%%%%%%%%%%%%%%%%%%%%%%%%%%%%%%%%%%%%%%%%%%%%%%%%%%%%%%
\newpage
\pagenumbering{arabic}
\tableofcontents


% RESUMO DO PLANO INICIAL %%%%%%%%%%%%%%%%%%%%%%%%%%%%%%%%%%%%%%%%%%%%%%%%%%%%%%%

\newpage
\section*{Resumo do plano inicial}
\addcontentsline{toc}{section}{Resumo do plano inicial}
\pagestyle{plain}

\hspace{4mm}  A bifurca\c c\~ao da Corrente Sul Equatorial (CSE), ao se aproximar da
costa brasileira, em superfície, ocorre nos entornos de 15$^\circ$S, resultando
 na Corrente Norte do Brasil (CNB), que flui para o norte e a
Corrente do Brasil (CB), para o sul. Em n\'iveis picnocl\'inicos, a
bifurca\c c\~ao ocorre em 20$^\circ$S em m\'edia, onde parte flui para
o norte, acompanhando a Corrente de Contorno Intermedi\'aria (CCI),
contribuindo com a forma\c c\~ao da Sub-corrente Norte do Brasil
(SNB), e parte vai para o sul, espessando verticalmente o fluxo da CB.
Na literatura, h\'a uma car\^encia de conhecimento acerca da
estrutura din\^amica do sistema de corrente de contorno oeste nos
trechos onde ocorrem a bifurca\c c\~ao de superfície e de nível picnoclínico da CSE.  O objetivo central desta
proposta \'e caracterizar dinamicamente o sistema de correntes destas
regi\~oes. Para tanto, estimar-se-á a estrutura vertical de
velocidades barocl\'inicas atrav\'es do uso de um modelo num\'erico
seccional para cálculo das velocidades a partir do campo
de massa de uma dada se\c c\~ao hidrogr\'afica. Buscar-se-á a comparação
com a climatologia Levitus \citep{boyler_etal2005} e campos oriundos de cruzeiros quase-sinóticos.  
A dissertação, ora em andamento, é parte integrante do Projeto ``Produtividade, 
Sustentabilidade e Utilização do Ecossistema do Banco de Abrolhos", 
financiado dentro do programa dos Institutos do Milênio (CNPq, Pr. 
2540208954204759).

% RESUMO - ANO I %%%%%%%%%%%%%%%%%%%%%%%%%%%%%%%%%%%%%%%%%%%%%%%%%%%%%%%

\newpage
\section*{Resumo das Atividades - Ano I}
\addcontentsline{toc}{section}{Resumo das Atividades - Ano I}

\hspace{4mm}  No âmbito do plano de trabalho inicial apresentado em março de 2007 à Comissão de Pós-graduação 
do  Instituto Oceanográfico da Universidade de São Paulo, algumas mo\-di\-fi\-ca\-ções 
foram feitas em relação aos objetivos inicialmente propostos. Tais modificações se tornaram interessantes
devido à disponibilidade de um novo conjunto de dados englobando a região de estudo, que permitiram
análises novas que nos motivaram a investigar de forma mais regional os aspectos dinâmicos 
da região de estudo. As modificações em questão foram discutidas e apresentadas no Relatório I, assim
como as etapas cumpridadas durante o Ano I (fevereiro/2006 à fevereiro de 2007) de realização desta dissertação de mestrado. 

Após o realinhamento do cronograma de trabalho apresentado no Relatório I, foi feita uma descrição 
das etapas cumpridas. Foi apresentado um novo conjunto de dados, 
que permitiu novas análises para o trabalho, enriquecendo o poder de in\-ter\-pre\-ta\-ção da circulação 
oceânica na área de estudo, ou seja, a elaboração de mapas de função de corrente geostrófica 
e observada em vários níveis. A metodologia a ser empregada nessas análises foi descrita em detalhe
e finalmente foi traçado um novo cronograma para o Ano II. O Relatório II se encarregou de 
descrever as atividades desenvolvidas na primeira metade do Ano II.  Considerando estarmos, na presente data, no final do 
Ano II, já temos praticamente todo o cronograma cumprido, a menos da finalização da redação do documento. Nos dedicamos, então,
neste Relatório III a descrever as etapas restantes cumpridas durante o Ano II.

% Cronograma de Trabalho - Ano I %%%%%%%%%%%%%%%%%%%%%%%%%%%%%%%%%%%%%%%%%%%%%%%
\newpage
\section{Atividades - agosto/2007 à fevereiro/2008}\label{sec:new}

\hspace{4mm} Recapitulando o que foi descrito no Relatório I, mostramos o cronograma referente ao Ano II.

\begin{enumerate}

\item Tratamento básico dos dados de velocidade observada (ADCP) da operação Oceano Leste II.

\item Mapeamento de função de corrente observada em vários níveis utilizando o conjunto 
de dados de ADCP da operação Oceano Leste II.

\item Mapeamento de função de corrente geostrófica para a climatologia  
{\it World Ocean Atlas 2001}, em área de maior abrangência em relação ao cruzeiro sinótico.

\item Descrição, discussão e interpretação dos resultados.

\item Divulgação dos resultados preliminares em congressos ou encontros.

\item Redação e defesa da dissertação.

\end{enumerate}

Dos itens acima, até a apresentação do Relatório II, haviam sido cumpridos 1, 3 e 5. Ao longo do desenvolvimento referente
ao Item 2, notamos a necessidade de adaptá-lo, diante da possibilidade de aplicar uma metodologia nova, que foi amplamente
investigada do ponto de vista de pesquisa bibliográfica. Trata-se do {\bf Método Dinâmico Referenciado}, que traz grandes vantagens 
em relação ao {\bf Método Dinâmico Clássico} e se torna viável ao passo que dispomos simultanemante de dados termohalinos e de 
velocidade. Esta metodologia será descrita em detalhe ao longo deste relatório. Com esta nova metodologia, fomos conduzidos 
a remodelar o cronograma para esta fase final de análise de dados da dissertação. De modo a clarificar as etapas realizadas
listaremos um novo conjunto de itens que se destinaram a ser cumpridos neste período de agosto/2007 à fevereiro/2008.

\begin{enumerate}

 \item Mapeamento de função de corrente
geostrófica segundo o {\bf Método Dinâmico Clássico}, baseando-se unicamente na estrutura termohalina da operação OEII, utilizando para
tanto um nível de movimento nulo raso o suficiente para que haja medidas de ADCP concomitantes, e profundo o suficiente para 
evitar as componentes ageostróficas da velocidade (150 m).
 
 \item Mapeamento de função de corrente
observada baseando-se nos dados de ADCP de casco durante a operação OEII. Este
campo será construído no nível de 150 m, de acordo com as razões acima. 

 \item Mapeamento de função de corrente geostrófica
absoluta conforme aplicação do {\bf Método Dinâmico Referenciado}.
 Isto se dá com a soma dos demais campos construídos.

\end{enumerate}

Estas 3 etapas foram cumpridas ao longo deste período e serão descritas detalhadamente ao longo do relatório.

   \subsection{Atividades Paralelas}\label{sec:extras}

\hspace{4mm} Mesmo que não constando no cronograma para o ano, acreditamos ser interessante 
descrever a participação do aluno em atividades extras, que não estão exatamente no escopo da dissertação, 
mas contribuem robustamente para sua formação acadêmica. 

Durante o segundo semestre letivo de 2007, o aluno participou do Programa de Aperfeiçoamento 
de Ensino da USP (PAE). Este programa consiste em um estágio de monitoria em uma disciplina 
do curso de graduação em Oceanografia do IOUSP, mais precisamente do ciclo da Oceanografia Física. 

Durante este estágio, sob supervisão do Professor Ilson Carlos Almeida da Silveira, responsável pela disciplina 
``Oceanografia Dinâmica II - IOF1222'', ministrada durante o quarto ano da grade curricular, o aluno 
pôde participar das atividades do docente durante o curso da disciplina. Suas atividades consistiram em
auxiliar o professor na elaboração e correção de provas e listas de exercícios, participar das aulas 
ministradas e promover plantões de dúvidas com os alunos matriculados. Tal atividade é valiosíssima para formação acadêmica
 do aluno, uma vez que o fazem adquirir uma primeira experiência em docência no ensino superior. 

   \subsection{Produção Científica}\label{sec:prod}

\hspace{4mm} Embora fuja do escopo da dissertação, durante o período, o aluno submeteu um artigo para a revista científica
``Revista Brasileira de Geofísica'' (RBGf), com co-autoria de seu orientador Ilson Carlos Almeida da Silveira. O conteúdo científico
deste artigo corresponde à sua monografia de graduação, que foi desenvolvida ao longo de 2005. Segue abaixo os dados do artigo.   

\begin{itemize}

\item[$\checkmark$] {\it The boundary current system off the eastern brazilian coast}\\ 
{\bf Soutelino, R.G.}; Silveira, I.C.A.; Cirano, M.;  de Paula, A. C.

\end{itemize}


\subsection{Aplicação do Método Dinâmico Referenciado}\label{sec:psi_oeii}

\hspace{6mm} O arcabouço teórico utilizado aqui é o mesmo, ou seja, a geostrofia. Apesar de estarmos
lidando com dados de velocidade observada, pretendemos ainda trabalhar em um espectro do movimendo no qual a geostrofia
é dominante. Com isso, a formulação matemática para o cálculo das velocidades geostróficas permanece
inalterada. O que muda é que substituiremos o {\it nível de movimento nulo} por um {\it nível de velocidade
conhecida}, oriunda dos dados de ADCP. Ao fazermos isto, estaremos migrando do {\bf Método Dinâmico Clássico} ($\mathcal{MDC}$) para 
o {\bf Método Dinâmico Referenciado} ($\mathcal{MDR}$). De agora em diante, quando nos referirmos ao $\mathcal{MDR}$
estamos fazendo alusão àquele que utiliza-se das velocidades observadas por ADCP de casco. Neste caso, para nos fazermos mais claros, expandiremos a abordagem 
matemática. Faremos isso, decompondo o escalar $\psi$ nas componentes zonal e meridional do vetor velocidade, 
conforme sua própria definição \citep{kundu1990} para um campo não-divergente horizontalmente, ou seja, onde $\nabla_H . \vec{v} = 0$, que é o caso do escoamento geostrófico. Temos então que  

\begin{equation}
u = \frac{\partial \psi}{\partial y} \ \ \ \ \ \ \ \ \  e  \ \ \ \ \ \ \ \ \  v = -\frac{\partial \psi}{\partial x}.
\vspace{0.5cm}
\label{eq:uvpsi}
\end{equation}

Recapitulando a equação do $\mathcal{MDC}$, em função de uma das componentes ($v$) da velocidade por exemplo, com a finalidade
de demonstrar como é feito o cálculo, temos

\begin{equation}
v(p) - v(p_0) = - \frac{1}{f_0} \int_{p_0}^p \frac{\partial \delta_\alpha}{\partial x} dp,
\vspace{0.5cm}
\label{eq:ugeo}
\end{equation}

onde $v(p)$ é a velocidade em um nível isobárico qualquer de interesse e $v(p_0)$ corresponderia à 
velocidade no $\mathcal{NR}$. Ou seja, segundo o $\mathcal{MDC}$, $v(p_0) = 0$. Para esta aplicação ($\mathcal{MDR}$), simplesmente
$v(p_0)$ será a velocidade medida pelo ADCP de casco. A Figura \ref{fig:ex_metdinref} ilustra como é feito este cálculo
através da representação de um perfil de velocidade idealizado calculado por ambos os métodos: o $\mathcal{MDC}$, 
que se baseia em um $\mathcal{NR}$, e o $\mathcal{MDR}$, que se baseia em um nível de 
velocidades medidas diretamente. Supondo ser uma representação idealizada do sistema CB/CCI por exemplo, note
o quão inadequada pode se tornar a representação do perfil de velocidades caso a escolha do $\mathcal{NR}$ seja mal sucedida. 
A medida que utilizamos dados observados de velocidade, a representação fica mais próxima ao cenário real, e ainda nos
tornamos livres de utilizar um nível {\bf constante} imposto de velocidades nulas. No nosso caso, teremos inúmeros perfis, 
e em cada um deles, o valor de velocidade observada é diferente.

\begin{figure}%[ht]
 \begin{center}
  \includegraphics[width=15cm,keepaspectratio=true]{/home/rafaelgs/mestrado/figuras/ex_metdinref.pdf}
 \end{center}
 \vspace{-.25cm}
 \renewcommand{\baselinestretch}{1}
 \caption{\label{fig:ex_metdinref} \small Representação idealizada de um perfil de velocidade típico da 
costa brasileira evidenciando as diferenças entre diferentes métodos: o $\mathcal{MDC}$ e o 
$\mathcal{MDR}$.}
\end{figure}

A próxima etapa, crucial, consiste em selecionar um $v(p_0)$ compatível com a teoria geostrófica, tendo em vista
que as velocidades diretamente observadas são uma soma de componentes oriundas de um conjunto de forçantes que 
não se limitam apenas às forçantes do movimento geostrófico. Isto significa que teremos que isolar apenas a 
componente geostrófica dos vetores de velocidade oriundos do ADCP, para que utilizemos grandezas de mesma natureza ao
aplicar a Eq. \ref{eq:ugeo}. A melhor estratégia para tal, de acordo com 
 \cite{pickard_lindstrom1993} e \cite{sutton_chereskin2002} está na escolha de $p_0$. Estes autores sugerem que 
por ser muito difícil eliminar as componentes ageostróficas da velocidade em um nível onde as mesmas são 
importantes, a melhor estratégia é escolher um $p_0$ que esteja sob mínima influência possível destas
componentes. Segundo estes autores, existem três principais fontes de 
velocidades ageostróficas para regiões oceânicas como esta que estamos estudando: 

\begin{itemize}

\item[$\checkmark$] a deriva de Ekman, provocada pela tensão de cisalhamento do vento;

\item[$\checkmark$] as correntes quase-inerciais;

\item[$\checkmark$] as correntes de maré.   

\end{itemize}

Dos três itens, as correntes quase-inerciais e as correntes de maré serão filtradas naturalmente através 
da interpolação por $\mathcal{AO}$ que aplicamos aos dados, por terem freq\"uências relativamente altas quando comparadas às
correntes geostróficas. Para evitarmos a contaminação dos campos pela deriva de Ekman, pretendemos, de acordo com \cite{sutton_chereskin2002},
buscar níveis de profundidade onde o efeito direto da tensão de cisalhamento do vento praticamente esvaneceu. 
Para tanto, utilizamos os vetores de velocidade correspondentes ao nível de 150 m, que estão praticamente livres deste efeito,
e localizam-se em uma profundidade onde os dados de ADCP ainda apresentam uma boa consistência, considerando sua 
capacidade de penetração.

Como nas investigações deste trabalho estamos construindo campos horizontais, voltemos a abordagem que 
considera $\psi$ em detrimento de tratar separadamente as componentes do vetor velocidade. Esta abordagem é 
mais elegante e confortável, pois carrega a informação dos dois componentes do vetor velocidade em uma só
grandeza escalar ($\psi$). Esta é também a abordagem utilizada por \cite{sutton_chereskin2002} que partem
do princípio que 

\begin{equation}
\psi_{tot}(p) = \psi(p_0-p) + \psi_{obs}(p_0).
\vspace{0.5cm}
\label{eq:psiref}
\end{equation}

O procedimento de cálculo representado através da Eq. \ref{eq:psiref} consiste na soma
 de dois campos horizontais de $\psi$, e pode ser resumido em três etapas: 

\begin{enumerate}

 \item {\bf Cálculo de $\psi_g(p_0-p)$:} consiste na construção de um campo de função de corrente
geostrófica segundo o $\mathcal{MDC}$, baseando-se unicamente na estrutura termohalina da operação OEII. 
O $\mathcal{NR}$ utilizado será convenientemente o de 150 m.
 
 \item {\bf Cálculo de $\psi_{obs}(p_0)$:} consiste na construção de um campo de função de corrente
observada baseando-se nos dados de velocidade via ADCP de casco durante a operação OEII. Este
campo será construído no nível de 150 m, de acordo com as razões apresentadas nos últimos parágrafos. 

 \item {\bf Cálculo de $\psi(p)$:} consiste construção de um campo de função de corrente geostrófica
absoluta. Isto se dá com a soma dos demais campos construídos, de acordo com a expressão
apresentada na Eq. \ref{eq:psiref}. Note que a etapa 1 utiliza-se de um $\mathcal{NR} = 150 m$, pois o campo
de função de corrente geostrófica absoluta neste nível deve ser idêntico ao campo de função de corrente
observada. Neste nível, somaremos o campo de $\psi_{obs}(p_0)$ com um campo de $\psi_g(p_0-p)$ {\bf nulo}. 
O resultado destes cálculos em um panorama vertical é semelhante ao exposto na Figura \ref{fig:ex_metdinref}.

\end{enumerate}

\vspace{1cm}

Evidentemente, é necessário tecermos maiores esclarecimentos à cerca da etapa 2, ou seja, do cálculo 
de $\psi_{obs}(p_0)$, uma vez que 
esta é oriunda da interpolação das componentes $u$ e $v$ do vetor velocidade fornecidas pelo ADCP, utilizando uma 
modalidade diferente da $\mathcal{AO}$: a {\bf Análise Objetiva Vetorial} ($\mathcal{AOV}$) \citep{bretherton_etal1976}.
Antes de esclarecermos
isto, enfatizamos que os campos de $\psi_g(p_0-p)$ são calculados exatamente aos moldes do $\mathcal{MDC}$, e dependem unicamente dos dados termohalinos. De acordo com esta abordagem, o interpolador $\mathcal{AO}$
recebe como entrada $\psi_g(p_0-p)$, e disponibiliza
como saída a mesma grandeza $\psi_g(p_0-p)$ interpolada otimamente. Já para o caso de $\psi_{obs}(p_0)$ é necessária
a utilização da $\mathcal{AOV}$, pois as grandezas de entrada e saída são de natureza dimensional diferente. Temos
como entrada as componentes $u$ e $v$ do vetor velocidade e como saída $\psi_{obs}(p_0)$.

O princípio físico que torna possível tal cálculo está calcado em garantir a não-divergência do campo de velocidade 
após a interpolação, que é o caso dos cálculos realizados pela $\mathcal{AOV}$.
Naturalmente, mesmo em níveis profundos como o de 150 m,
longe da influência da tensão de cisalhamento do vento, ainda resta nos campos alguma divergência horizontal, 
oriunda de movimentos ageostróficos de mais baixa freq\"uência. Com isso, a $\mathcal{AOV}$ além de necessária
para transformar $u$ e $v$ em $\psi$, nos dá mais uma garantia da eliminação praticamente completa dos ruídos
para a geostrofia. 

Em síntese, o esquema de interpolação por $\mathcal{AOV}$ utiliza o princípio de que todo e qualquer campo
 de velocidade, pode ser escrito segundo a decomposição de Cauchy-Riemman, que consiste em 

\begin{eqnarray}
\vec{v} = \vec{v}_{irot} + \vec{v}_{ndiv},
\label{eq:cauchyriemman}
\end{eqnarray}

onde $\vec{v}_{irot}$ é a parte irrotacional e divergente do campo de velocidade, e $\vec{v}_{ndiv}$ a parte 
rotacional e não-divergente. Para podermos obter um campo de função de corrente, obrigatoriamente devemos
remover a divergência.  Os cálculos efetuados durante a $\mathcal{AOV}$ seguem o seguinte princípio:
{\it ``Se existe uma relação funcional entre $\vec{v}_{ndiv}$ e $\psi$, existe também uma relação funcional 
entre suas funções de correlação, assumindo isotropia para o campo de velocidade.''} 

De acordo com 
\cite{bretherton_etal1976}, a Análise Objetiva em geral, é baseada num resultado estatístico padrão chamado Teorema 
de Gauss-Markov. Este teorema fornece uma expressão para a estimativa de erros mínimos 
quadráticos de variáveis físicas (temperatura, salinidade, velocidade, função de corrente, 
vorticidade, etc), sendo a estatística do campo estimada na forma de espectro espaço-temporal. 
Mapas de erros são obtidos através de estimativas do erro médio quadrático. 

\cite{carter_robinson1987} mostram que a Análise Objetiva de dados oceânicos pode ser pensada 
como um ajuste por mínimos quadrados onde as funções-peso dependem da correlação entre os dados. 
Logo, sua implementação requer um conhecimento prévio da função de correlação das 
variáveis, $C$, e da variância do erro amostral aleatório, $\epsilon^2$. Seguindo 
\cite{emery_thomson1998}, embora a especificação da matriz de correlação deva ser determinada na 
estrutura observada das variáveis do oceano, há a opção de se utilizar uma forma matemática teórica 
ajustada e baseada nesta estrutura. É o caso sugerido quando os dados amostrados são esparsos, 
como acontece com o conjunto OEII.

Para o mapeamento horizontal de escoamentos geofísicos, buscamos utilizar uma função 
de correlação dada por:

\begin{equation}
C(\Delta x,\Delta y) = (1-\epsilon^{2})e^{-\left [ (\Delta x)^2/l_x^2 + (\Delta y)^2/l_y^2 \right ]},
\vspace{0.25cm}
\label{func_cor_1}
\end{equation}
onde $\Delta x$ e $\Delta y$ representam os incrementos espaciais horizontais nas 
direções zonal e meridional, respectivamente, $l_x$ e $l_y$ são os chamados comprimentos de 
correlação e $\epsilon^{2}$ é a variância do erro amostral aleatório.

Para essa natureza de escoamentos, as diferenças entre os comprimentos de correla\-ção 
horizontais $l_x$ e $l_y$ costumam ser sutis como consequência da notável isotropia horizontal dos 
mesmos. Em muitos casos e principalmente naqueles envolvendo mapeamento de quantidades 
com estruturas de caráter radial, tal como característico das feições de larga escala, como a BiCSE,
e as feições ligadas à atividade de meso-escala da CB (meandros e vórtices),
é convidativo utilizar a forma isotrópica gaussiana da Equação \ref{func_cor_1}:

\begin{equation}
C(r) = (1-\epsilon^{2})e^{-{r^2/l_c^2}},
\vspace{0.25cm}
\label{func_cor_2}
\end{equation}
onde $r = \sqrt{x^2 + y^2}$ e $l_c$ é o comprimento de correlação horizontal na 
direção radial.

O comprimento de correlação e a variância do erro amostral podem ser determinados 
de duas maneiras conhecidas: (1) a partir do ajuste não-linear da função de correlação amostral 
dos dados à forma teórica (Eq. \ref{func_cor_2}) ou (2) através do conhecimento prévio da 
estrutura do campo investigado e das principais feições que se quer realçar. Nesta dissertação, 
a determinação destes parâmetros é realizada através do caso (1), seguindo \cite{silveira_etal2000B}. 
Fica claro o grau de subjetividade implícito no caso (2). Em muitas ocasiões, isto pode resultar 
em campos estimados ``contaminados'' com estruturas as quais não conseguem ser resolvidas pela 
grade de observações e resultam em feições artificais. Assim como no caso (2), há também uma 
limitação associada ao caso (1), não de caráter 
subjetivo, mas sim metodológico. Nem sempre a correlação apresentada entre os dados observados e 
aqueles estimados respeita aquela assumida como gaussiana pelas Eqs. \ref{func_cor_1} ou 
\ref{func_cor_2}. Logo, há uma implicação em determinarmos qual a forma da função de correlação, 
o que pode acarretar em um problema de solução a longo prazo.

A Figura \ref{fig:corr_oeii} mostra a curva de 
auto-correlação e os valores dos parâmetros encontrados são de comprimento de 
correlação \textbf{$l_c =$ 95 km $\approx$ 1$^\circ$} e variância do erro amostral 
aleatório \textbf{$\epsilon^2 =$ 0,19}. A título de informação, a Figura \ref{fig:grades_oeii}
 ilustra a grade curvilinear construída para 
interpolação, caracterizada por uma resolução de $50 \times 30$ pontos espaçados, em média, 
de 15 km.


\begin{figure}%[ht]
 \begin{center}
  \includegraphics[width=12cm,keepaspectratio=true]{/home/rafaelgs/mestrado/proc/hidrografia/figuras/comp_corr_leste2.pdf}
 \end{center}
 \vspace{-.25cm}
 \renewcommand{\baselinestretch}{1}
 \caption{\label{fig:corr_oeii} \small Correlação amostral da função de corrente 
 geostrófica calculada para os dados da Operação OEII. Conforme ajuste não-linear para a Equação 
 \ref{func_cor_2} (linha vermelha), obtemos um comprimento de correlação $l_c = 95$ km 
 $\approx 1^\circ$ e uma variância do erro amostral aleatório $\epsilon^2 =$ 0,19.}
\end{figure}

\begin{figure}%[ht]
 \begin{center}
  \includegraphics[width=12cm,keepaspectratio=true]{/home/rafaelgs/mestrado/proc/hidrografia/figuras/er_OEII_20m.pdf}
 \end{center}
 \vspace{-.25cm}
 \renewcommand{\baselinestretch}{1}
 \caption{\label{fig:grades_oeii} \small Representação da grade para interpolação dos campos da 
 da Operação OEII sobreposta ao mapa do erro médio quadrático percentual de interpolação por 
 $\mathcal{AO}$: $l_c =$ 1$^\circ$ e $\epsilon^{2} =$ 0,19.}
\end{figure}

A metodologia empregada para o cálculo de $l_c$ e $\epsilon^2$, apesar de ser meramente estatística, 
remete a um significado físico suficientemente robusto para adotarmos os valores calculados com a garantia
de que estamos evitando um futuro mapeamento contaminado por ``aliasing''. Ainda assim, seria interessante
obter uma contra-prova de que estes parâmetros estão adequados para construir campos que representem de forma
fidedigna as estruturas que pretendemos investigar. De modo a comprovar então a autenticidade desses parâmetros, 
para o caso da OEII, buscamos no sensoriamento remoto a assinatura de alguma estrutura que possa ser comparada 
com àquela estimada aqui, através da Análise Objetiva. 

O sensoriamento remoto é uma ferramenta poderosíssima de amostragem indireta dos oceanos. Imagens de satélite
são obtidas diariamente em todo o globo, em poucos minutos. Dentro do escopo da circulação oceânica, podemos
citar alguns tipos de produtos oriundos do sensoriamento remoto que nos auxiliam na identificação e interpretação
de feições de escoamento. Imagens da temperatura da superfície do mar são excelentes traçadoras de frentes térmicas
associadas à correntes de contorno, meandros, vórtices e ressurgências, pois tais feições são caracterizadas por
gradientes térmicos marcantes. Na região sudeste brasileira é muito comum a ocorrência de ressurgências costeiras  e
de plataforma, que fazem com que na maior parte do tempo, as águas na plataforma continental sejam mais frias do que 
na porção oceânica adjacente. Com isso, o meandramento de uma corrente de contorno como a CB é capaz de advectar estas
águas mais frias para a porção oceânica, ou injetar águas oceânicas para a plataforma continental. Ambos os mecanismos
fazem com que os meandros sejam reconhecidos através de uma imagem térmica. 
  
Porém, em nossa região de interesse, os meandros e vórtices não possuem assinatura
térmica, devido à ausência de águas mais frias junto a costa e a plataforma continental. Com isso, devemos procurar
outro traçador para identificar estas feições. Um meandro ou vórtice é conhecido pela sua capacidade de 
bombear água em direção à superfície (ciclônico) ou em direção à termoclina (anticiclônico). Em uma região oceânica, 
caracterizada por águas pobres em nutrientes, um bombeamento de águas provenientes da termoclina para o interior
da camada de mistura, por disponibilizar nutrientes na zona eufótica, pode promover ``blooms'' de fitoplâncton.
Estas microalgas são ricas em clorofila, que convenientemente para este estudo, pode ser medida indiretamente
por sensores remotos de satélite. 

Buscamos então imagens de clorofila disponíveis na rede, oriundas do satélite MODIS, que estivessem livre
de contaminação por nuvens e que tenham sido obtidas dentro do período de realização da Operação OEII. Escolhida uma
imagem com essas características, comparamo-a com um campo de função de corrente geostrófica calculada através
da metodologia descrita. Esta comparação está retratada na Figura \ref{fig:modis}. Ao inspecionarmos a figura, 
imediatamente remetemos a atenção ao vórtice anticiclônico centrado em aproximadamente 11.7$^\circ$S e 35$^\circ$W.
Fica evidente a correspondência entre as dimensões da estrutura nos dois métodos. Considerando que a imagem de 
clorofila é instantânea e que os vetores de velocidade são calculados sobre um conjunto termohalino que demorou
mais de um mês para ser obtido, temos aqui mais uma confirmação de que o $l_c$ calculado está fazendo com que as 
estruturas estejam sendo representadas adequadamente. É mais um indício de que não existe o risco de uma 
filtragem excessiva e tampouco uma contaminação por ``aliasing''.  

\begin{figure}%[ht]
 \begin{center}
  \includegraphics[width=10cm,keepaspectratio=true]{/home/rafaelgs/mestrado/proc/clorofila/figuras/clorofila_vgeo.pdf}
 \end{center}
 \vspace{-.25cm}
 \renewcommand{\baselinestretch}{1}
 \caption{\label{fig:modis} \small Sobreposição de vetores de velocidade geostrófica em superfície a
um campo de concentração de clorofila oriundo do satélite MODIS.}
\end{figure}   

Encerramos então esta seção, e com ela, encerramos a apresentação das atividades desenvolvidas durante o Ano II de realização 
desta dissertação. Na próxima seção apresentamos o cronograma para a finalização do trabalho. 

% Cronograma Final %%%%%%%%%%%%%%%%%%%%%%%%%%%%%%%%%%%%%%%%%%%%%%
\newpage
\section{Cronograma Final}\label{sec:cron_final}

\hspace{4mm} Cumpridas e descritas as etapas referentes ao Ano II de trabalho,
apresentamos o cronograma final de trabalho:

\begin{enumerate}

\item[$\checkmark$] Descrição dos resultados obtidos através do mapeamento de função de corrente geostrófica 
absoluta referentes aos dados da Operação OEII.

\item[$\checkmark$] Interpretação, discussão e conclusões.

\item[$\checkmark$] Redação e defesa da dissertação.

\end{enumerate}

% REFERENCIAS %%%%%%%%%%%%%%%%%%%%%%%%%%%%%%%%%%%%%%%%%%%%%%%%%%%%%%%%%%%%%%%%%%
\newpage
\addcontentsline{toc}{section}{Referências Bibliográficas}
\bibliographystyle{tese_hj}
\bibliography{/home/rafaelgs/mestrado/tex/ref_MSc.bib}

\end{document}
