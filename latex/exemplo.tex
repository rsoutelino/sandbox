\documentclass{beamer}
\usetheme{AnnArbor}
\usepackage[utf8x]{inputenc}
\usepackage{bookman} 
\usepackage[T1]{fontenc} 
\usepackage{textcomp}
\usepackage{graphics}

\mode<presentation>

\setbeamertemplate{navigation symbols}{}

\definecolor{darkblue}{RGB}{22,52,136}
\definecolor{midblue}{RGB}{72,118,162}
\definecolor{midgray}{RGB}{127,127,127}
\definecolor{lightgray}{RGB}{204,204,204}
\definecolor{yell}{RGB}{120,10,2}

\setbeamercolor{alerted text}{fg=yell}
\setbeamercolor*{palette primary}{fg=black,bg=midblue}
\setbeamercolor*{palette secondary}{fg=darkblue,bg=lightgray}
\setbeamercolor*{palette tertiary}{bg=darkblue,fg=white}
\setbeamercolor*{palette quaternary}{fg=darkblue,fg=midgray}

\setbeamercolor*{sidebar}{fg=darkblue,bg=orange!75!white}

\setbeamercolor*{palette sidebar primary}{fg=darkblue!10!black}
\setbeamercolor*{palette sidebar secondary}{fg=white}
\setbeamercolor*{palette sidebar tertiary}{fg=darkblue!50!black}
\setbeamercolor*{palette sidebar quaternary}{fg=yellow!10!orange}

%\setbeamercolor*{titlelike}{parent=palette primary}
\setbeamercolor{titlelike}{bg=lightgray,fg=darkblue}
\setbeamercolor{frametitle}{bg=lightgray,fg=darkblue}
\setbeamercolor{frametitle right}{bg=midgray,fg=black}

\setbeamercolor*{separation line}{}
\setbeamercolor*{fine separation line}{}

%\usebackgroundtemplate{\includegraphics[width=\paperwidth]{logoio.png}}

\mode
<all>


\title{Título do Seminário} 
\subtitle{Onde e Quando} 
\author[José da Silva]{Zé da Silva, Ph.D.  \hfill
  \texttt{zedasirva@usp.br}}
\institute[IOUSP]{Instituto Oceanográfico da USP\\
\includegraphics[height=1.5cm]{logolos_310.png}} 
% ----O teu logo aí em cima e o do io ali em baixo -----
\date{}
\mode<beamer>{\logo{\includegraphics[height=1cm]{IO.jpg}}}

\AtBeginSubsection[] {
  \begin{frame}<beamer> \frametitle{Roteiro}
    \tableofcontents[currentsubsection]
  \end{frame}
} \setbeamercovered{highly dynamic}

\begin{document}

% ------------------------------------------------------------
\usebackgroundtemplate{\includegraphics[width=\paperwidth]{header.png}}
\begin{frame}
\vspace{1.5cm}
  \titlepage
\end{frame}
\usebackgroundtemplate{}
% ------------------------------------------------------------
\begin{frame}
  \frametitle{Roteiro}
  \tableofcontents
\end{frame}
% ------------------------------------------------------------
% daqui para frente é por tua conta e risco
% ------------------------------------------------------------
\section{Introdução}
% ------------------------------------------------------------
\subsection{Porque Estudar Oceanografia Física?}
% ------------------------------------------------------------
\begin{frame}{Alguns Fatos Sobre os Oceanos}
  \begin{columns}
    \begin{column}{5.5cm}   
      \begin{itemize}\setlength{\itemsep}{2ex}
      \item A Terra(?) é coberta por 70\% de água.
      \item 97.2\% da água está nos oceanos, 
      \item 1,8\% é gelo, 0,9\% é subterrânea, 
      \item \alert{Rios são só 0,02\%} e vapor 0.001\%.
      \item 1m$^3$ de água armazena mais de 4000 vezes mais calor que
        1m$^3$ de ar.
      \end{itemize}
    \end{column}
    \begin{column}{6.5cm}
\begin{itemize}\setlength{\itemsep}{1ex}
    \item Mais de metade dos humanos vive a < 100km do mar.
    \item A largura típica dos oceanos é da ordem de 5000km.
    \item Pouquíssimas pessoas foram mais de 1 km mar adentro.
    \item A profundidade média dos oceanos é da ordem de 5km.
    \item Pouquíssimos desceram mais de 10m abaixo da superfície.
    \item Mais gente já foi à Lua que à Fossa das Marianas (-10.911m).
    \end{itemize}
    \end{column}
  \end{columns}
\end{frame}
% ------------------------------------------------------------
\end{document}
