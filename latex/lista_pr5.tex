
\documentclass[12pt,portuguese,a4paper,pdftex]{article}
%\usepackage{epstopdf}
\usepackage[pdftex]{hyperref}
%\documentclass[12pt,a4paper]{article}
\usepackage[brazil]{babel}
%\usepackage[latin1]{inputenc} %traduz acentos
\usepackage{natbib}
%\usepackage{epsf,psfig}
\usepackage[dvips]{graphicx}
%\usepackage[square]{natbib}
\usepackage{subfigure}
\usepackage{rotating}
\usepackage{amssymb}
\usepackage{ucs}
\usepackage[utf8x]{inputenc}

\renewcommand{\baselinestretch}{1.5}
\hoffset=-1cm % regula a margem esquerda 
\topmargin -1cm % regula a margem superior
\textwidth 16cm % largura da pagina
\textheight 24.5cm % altura da pagina

\begin{document}

\pagestyle{empty}

\hspace{-6mm}{\large \sc Mestrado em Oceanografia Física - IOUSP\\
IOF 5855 - Métodos e Técnicas de Análise de Dados Quase-Sinóticos em Oceanografia Física\\
\begin{center} Lista Prática \# 5 \end{center} }


\vspace{4cm}

\begin{center}
{\large \bf 
Rafael Guarino Soutelino\\} 
\end{center}

\vspace{4cm}

\begin{center}
Professor:\\
{\bf Ilson Carlos Almeida da Silveira\\}
\end{center}

\vspace{4cm}

\begin{center}
{\small SÃO PAULO, \\
25 de setembro de 2006}
\end{center}


\newpage
\pagenumbering{arabic}
\renewcommand{\baselinestretch}{1}

\pagestyle{empty}

O presente exercício consiste na construção de um modelo quase-geostrófico de 2 camadas otimamente calibrado para a região WESTRAX, seguido de um mapeamento de vorticidade potencial.\\

{\bf Problema 1:} Aplicando-se o primeiro e o segundo esquema de calibração de \textsl{Flierl} (1978), obteve-se valores de salto de densidade $\epsilon$ e prufundidade da picnoclina $-H_1$, sumarizados na tabela abaixo. A Calibração 1 prioriza o efeito dos movimentos de Ekman e a Calibração 2 prioriza efeitos não lineares.\\

\begin{table}

\begin{center}
% use packages: array
\renewcommand{\arraystretch}{0.9}
\renewcommand{\tabcolsep}{5mm}
\begin{tabular}{|c|c|c|}
\hline
 & $\epsilon$ & $H_1$ \\ 
\hline
Calibração 1 & 0.0277 $kg.m^{-3}$ & 238 m \\ 
\hline
Calibração 2 & 3.2135 $kg.m^{-3}$ & 648 m\\
\hline
\end{tabular}
\end{center}
\end{table}

\vspace{.5cm}

A Figura \ref{fig:modos} mostra os perfis verticais dos modos discretos para o modelo de camadas utilizando cada um dos métodos de calibração.\\

\begin{figure}[ht]
\begin{center}
\includegraphics[width=7cm,keepaspectratio=true]{modos_met1.pdf}
\includegraphics[width=7cm,keepaspectratio=true]{modos_met2.pdf}
\end{center}
\vspace{-0.5cm}
\renewcommand{\baselinestretch}{.5}
\caption{\label{fig:modos} \small{Modos verticais discretos.}}
\end{figure}

\newpage

{\bf Problema 2:} Consiste no mapeamento de função de corrente $\psi$ para as camadas do modelo. Para tal, utilizou-se o esquema de calibração que prioriza os efeitos não-lineares. Os modos normais discretos calculados no Problema 1 são utilizados aqui para obter as amplitudes modais dos modos discretos. As velocidades são finalmente calculadas através do produto entre as amplitudes modais e os modos discretos. Estas velocidades são mapeadas via AO para o modo baroclínico do modelo de camadas, e as velocidades na camada superior e inferios são apresentadas respectivamente nas Figuras \ref{fig:cam1} e \ref{fig:cam2}. \\

Através da análise deste campo, percebemos que a estrutura da retrofelxão da CNB é totalmente reproduzida, assim como a frente da corrente junto ao contorno dinâmico adotado. No que se refere a superfície, a aproximação de duas camadas adotada é adequada para descrever a dinâmica da CNB, principalmente se comparamos estes resultados com o campo de função de corrente observada em 50 m. O campo de $\psi$ na camada inferior do modelo representa bem a separação sub-termoclínica da CNB para alimentar a Sub-corrente Norte Equatorial. Esta última feição foi representada no campo de $\psi$ observada e 500 m. \\

\textbf{Problema 3:} Consiste no mapeamento de vorticidade potencial quase-geostrófica $q$. Para tanto, $q$ foi separada em 3 termos: {\bf Vorticidade Relativa, Vorticidade Planetária e Vorticidade de Estiramento}. As distribuições destas 3 quantidades mapeadas via AO, assim como o a distribuição de {\bf Vorticidade Potencial} são mostradas nas Figuras \ref{fig:cam1} e \ref{fig:cam2}, para as duas camadas do modelo.\\

Inspecionando separadamente as 3 componentes da VP quase-geostrófica para a camada superior do modelo, percebemos que o campo de VP segue excencialmente as linhas de vorticidade relativa, ou seja, esta componente, nesse caso, é a mais importante para o mapa final de VP. Para a camada inferior, os valores de vorticidade relativa são bem mais baixos que na camada superior, mas  ainda assim, se mostram predominantes em relação aos mapas de vorticidade de estiramento e vorticidade planetária.\\

A última etapa do Problema 3 é tentar obter informações a respeito da evolução dos fluxos obtidos através da interpretação de mapas de $\psi$ sobrepostos às isolinhas de VP quase-geostrófica, em ambas as camadas, assumindo da presente quantidade. \\

O estado estacionário, nesse caso, seria representado por isolinhas paralelas de $\psi$ e VP, enquanto que regiões onde as respectivas isolinhas se cruzam, representam advecção, ou seja, PV e $\psi$ estão mudando conforme o tempo varia. Estes mapas estão representados no útlimo painel das Figuras \ref{fig:cam1} e \ref{fig:cam2}. Para a camada superior, podemos identificar duas regiões aonde o cruzamento entre isolinhas de PV e $\psi$ é importante, as porções noroeste e sudeste do meandramento principal da retrofelxão. Isto sugere que este vórtice está se destacando da costa. 

\begin{figure}[ht]
\begin{center}
\includegraphics[width=15cm,keepaspectratio=true]{camada1.pdf}
\end{center}
\vspace{-0.5cm}
\renewcommand{\baselinestretch}{.5}
\caption{\label{fig:cam1} \small{Camada Superior.}}
\end{figure}

\begin{figure}[ht]
\begin{center}
\includegraphics[width=15cm,keepaspectratio=true]{camada2.pdf}
\end{center}
\vspace{-0.5cm}
\renewcommand{\baselinestretch}{.5}
\caption{\label{fig:cam2} \small{Camada Inferior.}}
\end{figure}

\end{document}
